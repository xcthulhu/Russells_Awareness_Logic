Neighborhood semantics were originally developed by Dana Scott and Richard
Montague in the early 1970s as a generalization of Kripke semantics \
{\cite{montague_universal_2008,scott_advice_1970}}.  In
{\cite{fagin_belief_1988}}, Halpern and Fagin adapted neighborhood semantics
for reasoning about epistemic agents without logical omniscience.  In this
section will demonstrate how these semantics may be merged with
awareness logic, such that each neighborhood corresponds to the
logical consequences of a different
awareness set.  
%This allows for using logics with neighborhood semantics for
%reasoning about multiple knowledge bases.

Russell does not provide any intuition which motivates the logic we
want to develop here.  
Instead, we rely on a thought experiment of our own invention. Consider
a physicist who has studied both \emph{classical thermodynamics} and
\emph{statistical mechanics}.  The former theory supports a proposition
known as \emph{Boltzman's $H$-theorem}, while the latter supports 
a result known as \emph{Poincar\'{e}'s recurrence theorem}.
  However, she refuses to recognize the \emph{conjunction} of these
  two statements, because Boltzman's
  $H$-theorem and Poincar\'{e}'s recurrence theorem logically exclusive.
  The former implies that spilt milk will not go back into a bottle
  unless work is performed, while the latter holds that given enough
  time spilt milk will eventually \emph{unspill} itself all on its
  own, infinitely often.  As a result, our
  physicists picks and chooses assumptions from both classical
  thermodynamics and statistical mechanics as application
  demands, but she is careful to never \emph{combine} the two theories
  entirely\footnote{This is a point of distinction of neighborhood
    semantics from SJL, since in SJL two awareness sets can
    \emph{always} be combined.}, on pain of contradiction.
  
  We model our thought experiment by first interpreting
  $\mathcal{A}(w)$ not as one
  unified awareness set, as in previous sections, but rather as
  \emph{a collection} of awareness sets.  Each awareness set
  represents a different sets
  of admissible assumptions.  The physicist can derive a
  proposition only if she has a suitable set of assumptions $X$ at her
  disposal.  We model this by stipulating that 
  $\mathbb{M},w\models \Box \phi$ if and only if there is some 
  $X \in \mathcal{A}(w)$ such that $Th(\mathbb{M}) \cup
  X \vdash \phi$.  In the case of the physicist, we imagine there
  being two sets of assumptions in $\mathcal{A}(w)$, $X$ and $Y$,
  representing the assumptions of classical thermodynamics
  and statistical mechanics respectively.  Likewise, we have represent
  Boltzman's $H$-theorem and Poincar\'{e}'s recurrence theorem as
  $\phi$ and $\psi$, respectively.  From the thought experiment we have $Th(\mathbb{M}) \cup
  X \vdash \phi$ and $Th(\mathbb{M}) \cup
  Y \vdash \psi$, while $Th(\mathbb{M})\vdash \phi \wedge \psi \to
  \bot$.  
  It is not hard to imagine cases, however, no such third set
  $Z$ being available in $\mathcal{A}(w)$ such that $Th(\mathbb{M}) \cup
  Z \vdash \phi \wedge \psi$, since the physicist is making an effort
  to be rational.

Unlike other sections, this section is not concerned with \emph{epistemic logic} in the strictest sense. 
% using neighborhood semantics, and another
% extending it with knowledge.  
Reflecting on our thought experiment, it seems impossible to \emph{know} both
Boltzman's $H$-theorem and Poincar\'{e}'s recurrence theorem, as they
form an antimony.  Based on his writings, we feel that Russell would
hold that neither principle is \emph{known}, and that each one of them
constitutes a \emph{probable
  opinion}\cite[pgs. 63--67]{russell_problems_1936} 
that physicists entertain whenever convenient.  
Accordingly, we interpret the neighborhood based logic we 
develop here as a \emph{doxastic logic} of probable opinion.

\begin{definition}
  Let $\Phi$ be a set of letters and define the language $\mathcal{L}_N (\Phi)$ as:
  \[ \phi\ : : =\ p \in \Phi\ |
    \ \bot\ |\ \phi \rightarrow \psi
    \ |\ \Box \phi\ |\ A :
     \phi\  \]
\end{definition}

% Just as previous semantics, ``$\Box \phi$'' is intended to mean that the agent
% has an argument for $\phi$ from some knowledge base.  A novelty we present
% here is that ``$K \phi$'' means that the agent has an argument for $\phi$ from
% some \tmtextit{sound} knowledge base.

\begin{definition}
  \label{neighborhoodmodels}A {\tmstrong{neighborhood model}} $\mathbbm{M}=
  \langle W, V, \mathcal{N}, \mathcal{A} \rangle$ has a neighborhood function
  $\mathcal{N} : W \rightarrow 2^{2^W}$ and a multi-awareness function $\mathcal{A} : W \rightarrow 2^{2^{\mathcal{L}_N (\Phi)}}$
    
  The semantics for $\vDash$ have the following modifications:
  \begin{eqnarray*}
    \mathbbm{M}, w \vDash \Box \phi & \iff & \text{there exists a $U \in
    \mathcal{N}(w)$ where $\mathbbm{M}, v \vDash \phi$ for all $v \in U$}\\
    % \mathbbm{M}, w \vDash K \phi & \text{iff} & \text{there exists a $U \in
    % \mathcal{N}_w$ where $w \in U$ and $\mathbbm{M}, v \vDash \phi$ for all $v
    % \in U$}\\
    \mathbbm{M}, w \vDash A : \phi & \iff & \phi \in \bigcup
    \mathcal{A}(w)
  \end{eqnarray*}
\end{definition}

% Here we modify our previous notion of \tmtextbf{CSQ} and \tmtextbf{SND} to
% match how our semantics are intended for reasoning over multiple bases:

\begin{definition}
  The following defines properties a neighborhood model may make true:
  \begin{descriptiondash}
    \item[NCSQ$_\vdash$] $\mathbbm{M}, w \vDash \Box \phi \text{ iff there exists a set
    $X \in \mathcal{A}(w)$ such that } \tmop{Th} (\mathbbm{M}) \cup X \vdash
    \phi$
    \item[NON-EMPTY] $\mathcal{A}(w)
      \neq \varnothing$ for all worlds $w$
%    \item[NSND] $\mathbbm{M}, w \vDash K \phi \text{ iff there exists a set $X
%    \in \mathcal{A}_w$ such that $\mathbbm{M}, w \vDash X$ and } \tmop{Th}
 %   (\mathbbm{M}) \cup X \vdash \phi$
  \end{descriptiondash}
  % as before, $\vdash$ is any sound logical consequence relation for $\vDash$
  % with \tmtextbf{modus ponens} and \tmtextbf{reflection}
\end{definition}

\begin{table}[h]
\begin{centering}
  \begin{tabular}{l}
    $\vdash \phi \rightarrow \psi \rightarrow \phi$\\
    $\vdash (\phi \rightarrow \psi \rightarrow \chi) \rightarrow (\phi
    \rightarrow \psi) \rightarrow \phi \rightarrow \chi$\\
    $\vdash ((\phi \rightarrow \bot) \rightarrow (\psi \rightarrow \bot))
    \rightarrow \psi \rightarrow \phi$\\
%    $\vdash K \phi \rightarrow \Box \phi$\\
%    $\vdash K \phi \rightarrow \phi$\\
    $\vdash A : \phi \rightarrow \Box \phi$\\
    $\vdash \Box \top$
  \end{tabular}

    \begin{tabular}{lll}
      $\displaystyle\frac{\vdash \phi \rightarrow \psi \ \ \vdash \phi}{\vdash
      \psi}$ & \ \ & $\displaystyle\frac{\vdash \phi \rightarrow \psi}{\vdash \Box \phi
      \rightarrow \Box \psi}$
    \end{tabular}
  \caption{\label{logic3}A Neighborhood Based Doxastic Logic of
    Probable Opinion}
\end{centering}
\end{table}

\begin{theorem}
  Assuming an infinite store of letters $\Phi$, the logic in Table
  \ref{logic3} is sound and weakly complete for neighborhood semantics making
  true $\textup{\tmtextbf{NCSQ}}_\vdash$ and \tmtextbf{NON-EMPTY}.
\end{theorem}

\begin{proof}
  Assume that $\nvdash \psi$, and define the finitary canonical model
  $\mathbbm{M}= \langle W, V, \mathcal{N}, \mathcal{A} \rangle$
  where:
  \begin{itemizedot}
    \item $W \assign \text{the maximally consistent sets of subformulae of
    $\psi$, closed under single negation}$
    \item $V (p) \assign \{v \in W \  | \  p \in v\}$
    % \item $\mathcal{K}_w \assign \{S \in 2^W \  | \  \{v
    % \in W \  | \  \phi \in v\} \subseteq S \  \&
    % \  K \phi \in w\}$
    \item $\mathcal{N}_w \assign \{S \in 2^W \  | \  \llbracket \phi
  \rrbracket \subseteq S \  \&
    \  \Box \phi \in w\}$
    \item $\mathcal{A}(w) \assign \{\{\phi\} \  | \  A :
    \phi \in w\}$
  \end{itemizedot}
  % Note that in this model, there are neighborhoods for both knowledge and
  % belief; more work is necessary to find a model based on this one which
  % conforms to the semantics in Definition \ref{neighborhoodmodels}.  For now,
  % assume that $K$ is just another modality governed by the neighborhoods in
  % $\mathcal{K}_w$.  In both neighborhood functions, 
  Here $\llbracket \phi \rrbracket$ denotes $\{ v \in W\ |\ \phi
\in v\}$.
  
  The proof of the finitary Lindenbaum Lemma is straightforward,
  although the Truth Theorem is a little less routine. The proof
  proceeds by induction.  
  We only show one direction for the step for $\Box \chi$.
  Assume $\mathbbm{M}, w \vDash \Box \chi$, where the Truth Lemma has been proven to
  hold for $\chi$; we have to show $\Box \chi \in w$.  There must be some
  $S \in \mathcal{N}(w)$ such that $\mathbb{M},v \models \chi$ for all
  $v \in S$. By construction there is some $\phi$ where $\Box \phi \in
  w$ such that $\llbracket \phi \rrbracket \subseteq S$.  We argue that $\vdash \phi
  \rightarrow \chi$; for if not then by the finitary Lindenbaum lemma there is
  some $u \in W$ where $\phi \in u$ and $\chi \nin u$.  Hence $u \in S$ and
  $\mathbbm{M}, u \nvDash \chi$, a contradiction.  Since $\vdash \phi
  \rightarrow \chi$ then we know that $\vdash \Box \phi \rightarrow \Box \chi$
  by our rules, which means that $\Box \chi \in w$ by maximality.
  
  By the Truth Theorem and the Finitary Lindenbaum lemma, we have that
  there is some $w \in W$ where
  $\mathbbm{M}, w \nvDash \psi$.  Moreover, $\mathbbm{M}$ makes true a number of
  properties:
  \begin{enumeratenumeric}
    \item $W$ and $\mathcal{A}_v$ are finite for all $v \in W$, and $\psi \in
    \bigcup \mathcal{A}(w)$ only if $\psi$ is a subformula of $\phi$   
   % \item $\mathcal{K}_w \subseteq \mathcal{N}_w$
%    \item If $S \in \mathcal{K}_w$ then $w \in S$
    \item $\mathbbm{M}, v \vDash A : \phi \rightarrow \Box \phi$ for all $v
    \in W$
    \item $W \in \mathcal{N}(w)$ for all $w \in W$
 %   \item $\mathcal{N}_w$ and $\mathcal{K}_w$ are closed under supersets
  \end{enumeratenumeric}
  
  % As in the proof of Theorem \ref{completeness1}, we use a bisimulation to
  % move to a model where property (3) is strengthened to (3$'$): for all $S \in
  % \mathcal{N}_w$, $w \in S$ if and only if $S \in \mathcal{K}'_w$.  We defer
  % the reader to {\cite{hansen_bisimulation_2007,pauly_bisimulation_1999}} for
  % details regarding bisimulation in neighborhood semantics.  Define
  % $\mathbbm{M}' \assign \langle W', V', \mathcal{K}', \mathcal{N}',
  % \mathcal{A}' \rangle$ such that:
  % \begin{itemizedot}
  %   \item $W' \assign W \uplus W$ wher.  $l, r$ its associated canonical
  %   injections and $\theta (v_l) \assign \theta (v_r) \assign v$ is a
  %   left-inverse of $l, r$
    
  %   \item $V' (p) \assign \{v_l, v_r \  | \  v \in V
  %   (p)\}$
    
  %   \item $\mathcal{K}' (v_i) \assign \{S \  | \  v_i \in
  %   S \  \& \  \theta [S] \in \mathcal{K_v} \}$ where $i =
  %   l, r$
    
  %   \item $\mathcal{N}' (v_i) \assign \mathcal{K}' (v_i) \cup \{S \ 
  %   | \  v_i \nin S \  \& \  \theta [S] \in
  %   \mathcal{N_v} \}$ where $i = l, r$
    
  %   \item $\mathcal{A}' (v_l) \assign \mathcal{A}' (v_r) \assign \mathcal{A}
  %   (v)$
  % \end{itemizedot}
  
  
  % If we let $Z \assign \{(v, v_r), (v, v_l) \  | \  v \in
  % W\}$, it is straightforward to verify that $Z$ is a neighborhood
  % bisimulation.  Hence $\mathbbm{M}, w_l \nvDash \psi$.  Along with (3$'$),
  % this model makes true (1), (2), (4).  In fact, in light of ($3'$), we can
  % see that this model does not need $\mathcal{K}'$, and conforms to the
  % semantics in Definition \ref{neighborhoodmodels}.{\hspace*{\fill}}
  
  
  Let $\Lambda$ be the proposition letters occurring in
  $\psi$ and define an injection $\iota : W \hookrightarrow \Phi \backslash
  \Lambda$, assigning a nominal to every world in $\mathbbm{M}'$.  Define
  $\mathbbm{M}' \assign \langle W', V', \mathcal{N}', \mathcal{A}'
  \rangle$ such that:
  
  \begin{center}
    \begin{tabular}{lllll}
      $\bullet$ & $W' \assign W$ &  & $\bullet$ & $V' (p) \assign \left\{
      \begin{array}{ll}
        V (p) & p \in \Lambda\\
        \{v\} & p = \iota (v)\\
        \varnothing & o / w
      \end{array} \right.$\\
      $\bullet$ & $\mathcal{N}'(w) \assign \mathcal{N}(w)$ & {\hspace{3em}} &  $\bullet$ & $\mathcal{A}'(w) \assign \{\delta_w S \  |
      \  S \in \mathcal{N}'(w) \}$
    \end{tabular}
  \end{center}
  Where $\delta_w S \assign \{ \phi \in
  \bigcup \mathcal{A}(w) \  | \ S \subseteq \left\llbracket \phi
  \right\rrbracket \} \cup \{\neg \iota (u) \  |
  \  u \nin S\}$.
  
  By induction, this model agrees with $\mathbbm{M}$ on all subformulae of $\psi$, hence
  $\mathbbm{M}', w \nvDash \phi$ for some $w\in W'$.
  
  We first establish that \textbf{NON-EMPTY} is true for
  $\mathbb{M}$. We must show for every $v \in W'$ that $\mathcal{A}'(v)
  \neq \varnothing$.  We know by construction that $W' \in
  \mathcal{N}'(v)$, since $W \in \mathcal{N}(u)$ for all of the worlds
  $u$ in model $\mathbb{M}$.  Hence by construction $\delta_v W' \in
  \mathcal{A}'(v)$ by definition.

  All that is left is to show $\text{\tmtextbf{NCSQ}}_\vdash$ holds,
  and we will use the logic we are verifying completeness for. The way we have
  defined $\mathbb{M}'$ gives rise to certain properties:
  \begin{enumerateroman}
    \item For all words $v$ and all $X \in \mathcal{A}'(v)$, $X$ is finite and  $X = \delta_v S$ for some $S \in \mathcal{N}'(v)$
    \item For all $v$ and all $S \in \mathcal{N}'(v)$, $u \in S$ if and only if
    $\mathbbm{M}', u \vDash \bigwedge \delta_v S$
    \item The logic presented in Table \ref{logic3} is sound for
    $\mathbbm{M}'$
  \end{enumerateroman}

  Using the deduction theorem for $\vdash$ along with these
  properties, we may establish  $\text{\tmtextbf{NCSQ}}_\vdash$ via a familiar line of reasoning:
  
  \begin{align*}
    \text{$\exists X \in \mathcal{A}'(v) \text{ s.t. } \tmop{Th}
    (\mathbbm{M}') \cup X \vdash \phi$} & {\Longleftrightarrow}\text{$\exists
    S \in \mathcal{N'}(v) \text{ s.t. } \tmop{Th} (\mathbbm{M}') \cup
    \delta_v S \vdash \phi$}\\
    & {\Longleftrightarrow}\text{$\exists S \in \mathcal{N}'(v) \text{ s.t. }
    \tmop{Th} (\mathbbm{M}') \vdash \bigwedge \delta_v S \rightarrow
    \phi$}\\
    & {\Longleftrightarrow}\text{$\exists S \in \mathcal{N}'(v) \text{ s.t. }
    \bigwedge \delta_v S \rightarrow \phi \in \tmop{Th} (\mathbbm{M}')$}\\
    % & {\Longleftrightarrow}\text{$\exists S \in \mathcal{N}'(v) \text{ s.t. }
    % \left\llbracket \bigwedge \delta_v S \right\rrbracket \subseteq
    % \left\llbracket \phi \right\rrbracket$}\\
    % & {\Longleftrightarrow}\text{$\exists S \in \mathcal{N}'(v) \text{ s.t.
    % $\forall u \in W$ if } \mathbbm{M}, u \vDash \bigwedge \delta_v S  \phi$}\\
    & {\Longleftrightarrow}\text{$\exists S \in \mathcal{N}'(v) \text{ s.t.
    $\forall u \in W$ if } \mathbbm{M}, u \vDash \bigwedge \delta_v S \text{
    then } \mathbbm{M}, u \vDash \phi$}\\
    & {\Longleftrightarrow}\text{$\exists S \in \mathcal{N}'(v) \text{ s.t. }
    \mathbbm{M}, u \vDash \phi \text{ for all } u \in S$}\\
    & {\Longleftrightarrow}{\mathbbm{M}},v{\vDash}\Box{\phi}
  \end{align*}
  This completes the proof.
\end{proof}
%%% Local Variables: 
%%% mode: latex
%%% TeX-master: "paper"
%%% End: 


% \subsection{Logic for Knowledge and Belief}
% In this section, we conservatively extend the neighborhood semantics
% based logic of belief to account for knowledge as well. According to
% Russell, belief which is not knowledge falls into two categories:
% \emph{probable opinion} and \emph{error
% As in previous sections, we model knowledge as being minimally \emph{true}, however
% we place no further constraints on it.  
%%% Local Variables: 
%%% mode: latex
%%% TeX-master: "paper"
%%% End: 