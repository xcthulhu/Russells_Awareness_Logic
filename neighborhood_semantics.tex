Neighborhood semantics were originally developed by Dana Scott and Richard
Montague in the early 1970s as a generalization of Kripke semantics \
{\cite{montague_universal_2008,scott_advice_1970}}.  In
{\cite{fagin_belief_1988}}, Halpern and Fagin adapted neighborhood semantics
for reasoning about epistemic agents without logical omniscience.  In this
section will demonstrate how these semantics may be modified so that
neighborhoods corresponds to the logical consequences of a different knowledge
bases.  This allows for using logics with neighborhood semantics for
reasoning about multiple knowledge bases.  A point of distinction of
neighborhood semantics from JL is that it may be possible that two distinct
knowledge bases cannot be merged by the agent.

\begin{definition}
  Let $\Phi$ be a set of letters and define the language $\mathcal{L}_N (\Phi)$ as:
  \[ \phi \hspace{1em} : : = \hspace{1em} p \in \Phi \hspace{1em} |
     \hspace{1em} \bot \hspace{1em} | \hspace{1em} \phi \rightarrow \psi
     \hspace{1em} | \hspace{1em} \Box \phi \hspace{1em} | \hspace{1em} A :
     \phi \hspace{1em} | \hspace{1em} K \phi \]
\end{definition}

Just as previous semantics, ``$\Box \phi$'' is intended to mean that the agent
has an argument for $\phi$ from some knowledge base.  A novelty we present
here is that ``$K \phi$'' means that the agent has an argument for $\phi$ from
some \tmtextit{sound} knowledge base.



\begin{definition}
  \label{neighborhoodmodels}A {\tmstrong{neighborhood model}} $\mathbbm{M}=
  \langle W, V, \mathcal{N}, \mathcal{A} \rangle$ has a neighborhood function
  $\mathcal{N} : W \rightarrow 2^{2^W}$ and a multi-awareness function
  $\mathcal{A} : W \rightarrow 2^{2^{\mathcal{L}_N (\Phi)}}$
  
  
  
  The semantics for $\vDash$ have the following modifications:
  \begin{eqnarray*}
    \mathbbm{M}, w \vDash \Box \phi & \text{iff} & \text{there exists a $U \in
    \mathcal{N}_w$ where $\mathbbm{M}, v \vDash \phi$ for all $v \in U$}\\
    \mathbbm{M}, w \vDash K \phi & \text{iff} & \text{there exists a $U \in
    \mathcal{N}_w$ where $w \in U$ and $\mathbbm{M}, v \vDash \phi$ for all $v
    \in U$}\\
    \mathbbm{M}, w \vDash A : \phi & \text{iff} & \phi \in \bigcup
    \mathcal{A_w}
  \end{eqnarray*}
\end{definition}

Here we modify our previous notion of \tmtextbf{CSQ} and \tmtextbf{SND} to
match how our semantics are intended for reasoning over multiple bases:

\begin{definition}
  The following defines properties a neighborhood model may make true:
  
  \begin{descriptiondash}
    \item[NCSQ] $\mathbbm{M}, w \vDash \Box \phi \text{ iff there exists a set
    $X \in \mathcal{A}_w$ such that } \tmop{Th} (\mathbbm{M}) \cup X \vdash
    \phi$
    
    \item[NSND] $\mathbbm{M}, w \vDash K \phi \text{ iff there exists a set $X
    \in \mathcal{A}_w$ such that $\mathbbm{M}, w \vDash X$ and } \tmop{Th}
    (\mathbbm{M}) \cup X \vdash \phi$
  \end{descriptiondash}
  
  as before, $\vdash$ is any sound logical consequence relation for $\vDash$
  with \tmtextbf{modus ponens} and \tmtextbf{reflection}
\end{definition}

\begin{table}[h]
  \begin{tabular}{l}
    $\vdash \phi \rightarrow \psi \rightarrow \phi$\\
    $\vdash (\phi \rightarrow \psi \rightarrow \chi) \rightarrow (\phi
    \rightarrow \psi) \rightarrow \phi \rightarrow \chi$\\
    $\vdash ((\phi \rightarrow \bot) \rightarrow (\psi \rightarrow \bot))
    \rightarrow \psi \rightarrow \phi$\\
    $\vdash K \phi \rightarrow \Box \phi$\\
    $\vdash K \phi \rightarrow \phi$\\
    $\vdash A : \phi \rightarrow \Box \phi$\\
    \\
    \begin{tabular}{llll}
      $\frac{\vdash \phi \rightarrow \psi \hspace{4em} \vdash \phi}{\vdash
      \psi}$ & {\hspace{6em}}$\frac{\vdash \phi \rightarrow \psi}{\vdash \Box
      \phi \rightarrow \Box \psi}$ & {\hspace{6em}} & $\frac{\vdash \phi
      \rightarrow \psi}{\vdash K \phi \rightarrow K \psi}$
    \end{tabular}
  \end{tabular}
  \caption{\label{logic3}A Neighborhood Logic for \tmtextbf{NCSQ} and
  {\tmstrong{NSND}}}
\end{table}

\begin{theorem}
  Assuming an infinite store of letters $\Phi$, the logic in Table
  \ref{logic3} is sound and weakly complete for neighborhood semantics making
  true \tmtextbf{NCSQ} and \tmtextbf{NSND}.
\end{theorem}

\begin{proof}
  Assume that $\nvdash \psi$, and define the finitary canonical model
  $\mathbbm{M}= \langle W, V, \mathcal{K}, \mathcal{N}, \mathcal{A} \rangle$
  where:
  \begin{itemizedot}
    \item $W \assign \text{the maximally consistent sets of subformulae of
    $\psi$, closed under single negation}$
    
    \item $V (p) \assign \{v \in W \hspace{1em} | \hspace{1em} p \in v\}$
    
    \item $\mathcal{K}_w \assign \{S \in 2^W \hspace{1em} | \hspace{1em} \{v
    \in W \hspace{1em} | \hspace{1em} \phi \in v\} \subseteq S \hspace{1em} \&
    \hspace{1em} K \phi \in w\}$
    
    \item $\mathcal{N}_w \assign \{S \in 2^W \hspace{1em} | \hspace{1em} \{v
    \in W \hspace{1em} | \hspace{1em} \phi \in v\} \subseteq S \hspace{1em} \&
    \hspace{1em} \Box \phi \in w\}$
    
    \item $\mathcal{A}_w \assign \{\{\phi\} \hspace{1em} | \hspace{1em} A :
    \phi \in w\}$
  \end{itemizedot}
  Note that in this model, there are neighborhoods for both knowledge and
  belief; more work is necessary to find a model based on this one which
  conforms to the semantics in Definition \ref{neighborhoodmodels}.  For now,
  assume that $K$ is just another modality governed by the neighborhoods in
  $\mathcal{K}_w$.  In both neighborhood functions, our idea is that the
  neighborhoods of $w$ in are supersets $S \supseteq \left\llbracket \phi
  \right\rrbracket$ where $\Box \phi \in w$ or $K \phi \in w$, respectively.
  
  
  
  There are steps of the inductive proof of the Truth Lemma for this structure
  which are not obvious nor obviously documented elsewhere.  Assume
  $\mathbbm{M}, w \vDash \Box \chi$, where the Truth Lemma has been proven to
  hold for $\chi$; we must show $\Box \chi \in w$.  By our assumption there
  is some $\phi$ and $S$ where $\Box \phi \in w$ and $S \supseteq \{v \in W
  \hspace{1em} | \hspace{1em} \phi \in v\}$.  We argue that $\vdash \phi
  \rightarrow \chi$; for if not then by the finitary Lindenbaum lemma there is
  some $u \in W$ where $\phi \in u$ and $\chi \nin u$.  Hence $u \in S$ and
  $\mathbbm{M}, u \nvDash \chi$, a contradiction.  Since $\vdash \phi
  \rightarrow \chi$ then we know that $\vdash \Box \phi \rightarrow \Box \chi$
  by our rules, which means that $\Box \chi \in w$ by maximality.
  
  
  
  By the Lindenbaum Lemma we have that there is some $w \in W$ where
  $\mathbbm{M}, w \nvDash \psi$.  $\mathbbm{M}$ makes true a number of
  properties:
  \begin{enumeratenumeric}
    \item $W$ and $\mathcal{A}_v$ are finite for all $v \in W$, and $\psi \in
    \bigcup \mathcal{A_w}$ only if $\psi$ is a subformula of $\phi$
    
    \item $\mathcal{K}_w \subseteq \mathcal{N}_w$
    
    \item If $S \in \mathcal{K}_w$ then $w \in S$
    
    \item $\mathbbm{M}, v \vDash A : \phi \rightarrow \Box \phi$ for all $v
    \in W$
    
    \item $\mathcal{N}_w$ and $\mathcal{K}_w$ are closed under supersets
  \end{enumeratenumeric}
  
  
  As in the proof of Theorem \ref{completeness1}, we use a bisimulation to
  move to a model where property (3) is strengthened to (3$'$): for all $S \in
  \mathcal{N}_w$, $w \in S$ if and only if $S \in \mathcal{K}'_w$.  We defer
  the reader to {\cite{hansen_bisimulation_2007,pauly_bisimulation_1999}} for
  details regarding bisimulation in neighborhood semantics.  Define
  $\mathbbm{M}' \assign \langle W', V', \mathcal{K}', \mathcal{N}',
  \mathcal{A}' \rangle$ such that:
  \begin{itemizedot}
    \item $W' \assign W \uplus W$ wher.  $l, r$ its associated canonical
    injections and $\theta (v_l) \assign \theta (v_r) \assign v$ is a
    left-inverse of $l, r$
    
    \item $V' (p) \assign \{v_l, v_r \hspace{1em} | \hspace{1em} v \in V
    (p)\}$
    
    \item $\mathcal{K}' (v_i) \assign \{S \hspace{1em} | \hspace{1em} v_i \in
    S \hspace{1em} \& \hspace{1em} \theta [S] \in \mathcal{K_v} \}$ where $i =
    l, r$
    
    \item $\mathcal{N}' (v_i) \assign \mathcal{K}' (v_i) \cup \{S \hspace{1em}
    | \hspace{1em} v_i \nin S \hspace{1em} \& \hspace{1em} \theta [S] \in
    \mathcal{N_v} \}$ where $i = l, r$
    
    \item $\mathcal{A}' (v_l) \assign \mathcal{A}' (v_r) \assign \mathcal{A}
    (v)$
  \end{itemizedot}
  
  
  If we let $Z \assign \{(v, v_r), (v, v_l) \hspace{1em} | \hspace{1em} v \in
  W\}$, it is straightforward to verify that $Z$ is a neighborhood
  bisimulation.  Hence $\mathbbm{M}, w_l \nvDash \psi$.  Along with (3$'$),
  this model makes true (1), (2), (4).  In fact, in light of ($3'$), we can
  see that this model does not need $\mathcal{K}'$, and conforms to the
  semantics in Definition \ref{neighborhoodmodels}.{\hspace*{\fill}}
  
  
  
  As in previous proofs, let $\Lambda$ be the proposition letters occuring in
  $\psi$ and define an injection $\iota : W' \hookrightarrow \Phi \backslash
  \Lambda$, assigning a nominal to every world in $\mathbbm{M}'$.  Define
  $\mathbbm{M}'' \assign \langle W'', V'', \mathcal{N}'', \mathcal{A}''
  \rangle$ such that:
  
  
  
  \begin{center}
    \begin{tabular}{lllll}
      $\bullet$ & $W'' \assign W'$ &  & $\bullet$ & $V'' (p) \assign \left\{
      \begin{array}{ll}
        V' (p) & p \in \Lambda\\
        \{v\} & p = i (v)\\
        \varnothing & o / w
      \end{array} \right.$\\
      $\bullet$ & $\mathcal{N}''_w \assign \mathcal{N}'_w$ & {\hspace{3em}} &
      $\bullet$ & $\mathcal{A}''_v \assign \{\partial_v S \hspace{1em} |
      \hspace{1em} S \in \mathcal{N}'_w \}$
    \end{tabular}
  \end{center}
  
  
  
  
  
  Where $\partial_v S \assign \{\phi \hspace{1em} | \hspace{1em} \phi \in
  \bigcup \mathcal{A}_v', \text{$\phi$ is a subformula of $\psi$ and}
  \hspace{1em} S \subseteq \left\llbracket \phi
  \right\rrbracket^{\mathbbm{M}'} \} \cup \{\neg \iota (u) \hspace{1em} |
  \hspace{1em} u \nin S\}$.
  
  As before, this model agrees with $\mathbbm{M}'$ on all subformulae of
  $\psi$, using the semantics in Definition \ref{neighborhoodmodels}, hence
  $\mathbbm{M}'', w_l \nvDash \phi$. \
  
  
  
  All that is left is to show \tmtextbf{NCSQ} and \tmtextbf{NSND} from three
  new properties:
  \begin{enumerateroman}
    \item For all words $v$ and all $X \in \mathcal{A}_v''$, $X$ is finite and
  .  $X = \partial_v S$ for some $S \in \mathcal{N}_v''$
    
    \item For all $v$ and all $S \in \mathcal{N}_v$, $u \in S$ if and only if
    $\mathbbm{M}'', u \vDash \bigwedge \partial_v S$
    
    \item The logic presented in Table \ref{logic3} is sound for
    $\mathbbm{M}''$
  \end{enumerateroman}
  Along with the deduction theorem for $\vdash$, these properties establish
  \tmtextbf{NCSQ} for this model using the logic of Table \ref{logic3} itself:
  
  \begin{align*}
    \text{$\exists X \in \mathcal{A''_v} \text{ s.t. } \tmop{Th}
    (\mathbbm{M}'') \cup X \vdash \phi$} & {\Longleftrightarrow}\text{$\exists
    S \in \mathcal{N''_v} \text{ s.t. } \tmop{Th} (\mathbbm{M}'') \cup
    \partial_v S \vdash \phi$}\\
    & {\Longleftrightarrow}\text{$\exists S \in \mathcal{N''_v} \text{ s.t. }
    \tmop{Th} (\mathbbm{M}'') \vdash \bigwedge \partial_v S \rightarrow
    \phi$}\\
    & {\Longleftrightarrow}\text{$\exists S \in \mathcal{N''_v} \text{ s.t. }
    \bigwedge \partial_v S \rightarrow \phi \in \tmop{Th} (\mathbbm{M}'')$}\\
    & {\Longleftrightarrow}\text{$\exists S \in \mathcal{N''_v} \text{ s.t. }
    \left\llbracket \bigwedge \partial_v S \right\rrbracket \subseteq
    \left\llbracket \phi \right\rrbracket$}\\
    & {\Longleftrightarrow}\text{$\exists S \in \mathcal{N''_v} \text{ s.t.
    $\forall u \in W$ if } \mathbbm{M}, u \vDash \bigwedge \partial_v S \text{
    then } \mathbbm{M}, u \vDash \phi$}\\
    & {\Longleftrightarrow}\text{$\exists S \in \mathcal{N''_v} \text{ s.t. }
    \mathbbm{M}, u \vDash \phi \text{ for all } u \in S$}\\
    & {\Longleftrightarrow}{\mathbbm{M}},v{\vDash}\Box{\phi}
  \end{align*}
  
  \tmtextbf{NSND} follows from \tmtextbf{NCSQ} and the semantics of Definition
  \ref{neighborhoodmodels}.
\end{proof}

It is not too hard to identify the fragment of this logic that governs
knowledge alone.  The fragment given in Table \ref{logic4} has special
significance to us, since it governs the notion of knowledge presented in
{\S}\ref{quantifying}.

\begin{table}[h]
  \begin{tabular}{l}
    $\vdash \phi \rightarrow \psi \rightarrow \phi$\\
    $\vdash (\phi \rightarrow \psi \rightarrow \chi) \rightarrow (\phi
    \rightarrow \psi) \rightarrow \phi \rightarrow \chi$\\
    $\vdash ((\phi \rightarrow \bot) \rightarrow (\psi \rightarrow \bot))
    \rightarrow \psi \rightarrow \phi$\\
    $\vdash K \phi \rightarrow \phi$\\
    \\
    \begin{tabular}{llll}
      $\frac{\vdash \phi \rightarrow \psi \hspace{4em} \vdash \phi}{\vdash
      \psi}$ &  &  & $\frac{\vdash \phi \rightarrow \psi}{\vdash K \phi
      \rightarrow K \psi}$
    \end{tabular}
  \end{tabular}
  \caption{\label{logic4}A knowledge-only neighborhood logic for
  \tmtextbf{NCSQ} and {\tmstrong{NSND}}}
\end{table}

\begin{proposition}
  The knowledge-only logic presented governs the $\Box$-free fragment of the
  logic in Table \ref{logic3}
\end{proposition}

This concludes our analysis of knowledge bases using neighborhood
semantics.
%%% Local Variables: 
%%% mode: latex
%%% TeX-master: "paper"
%%% End: 
