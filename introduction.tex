The modern tradition in epistemic logic is to assume knowledge modalities
conform to the $S 5$ axiom schema.  As noted in
{\cite{halpern_set-theoretic_1999,rubinstein_modeling_1998}}, the semantics of
$S 5$ knowledge correspond exactly to partitioning a collection of situations
into \tmtextit{information sets}, which is the traditional approach in game
theory and decision theory.  While it is not commonly acknowledged in
epistemic logic, economists and philosophers accept that traditional decision
theory is externalist and behaviorist in nature
\footnote{An early essay by
Amartya Sen on the philosophical foundations of traditional decision theory
makes the behaviorist reading of decision theory clear
\cite{sen_behaviour_1973}.  Kaushik Basu also discusses the behaviorist
nature of decision theory {\cite[pgs. 53--54]{basu_revealed_1980}}. 
Finally, Donald Davidson appeals to decision theory to motivate an externalist
epistemology in \cite{davidson_could_1995}.}.

In {\cite{benthem_reflections_1991}}, Johan van Benthem proposed
a research program to find logics for explicit knowledge, providing the first
suggestion of an internalist perspective on epistemic logic.  Subsequently,
Sergei Artemov and Elena Nogina proposed a logic of explicit justification,
which has come to be known as \tmtextit{Justification Logic} (JL)
{\cite{artemov_introducing_2005}}, based on Artemov's \tmtextit{Logic of
Proofs} {\cite{artemov_logic_1994}}.  While the original semantics of
JL were based on interpretability into Peano Arithmetic, Melvin Fitting proposed Kripke
semantics for JL in {\cite{fitting_logic_2005}}.  Recently, Fernando
Vel\'azquez-Quesada and Johan van Benthem have developed a simpler framework
for explicit epistemics in {\cite{benthem_inference_2009}}, entitled the
\tmtextit{Dynamics of Awareness}.  This work was based on Joseph Halpern and
Ronald Fagin's original \tmtextit{Awareness Logic} {\cite{fagin_belief_1987}}.



In this essay we repurpose various externalist logics to take on an
internalist reading.  The concept of a \tmtextit{knowledge base}, from which
beliefs may be implicitly deduced, will play a crucial role in our discussion.
\ We propose this as an avenue for representing foundationalist perspectives
on epistemology in epistemic logic.  Our philosophical motivation is taken
from two sources.  The first is Vincent Hendricks in
{\cite{hendricks_mainstream_2006}}, where he characterizes the principal of
{\tmem{logical omniscience}} for implicit knowledge in epistemic
logic{\footnote{We have modified Hendricks' notation here slightly to match
our own.}}:

\begin{quote}
  {\tmem{Whenever an agent $\Xi$ knows all of the formulae in $\mathcal{A}$,
  and $\phi$ follows logically from $\mathcal{A}$, then $\Xi$ [implicitly]
  knows $\phi$.}}
\end{quote}

We will design our semantics such that ``$\Box \phi$'' may be equated with
``$\phi$ follows logically from a knowledge base $\mathcal{A}$,'' which is
sometimes written as $\phi \in \tmop{Cn} ( \mathcal{A})$ in the artificial
intelligence literature.  Our second inspiration comes from what Hilary
Kornblith calls the \tmtextit{the arguments on paper thesis}
{\cite{kornblith_beyond_1980}}, which he feels characterizes apsychological,
internalist theories of knowledge:

\begin{quote}
  Let us suppose that, for any person, it is possible, at least in principle,
  to list all of the propositions that person believes.  The
  arguments-on-paper thesis is just the view that a person has a justified
  belief that a particular proposition is true just in case that proposition
  appears on the list of propositions that person believes, and either it
  requires no argument, or a {\tmem{good argument}} can be given for it which
  takes as premises certain other propositions on the list.
\end{quote}

Kornblith asserts that foundationalism and coherentism vie for accounts of a
``good argument'' in the above thesis.  He provides an extensive bibliography
citing proposals for this principle by a number of 20th century
epistemologists, including figures such as A.J. Ayer and C.I. Lewis.  The
rest of Kornblith's paper is devoted to attacking this view and proposing a
form of naturalized epistemology; we will not address this debate here,
however.

We adopt a arguments-on-paper-thesis perspective on epistemic logic in this
paper.  We consider a \tmtextit{good argument} to be a logical derivation
from propositions present in a knowledge base in this setting.  \
Special attention will be given to \tmtextit{sound derivations}, which will be
thought of as a form of knowledge.  In \emph{Awareness Logic}, we will interpret
awareness of a formula as membership in a knowledge base.  We prove
completeness for basic awareness logic and a hybrid logic extension.  Logics
of multiple knowledge bases are also presented: a simplified form of
Justification Logic, a logic with neighborhood semantics, an a logic with
modalities for quantifying over knowledge bases.  We conclude with an
application to naturalized epistemology found in the psychological
literature.
%%% Local Variables: 
%%% mode: latex
%%% TeX-master: "paper"
%%% End: 
