Modern epistemic logic is commonly acknowledged to begin with
Hintikka's \emph{Knowledge and Belief}
\cite{hintikka_knowledge_1969}. While Hintikka was steeped in
profound philosophical insights, the introduction of epistemic logic
bifurcated 20th century research into epistemology.  A number of
logicians and philosophers have acknowledged this rift, most notably
Vincent Hendricks \cite{hendricks_wheres_2006,
  hendricks_mainstream_2006}.

The purposes of this essay is to try to close the gap between these
traditions, by showing that awareness logics are suitable for
reasoning about the form of foundationalism proposed
by Bertrand Russell in \emph{The Problems of Philosophy}
\cite{russell_problems_1936}.  We begin by reviewing Russell's
theory, and contrasting it with modern perspectives.  In particular,
we illustrate that the modern concepts of \emph{explicit} and
\emph{implicit}, due to Levesque\footnote{While we will refer often to
  the terms coined by Levesque in this essay, we will not use his
  proposed semantics.  We will instead prefer variations on
  \emph{awareness logic}, originally developed by Fagin and Halpern \cite{fagin_belief_1988}.}\cite{levesque_logic_1984}, knowledge in epistemic logic are reinventions of
Russell's distinction between \emph{immediate} and \emph{derivative}
knowledge.  We propose a novel property of awareness models which
directly reflects Russell's foundationalist epistemology.  The
remainder of this paper is then devoted to exploring how awareness
logics can be designed to express our this foundationalist property.

%%% Local Variables: 
%%% mode: latex
%%% TeX-master: "paper"
%%% End: 
