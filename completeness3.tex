\begin{proof}
  As in the previous proofs, we only show completeness.  Assume that $\nvdash
  \psi$.  Let $\Xi \subseteq \tau (\Pi)$ be all of the subterms occurring in
  $\psi$. 

  We assume that $\Xi$ is non-empty, since otherwise we may obtain the
  theorem by using the usual completeness theorem for classical propositional
  logic to find a valuation $V$, which we would use in trivial a
  Kripke model $\mathbb{T}$ with a
  single world $w$ corresponding to that valuation, such that
  $\mathbb{T},w\nmodels \psi$. 

Let $\Sigma_0$ be the subformulae of $\psi$.  Define:
  
  \begin{align*}
    {\Sigma}_1 &
    {\assign}{\Sigma}_0{\cup}\{\Box_Y{\phi}{\ }|{\ }\Box_X{\phi}{\in}{\Sigma}_0{\ }\&{\ }Y{\in}{\Xi}\}\\
    {\Sigma}_2 &
    {\assign}{\Sigma}_1{\cup}\{{\neg}{\phi}{\ }|{\ }{\phi}{\in}{\Sigma}_1\}
  \end{align*}
  
  As in the proof of Theorem \ref{completeness2}, it is simple to verify that
  $\Sigma_2$ is closed under subformulae and single negations.  Let
  $\mathbbm{M} \assign \langle W, V, R, \mathcal{A} \rangle$ be the finite
  canonical model formed of maximally consistent subsets of $\Sigma_2$, where
  $V$ and $\mathcal{A}$ are defined as usual, and $R : \Xi \rightarrow 2^{W
  \times W}$ is defined by:
  \[ w R_X v \Longleftrightarrow \{\phi \  | \  \Box_X
     \phi \in w\} \subseteq v \]
  As in previous constructions we may prove the usual Lindenbaum and Truth
  Lemmas an use them obtain a world $w \in W$ where $\mathbbm{M}, w \nvDash
  \psi$.
   
 $\mathbbm{M}$ makes true various properties:
  \begin{enumeratenumeric}
     \item \label{fin}$W$ and $\mathcal{A}_X (v)$ are finite, and $\phi \in
    \mathcal{A}_X (v)$ only if $\phi \in \Sigma_2$
     \item All of the relations $R_X$ in $\mathbb{M}$ are reflexive    
    \item If $\mathbbm{M}, v \vDash X : \phi$ then $\mathbbm{M}, v \vDash
    \Box_X \phi$
    \item \label{union}For all $X \oplus Y \in \Xi$, $\mathcal{A}_{X \oplus
    Y}(v) = \mathcal{A}_X(v) \cup \mathcal{A}_Y (v)$
    \item \label{sub}For all $X, Y \in \Xi$, if $\twonotes (X) \subseteq
    \twonotes (Y)$ then $R_Y \subseteq R_X$
  \end{enumeratenumeric}
  Where the operator $\twonotes : \tau (\Pi) \rightarrow 2^{\Pi}$ is
  defined as follows:
  \[ \twonotes (X) \assign \{t \in \Pi \  | \  t \text{
     occurs in } X\} \]
  In other words, $\twonotes (X)$ is the set of atomic terms in $X$. 

  We next refine this model to achieve a property which is not
  not modally definable.  To do this, we will construct a \emph{bisimular model}{\footnote{The notion of bisimulation is
   discussed at length in {\cite[chapter 2.2]{blackburn_modal_2001}}.  We
  note that the basic definition and all of the usual theorems
  may be generalized to awareness models in a straightforward fashion,
  such that bisimulation entails modal equivalence. }}.  
  Define  $\mathbbm{M}' \assign \langle W', V', R'', \mathcal{A}' \rangle$ such
  that{\footnote{Let $\{i_X \  | \  X \in \Xi\}$ be the
  family of canonical injections into the coproduct $\biguplus_{\Xi}
  W$. By convention we use $v_X$ as shorthand for $i_X (v)$.}}:
  \begin{itemizedot}
    \item $W' \assign \biguplus_{\Xi} W$
    \item $V' (p) \assign \{v_X \  | \  X \in \Xi
    \  \& \  v \in V (p)\}$
    \item $R'_X \assign \{(v_Y, u_Z) \  | \  Y, Z \in \Xi
    \  \& \  v R_Z u \  \& \ 
    \twonotes (X) \subseteq \twonotes (Z)\}$
    \item $\mathcal{A}'_X(v_Y) \assign \mathcal{A}_{X} (v)$
  \end{itemizedot}
  Note that this construction makes use of our assumption that $\Xi$ is
  non-empty.  Let $Z \assign \{(v, v_X) \  | \  X \in \Xi
  \  \& \  v \in W' \}$; it is straightforward to prove
  that $Z$ is a bisimulation between $\mathbbm{M}$ and $\mathbbm{M}'$. Hence $\mathbbm{M}', w_X
  \nvDash \psi$ for some $w_X \in W'$.  In addition, $\mathbbm{M}'$
  inherits (\ref{fin}) through (\ref{sub}) from $\mathbbm{M}$.  Next
  observe, forall terms $X$ and $Y$ where $X \oplus Y \in \Xi$:
  \begin{eqnarray*}
    \text{$(v_V, u_Z) \in R'_X \cap R'_Y$} & \Longleftrightarrow & \exists Z
    \in \Xi . \text{$v R_Z u \  \& \  \twonotes (X)
    \subseteq \twonotes (Z) \  \& \  \twonotes (Y)
    \subseteq \twonotes (Z)$}\\
    & \Longleftrightarrow & \exists Z \in \Xi . \text{$v R_Z u \ 
    \& \  \twonotes (X) \cup \twonotes (Y) \subseteq \twonotes
    (Z)$}\\
    & \Longleftrightarrow & \exists Z \in \Xi . \text{$v R_Z u \ 
    \& \  \twonotes (X \oplus Y) \subseteq \twonotes (Z)$}\\
    & \Longleftrightarrow &  \text{$(v_V, u_Z) \in R'_{X \oplus Y}$}
  \end{eqnarray*}
  As in our previous proofs, let  $\iota : W' \hookrightarrow \Phi \backslash \Psi$ be an injection
  assigning fresh nominals to worlds.  Define $\mathbbm{M}'' \assign \langle W'',
  V'', R'', \mathcal{A}'' \rangle$ such that:
  \begin{center}
    \begin{tabular}{lllll}
      $\bullet$ & $W'' \assign W'$ &  & $\bullet$ & $V'' (p) \assign \left\{
      \begin{array}{ll}
        V' (p) & p \in \Lambda\\
        \{v\} & p = \iota (v)\\
        \varnothing & o / w
      \end{array} \right.$
    \end{tabular}
  \end{center}
Moreover,  $R''_X$ and $\mathcal{A}_X'' (v)$ are defined inductively as follows:
  \begin{center}
    \begin{tabular}{lll}
      $\bullet$ & $R''_t \assign \left\{ \begin{array}{ll}
        R'_t & t \in \Xi\\
        W'' \times W'' & o / w
      \end{array} \right.$ & where $t \in \Pi$ \\
      $\bullet$ & $R''_{X \oplus Y} \assign R''_X \cap R''_Y$ & \\
      &  & \\
      $\bullet$ & $\mathcal{A}''_t (v) \assign \{\phi \  | \phi \in
      \mathcal{A}'_t \} \cup \{\neg \iota (u) \  | \ 
      \neg v R''_t u\}$ & where $t \in \Pi$\\
      $\bullet$ & $\mathcal{A}''_{X \oplus Y} (v) \assign \mathcal{A}''_{X} (v)
      \cup \mathcal{A}''_Y (v)$ & 
    \end{tabular}
  \end{center}
  Induction over complexity of subformulae $\phi$ of $\psi$ yields
  $\mathbbm{M}', w \vDash \phi \Longleftrightarrow \mathbbm{M}'', w \vDash
  \phi$, so we know that there is some world $w \in W''$ such that
  $\mathbbm{M}'', w \nvDash \psi$.  All that is left is to illustrate that
  $\mathbbm{M}''$ has the properties we want.
  
  First note that by definition, $\mathbbm{M}''$ makes true
  \tmtextbf{CHOICE}, since the awareness sets are defined inductively
  to enforce this property. Next we may see that $\mathbbm{M}''$
  inherits (\ref{fin}) through (\ref{sub}) from $\mathbbm{M}'$.  Hence
  $\mathbb{M}''$ is reflexive.  

  The model $\mathbb{M}''$ also inherits $(v,u) R''_{X \oplus Y} \iff
  (v,u) \in  (R''_{X}\cap R''_{Y})$,
  and strengthens this property to cover \emph{all} terms $X$ and $Y$,
  not just those occurring in $\Xi$. This may
  be verified by double induction on complexity of terms $X$ and $Y$.

  A final induction argument on the complexity of a term $X$ yields:
  \[ u R_X'' v \Longleftrightarrow \mathbbm{M}'', v \vDash \bigwedge
     \mathcal{A}''_X (v) \]
  From these facts, we have that SJL itself is SJL-suitable for $\mathbb{M}''$. By repeating our previous arguments
  from Theorem \ref{completeness1} we can prove that $\textup{\tmtextbf{JCSQ}}_\vdash$.
\end{proof}

%%% Local Variables: 
%%% mode: latex
%%% TeX-master: "paper"
%%% End: 
