The method of the previous completeness proofs make implicit use of concepts from
hybrid logic{\footnote{Hybrid logic was first presented in
{\cite{prior_revised_1969}} and later formally developed in
{\cite{bull_approach_1970}}.}}.  In this section we extend the logic
we have been developing to full hybrid logic.

No such concept like hybrid logic is present in Russell's writing,
however it is not hard to motivate an epistemic reading.
Colloquially, we might describe a particular fantasy fiction
as ``knowing a lot about Tolkien's \emph{middle earth}''.  One way of
making sense of how she ``knows'' is that her explicit and implicit reasoning about
Tolkien's fantasy world is  \emph{valid at middle
  earth}.  To model this in a hybrid framework, assume there is a
nominal entitled $\textup{\emph{middle earth}}$, and some world with
that label. We denote that the agent's reasoning is valid there as $\PP^{\textup{\emph{middle
      earth}}}$. 
We permit that worlds may be multiply labeled or not labeled at
all, and labels may fail to refer to anything.

We do not express $\PP^i$ directly; rather, we will define it in terms
of a universal modality.

\begin{definition}
  Let $\Phi$ be a set of letters and $\Psi$ a set of nominals, and define the
  language $\mathcal{L}_H (\Phi, \Psi)$ as:
  \[ \phi \  : : = \  p \in \Phi \  |
     \  i \in \Psi \  | \  \bot \  |
     \  \phi \rightarrow \psi \  | \  \Box \phi
     \  | \  A : \phi \  | \  \forall
     \phi \]
\end{definition}

Our approach in hybrid logic is to employ a universal modality along with
nominals.  This framework presents a logic where the agent may reason about
various labeled scenarios.    From these intuitions we have the following definition:

\begin{definition}
  \label{hybridsemantics}Let a \tmtextbf{hybrid model} $\mathbbm{M}= \langle
  W, V, R, \mathcal{A}, \ell \rangle$ be an awareness model as in Definition
  \ref{awarenessmodels}, along with a partial labeling function $\ell : \Psi
  \nrightarrow W$
  
  The semantics for $\vDash$ are the same as in Definition
  \ref{awarenessmodels0}, with the following extensions
  \begin{eqnarray*}
    \mathbbm{M}, w \vDash \forall \phi & \iff & \forall v \in
    W.\mathbbm{M}, v \vDash \phi\\
    \mathbbm{M}, w \vDash i & \iff & \ell (i) = w
  \end{eqnarray*}
\end{definition}

The other connectives and operators are defined as usual. We also employ a
the following shorthand:

\begin{notation}
  We employ the following special abbreviations:
  \begin{eqnarray}
    \text{\tmtextup{@}}^i \phi \assign \forall (i \rightarrow \phi)
    \hspace{2em} & \exists \phi \assign \neg \forall \neg \phi \hspace{3em} &
    \circlearrowleft^i \assign \diamondsuit i \nonumber
  \end{eqnarray}
\end{notation}

We have a validity reflecting one of the axioms we saw in
{\S}\ref{awarenesslogic}:
\[ \vDash \circlearrowleft^i \rightarrow \Box \phi \rightarrow
   \text{\tmtextup{@}}^i \phi \]
This can be read as ``If the agent's knowledge base is sound at world $i$,
then if they can deduce something from their knowledge base, what they deduce
must be true at world $i$.'.  In a way, this may be considered as
{\tmem{relativising}} knowledge particular named situations.



The semantics in Definition \ref{hybridsemantics} obviates the \tmtextbf{SND}
principle we previously presented, since there is no explicit
$\circlearrowleft$ symbol in this setting. \tmtextbf{CSQ} still makes sense
without modification, however.  The following gives a logic for hybrid models
making true \tmtextbf{CSQ}:
\begin{table}[h]
\begin{centering}
  \begin{tabular}{ll}
    $\vdash \phi \rightarrow \psi \rightarrow \phi$ & $\vdash \forall (\phi
    \rightarrow \psi) \rightarrow \forall \phi \rightarrow \forall \psi$\\
    $\vdash (\phi \rightarrow \psi \rightarrow \chi) \rightarrow (\phi
    \rightarrow \psi) \rightarrow \phi \rightarrow \chi$ & $\vdash \forall
    \phi \rightarrow \phi$\\
    $\vdash ((\phi \rightarrow \bot) \rightarrow (\psi \rightarrow \bot))
    \rightarrow \psi \rightarrow \phi$ & $\vdash \forall \phi \rightarrow
    \forall \forall \phi$\\
    $\vdash \Box(\phi \rightarrow \psi) \rightarrow \Box \phi \rightarrow \Box
    \psi$ & $\vdash \exists \phi \rightarrow \forall \exists \phi$\\
    $\vdash A : \phi \rightarrow \Box \phi$ & $\vdash \forall \phi \rightarrow
    \Box \phi$\\
    & $\vdash i \rightarrow \phi \rightarrow \text{\tmtextup{@}}^i \phi$\\
    &  
  \end{tabular}
\begin{tabular}{lllll}
      $\displaystyle\frac{\vdash \phi \rightarrow \psi \ \ \vdash \phi}{\vdash
      \psi}$ & \ \  & $\displaystyle \frac{\vdash \phi}{\vdash \Box \phi}$ &
      \ \  & $\displaystyle \frac{\vdash \phi}{\vdash \forall \phi}$
    \end{tabular}
  \caption{\label{logic2}Hybrid Awareness Logic for \tmtextbf{CSQ}}
\end{centering}
\end{table}

\begin{theorem}
  \label{completeness2}Assuming an infinite store of nominals $\Psi$, the
  hybrid awareness logic presented is sound and weakly complete for all hybrid
  models making true $\textup{\tmtextbf{CSQ}}_{\vdash'}$ for some
  suitable logic $\vdash'$.
\end{theorem}
\begin{proof}
  As before, soundness is trivial.  Completeness is based on a modification
  of techniques found in {\cite[chapter 5]{boolos_logic_1995}}.  Assume
  that $\nvdash \psi$ and let $\Sigma_0$ be the set of subformulae of $\psi$,
  and let $\Upsilon$ be the set of nominals occurring in $\psi$.  Define:
  
  \begin{align*}
    {\Sigma}_1 &
    {\assign}{\Sigma}_0{\cup}\{{\forall}{\phi}{\ }|{\ }\Box{\phi}{\in}{\Sigma}_0\}\\
    {\Sigma}_2 &
    {\assign}{\Sigma}_1{\cup}\{\text{@}_i{\phi}{\ },i{\rightarrow}{\phi}|{\ }{\phi}{\in}{\Sigma}_1{\ }\&{\ }i{\in}{\Upsilon}\}\\
    {\Sigma}_3 &
    {\assign}{\Sigma}_2{\cup}\{{\neg}{\phi}{\ }|{\ }{\phi}{\in}{\Sigma}_2\}
  \end{align*}
  
  It is easy to see that $\Sigma_3$ is finite and closed under subformulae and
  single negations. Let $\mathbbm{M}= \langle W, V, R_{\Box}, \sim_{\forall},
  \mathcal{A} \rangle$ be the finite canonical model formed of maximally
  consistent subsets of $\Sigma_3$, where everything is defined as usual,
  except $\sim_{\forall}$, which obeys the following specification:
  
  \begin{align*}
    w {\sim}_{{\forall}}v & {\Longleftrightarrow}(({\forall}{\phi}){\in}w
    {\Longleftrightarrow}({\forall}{\phi}){\in}v)
  \end{align*}
  
  This model makes true the following properties:
  \begin{enumerateroman}
    \item $\sim_{\forall}$ is an equivalence relation    
    \item $R_{\Box} \subseteq \sim_{\forall}$
    \item In each equivalence class specified by $\sim_{\forall}$, there is at
    most one world making true $i$ for all $i \in \Upsilon$
  \end{enumerateroman}
  As in our previous construction we have that there is some world $w$ where
  $\mathbbm{M}, w \nvDash \psi$.  Let $\mathbbm{M}' = \langle W', V',
  R_{\Box}', \sim_{\forall}', \mathcal{A}' \rangle$ be the submodel generated
  by $\{w\}$; we have that $\mathbbm{M}', w \nvDash \psi$ (see {\cite[chapter
  2.1]{blackburn_modal_2001}} for details on generated submodels).  In this
  model $\forall$ is a universal modality and either $\left\llbracket i
  \right\rrbracket^{\mathbbm{M}'} = \varnothing$ or $\left\llbracket i
  \right\rrbracket^{\mathbbm{M}'} =\{v\}${\footnote{As per the usual
  convention, here $\left\llbracket \phi \right\rrbracket^{\mathbbm{M}}$
  denotes $\{w \in W \  | \  \mathbbm{M}, w \vDash
  \phi\}$.  We will drop $\mathbbm{M}$ where it is unambiguous.}}. Since the
  store $\Psi$ of nominals is infinite and $\Upsilon$ is finite, we have that
  $\Psi \backslash \Upsilon$ is infinite, so there is some injection $\iota :
  W' \hookrightarrow \Psi \backslash \Upsilon$ which assigns a fresh nominal
  to every world.  Let $\mathbbm{M}'' = \langle W'', V'', R'', \mathcal{A}'',
  \ell \rangle$, where:
  \begin{center}
    \begin{tabular}{lllll}
      $\bullet$ & $W'' \assign W'$ &  & $\bullet$ & $V'' (p) \assign V' (p)$\\
      $\bullet$ & $R'' \assign R'_{\Box}$ & \ \  & $\bullet$ &
      $\mathcal{A}'' (v) \assign \{\phi \  | \  \phi \in
      \mathcal{A}' (v)\} \cup \{\neg \iota (u) \  | \ 
      \neg v R' u\}$\\
       $\bullet$ & \begin{tabular}{ll}
       $\ell (i) \assign \left\{ \begin{array}{ll}
          w & i \in \Upsilon \  \& \  \mathbbm{M}, w
          \vDash i\\
          \iota^{- 1} (i) & i \in \iota [W]\\
          \text{undefined} & \text{o/w}
        \end{array} \right.$
      \end{tabular} &  & $\bullet$ &  $R'' \assign R'_{\Box}$ 
    \end{tabular}
  \end{center}
  
  The the worlds is $\mathbbm{M}'$ and $\mathbbm{M}''$ agree on all
  subformulae of $\psi$, hence $\mathbbm{M}'', w \nvDash
  \psi$. Moreover, the logic in Table \ref{logic2} itself 
  is suitable for $\mathbbm{M}''$ and
  $\mathbbm{M}''$ makes true $\textup{\tmtextbf{CSQ}}_{\vdash}$.  From
  this it follows that we have completeness.
\end{proof}
%%% Local Variables: 
%%% mode: latex
%%% TeX-master: "paper"
%%% End: 


%%% Local Variables: 
%%% mode: latex
%%% TeX-master: "paper"
%%% End: 
