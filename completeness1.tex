\begin{proof}
  Soundness is straightforward, so we will only address completeness.  Assume
  $\nvdash \psi$.  Consider the finite canonical model $\mathbbm{M}= \langle
  W, V, R, \mathcal{A} \rangle$ formed of maximally consistent sets of
  subformulae of $\phi$ (closed under single negations{\footnote{A discussion
  of this property may be found in {\cite[pg.
  243]{blackburn_modal_2001}}.}}), as per the finitary modal completeness
  proofs found in {\cite[chapter 4.8]{blackburn_modal_2001}} and
  {\cite[chapter 5]{boolos_logic_1995}}.  Typical presentations do not
  include awareness, however it is defined intuitively in this case:
  \[ \phi \in \mathcal{A}_w \Longleftrightarrow A : \phi \in w \]
  Since $\psi$ is a subformula of itself, we have $\mathbbm{M}, w \nvDash
  \psi$ for some world $w$ by a finitary {\tmem{Lindenbaum Lemma}} and
  {\tmem{Truth Theorem}}.  Moreover, it is straightforward to verify that
  $\mathbbm{M}$ makes true the following properties:
  \begin{enumeratenumeric}
    \item $W$ and $\mathcal{A}_v$ are finite, and $\psi \in \mathcal{A}_v$
    only if $\psi$ is a (possibly negated) subformula of $\phi$
    
    \item if $\mathbbm{M}, v \vDash A : \psi$ then $\mathbbm{M}, v \vDash \Box
    \psi$
    
    \item if $\mathbbm{M}, v \vDash \circlearrowleft$ then $v R v$
  \end{enumeratenumeric}
  
  We next produce a bisimular{\footnote{The notion of bisimulation is
  discussed at length in {\cite[chapter 2.2]{blackburn_modal_2001}}.  We
  note that the basic definition and all of the usual theorems (modal
  equivalence, Henessy-Milner, etc.) may be generalized to awareness models in
  a straightforward fashion. }} model $\mathbbm{M}'$ which makes true (1), (2)
  and a stronger from of (3):
  
  {\hspace*{\fill}}\begin{tabular}{ll}
    \tmtextbf{3$'$.} & $\mathbbm{M}', v \vDash \circlearrowleft \iff v R v$
  \end{tabular}{\hspace*{\fill}}
  
  
  
  To this end define $\mathbbm{M}' \assign \langle W', V', R', \mathcal{A}'
  \rangle$ such that{\footnote{Throughout this article, we use $l$ and $r$ to
  denote the two canonical injections associated with the coproduct $W \uplus
  W$. We will use $v_l$ and $v_r$ as the shorthand for $l (v)$ and $r (v)$
  respectively.}}: {\hspace*{\fill}}
  \begin{itemizedot}
    \item $W' \assign W \uplus W$
    
    \item $V' (p) \assign \{v_l, v_r \  | \  v \in V
    (p)\}$
    
    \item $R' \assign \{(v_l, u_r), (v_r, u_l) \  | \  v R
    u\} \cup \{(v_l, v_l), (v_r, v_r) \  | \  \mathbbm{M},
    v \vDash \circlearrowleft\}$
    
    \item $\mathcal{A}' (v_l) \assign \mathcal{A}' (v_r) \assign \mathcal{A}
    (v)$
  \end{itemizedot}
  
  
  It is straightforward to verify that $\mathbbm{M}'$ makes true the desired
  properties.  Let $Z \assign \{(v, v_l), (v, v_r) \  |
  \  v \in W\}$.  then $Z$ is a bisimulation between $\mathbbm{M}$
  and $\mathbbm{M}'$.  Therefore we know there is some $w \in W$ such that
  $\mathbbm{M}', w_l \nvDash \psi$ and $\mathbbm{M}', w_r \nvDash \psi$.
  
  
  
  Finally we construct a model $\mathbbm{M}''$ which agrees with
  $\mathbbm{M}'$ for all subformulae of $\psi$, which makes true
  \tmtextbf{CSQ} and \tmtextbf{SND}.  Let $\Lambda$ be the set of proposition
  letters occurring in $\psi$.  We will make use members of $\Phi \backslash
  \Lambda$ as {\tmem{nominals}}, as per the tradition in hybrid logic.  This
  is possible since both $W'$ and $\Lambda$ are finite and $\Phi$ is infinite
  by hypothesis.  Let $\iota : W \uplus W \hookrightarrow \Phi \backslash
  \Lambda$ be an injection assigning a fresh nominal associated with each
  world in $\mathbbm{M}'$.  Now define $\mathbbm{M}'' \assign \langle W'',
  V'', R'', \mathcal{A}'' \rangle$, where:
  
  
  
  \begin{center}
    \begin{tabular}{lllll}
      $\bullet$ & $W'' \assign W'$ &  & $\bullet$ & $V'' (p) \assign \left\{
      \begin{array}{ll}
        V' (p) & p \in \Lambda\\
        \{v\} & p = \iota (v)\\
        \varnothing & o / w
      \end{array} \right.$\\
      $\bullet$ & $R'' \assign R'$ & {\hspace{3em}} & $\bullet$ &
      $\mathcal{A}''_v \assign \{\phi \  | \phi \in \mathcal{A}'_v
      \} \cup \{\neg \iota (u) \  | \  \neg v R' u\}$
    \end{tabular}
  \end{center}
  
  
  
  By induction we have that for every subformula $\phi$ of $\psi$ that
  $\mathbbm{M}', v \vDash \phi$ if and only if $\mathbbm{M}'', v \vDash \phi$,
  hence $\mathbbm{M}'', w_l \nvDash \psi$.
  
  
  
  All that is left to show is that $\mathbbm{M}''$ makes true \tmtextbf{CSQ}
  and \tmtextbf{SND}.  $\mathbbm{M}''$ has three further properties:
  \begin{enumerateroman}
    \item $\mathcal{A}''_v$ is finite for all worlds $v$
    
    \item  $v R' u$ if and only if $\mathbbm{M}'', u \vDash \bigwedge
    \mathcal{A}'' (v)$
    
    \item The logic presented in Table \ref{logic1} is sound for
    $\mathbbm{M}''$
  \end{enumerateroman}
  From (ii) and the fact that $\mathbbm{M}'$ makes true (3$'$), we have
  \tmtextbf{SND} for $\mathbbm{M}''$.
  
  
  
  All that is left is to demonstrate \tmtextbf{CSQ} for some sound logic.  We
  will use the logic in Table \ref{logic1} itself.  From (i), (ii), (iii),
  and the \tmtextit{deduction theorem} for modal logic, we have the following
  line of reasoning:
  
  \begin{align*}
    Th({\mathbbm{M}}''){\cup}\mathcal{A}''_v{\vdash}{\phi} &
    {\Longleftrightarrow}Th({\mathbbm{M}}''){\vdash}\bigwedge\mathcal{A}''_v{\rightarrow}{\phi}\\
    &
    {\Longleftrightarrow}\bigwedge\mathcal{A}''_v{\rightarrow}{\phi}{\in}Th({\mathbbm{M}}'')\\
    &
    {\Longleftrightarrow}{\mathbbm{M}}'',u{\vDash}\bigwedge\mathcal{A}''_v{\rightarrow}{\phi}\text{
    for all $u \in W''$}\\
    & {\Longleftrightarrow}\text{ for all $u \in W''$, if
    }{\mathbbm{M}}'',u{\vDash}\mathcal{A}''_v\text{ then
    }{\mathbbm{M}}'',u{\vDash}{\phi}\\
    & {\Longleftrightarrow}\text{ for all $u \in W''$, if }v R 'u \text{ then
    }{\mathbbm{M}}'',u{\vDash}{\phi}\\
    & {\Longleftrightarrow}{\mathbbm{M}}'',v{\models}\Box{\phi}
  \end{align*}
\end{proof}
%%% Local Variables: 
%%% mode: latex
%%% TeX-master: "paper"
%%% End: 
