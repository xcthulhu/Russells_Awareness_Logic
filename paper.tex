\documentclass[draft]{article}

\usepackage{hyperref}
 \hypersetup{
     colorlinks,%
     citecolor=black,%
     filecolor=black,%
     linkcolor=black,%
     urlcolor=black
 }

\usepackage{stmaryrd}
\usepackage{todonotes}

\usepackage{geometry,amsmath,amssymb,wasysym,enumerate,bbm,color}
\geometry{letterpaper}

%%%%%%%%%% Start TeXmacs macros
\newcommand{\assign}{:=}
\newcommand{\nin}{\not\in}
\newcommand{\tmem}[1]{{\em #1\/}}
\newcommand{\tmop}[1]{\ensuremath{\operatorname{#1}}}
\newcommand{\tmstrong}[1]{\textbf{#1}}
\newcommand{\tmtextbf}[1]{{\bfseries{#1}}}
\newcommand{\tmtextit}[1]{{\itshape{#1}}}
\newcommand{\tmtextsf}[1]{{\sffamily{#1}}}
\newcommand{\tmtextup}[1]{{\upshape{#1}}}
\newenvironment{descriptioncompact}{\begin{description} }{\end{description}}
\newenvironment{descriptiondash}{\begin{description} }{\end{description}}
\newenvironment{enumeratenumeric}{\begin{enumerate}[1.] }{\end{enumerate}}
\newenvironment{enumerateroman}{\begin{enumerate}[i.] }{\end{enumerate}}
\newenvironment{itemizedot}{\begin{itemize} \renewcommand{\labelitemi}{$\bullet$}\renewcommand{\labelitemii}{$\bullet$}\renewcommand{\labelitemiii}{$\bullet$}\renewcommand{\labelitemiv}{$\bullet$}}{\end{itemize}}
\newenvironment{proof}{\noindent\textbf{Proof\ }}{\hspace*{\fill}$\Box$\medskip}
\definecolor{grey}{rgb}{0.75,0.75,0.75}
\definecolor{orange}{rgb}{1.0,0.5,0.5}
\definecolor{brown}{rgb}{0.5,0.25,0.0}
\definecolor{pink}{rgb}{1.0,0.5,0.5}
\newtheorem{definition}{Definition}
\newtheorem{proposition}{Proposition}
\newtheorem{theorem}{Theorem}
%%%%%%%%%% End TeXmacs macros

\begin{document}

\title{Epistemic Logics For Knowledge Bases}\author{Matthew P.
Wampler-Doty}\date{}\maketitle

\begin{abstract}
  In {\cite{van_benthem_reflectionsepistemic_1991}}, Johan van Benthem
  proposed a research program for logic of \tmtextit{explicit knowledge}. 
  Two efforts have since emerged with proposals for logic of explicit
  knowledge: the \tmtextit{Dynamics of Awareness}
  {\cite{van_benthem_inference_2009}} and \tmtextit{Justification Logic}
  {\cite{artemov_introducing_2005}}.  The purpose of this paper is to
  repurpose various epistemic logics from these previous efforts for reasoning
  about logical closures over knowledge bases.  The results in this paper are
  based on previous work carried out in
  {\cite{wampler-doty_evidentialist_2010}}.
\end{abstract}

\listoftodos

\section{Introduction}
\label{intro}
Modern epistemic logic is commonly acknowledged to begin with
Hintikka's \emph{Knowledge and Belief}
\cite{hintikka_knowledge_1969}. While Hintikka began by considering
traditional topics in the theory of knowledge, epistemic logic
resulted in a bifurcation 20th century research into epistemology.  A number of
logicians and philosophers have acknowledged this rift, most notably
Vincent Hendricks \cite{hendricks_wheres_2006,
  hendricks_mainstream_2006}.

The purposes of this essay is to try to close the gap between these
traditions, by showing that awareness logics are suitable for
reasoning about the form of foundationalism proposed
by Bertrand Russell in \emph{The Problems of Philosophy}
\cite{russell_problems_1936}.  We begin by reviewing Russell's
theory, and contrasting it with modern perspectives.  In particular,
we illustrate that the modern concepts of \emph{explicit} and
\emph{implicit} knowledge, originally due to Levesque\footnote{While we will refer often to
  terms coined by Levesque in this essay, we will not use his
  proposed semantics.  We instead prefer variations on
  \emph{awareness logic}, originally developed by Fagin and Halpern
  \cite{fagin_belief_1988}.}\cite{levesque_logic_1984},  are reflect Russell's distinction between \emph{immediate} and \emph{derivative}
knowledge.  We propose a novel property of awareness models which
inspired by Russell's epistemology.  The
remainder of this paper is then devoted to exploring how awareness
logics can be designed to express this foundationalist property.

%%% Local Variables: 
%%% mode: latex
%%% TeX-master: "paper"
%%% End: 


\section{Basic Explicit Logics of Knowledge Bases}

\subsection{Awareness Logic}
\label{awarenesslogic}
In this section, we give the explicit semantics to basic awareness logic,
originally presented in {\cite{fagin_belief_1987}}, with an additional axiom.
 One difference in our presentation is the inclusion of a novel letter and a
superficial change in notation.  This work is inspired by the developments in
van Benthem and Vel\'azquez-Quesada {\cite{van_benthem_inference_2009}}.

\begin{definition}
  Let $\Phi$ be a set of letters and define the language $\mathcal{L}_A
  (\Phi)$ as:
  \[ \phi \hspace{1em} : : = \hspace{1em} p \in \Phi \hspace{1em} |
     \hspace{1em} \circlearrowleft \hspace{1em} | \hspace{1em} \bot
     \hspace{1em} | \hspace{1em} \phi \rightarrow \psi \hspace{1em} |
     \hspace{1em} \Box \phi \hspace{1em} | \hspace{1em} A : \phi \hspace{1em}
  \]
\end{definition}

We define the other connectives and modal operators as usual for classical
modal logics.

\begin{definition}
  \label{awarenessmodels}An {\tmstrong{awareness model}} $\mathbbm{M}= \langle
  W, V, R, \mathcal{A} \rangle$ is a Kripke model with a valuation $V : \Phi
  \cup \{\circlearrowleft\} \rightarrow 2^W$ and an awareness function
  $\mathcal{A} : W \rightarrow 2^{\mathcal{L}_A (\Phi)}$.  We will usually
  denote $\mathcal{A} (w)$ as $\mathcal{A}_w$.
  
  
  
  Define the semantic entailment relation $\vDash$ where atomic propositions,
  logical connectives and modality are interpreted as usual.  We employ the
  following novel definition:
  \begin{eqnarray*}
    \mathbbm{M}, w \vDash A : \phi & \text{iff} & \phi \in \mathcal{A_w}
  \end{eqnarray*}
\end{definition}

The following definition distinguishes us from previously investigations of
awareness logic:

\begin{definition}
  Define the ``\tmtextbf{theory of a model} $\mathbbm{M}$'' as:
  \[ \tmop{Th} (\mathbbm{M}) \assign \{\phi \hspace{1em} | \hspace{1em}
     \mathbbm{M}, v \vDash \phi \text{ for all } v \in W\} \]
  We define the following properties that a model $\mathbbm{M}$ may posses:
  
  \begin{descriptiondash}
    \item[CSQ] $\mathbbm{M}, w \vDash \Box \phi \text{ iff } \tmop{Th}
    (\mathbbm{M}) \cup \mathcal{A}_w \vdash \phi$
    
    \item[SND] $\mathbbm{M}, w \vDash \circlearrowleft \text{ iff }
    \mathbbm{M}, w \vDash \phi$ for all $\phi \in \mathcal{A}_w$
  \end{descriptiondash}
  
  Here $\vdash$ is any logical consequence relation which is sound for
  $\vDash$, making true \tmtextbf{modus ponens} and \tmtextbf{reflection} (ie,
  $\Gamma \vdash \phi$ if $\phi \in \Gamma$)
\end{definition}

Neither \tmtextbf{CSQ} nor \tmtextbf{SND} correspond to simple modally
definable properties.



Intuitively, the \tmtextbf{CSQ} asserts that an agent believes a formula
$\phi$ if and only if it follows logically from their knowledge base and their
background knowledge, represented by $\tmop{Th} (\mathbbm{M})$. 
\tmtextbf{CSQ} is Vincent Hendrik's principle of logical omniscience mentioned
in {\S}\ref{intro}.



\tmtextbf{SND} asserts that $\circlearrowleft$ corresponds to the agent's
knowledge base being sound. A sound knowledge base will only render true
conclusions when used in deductions.  If one considers ``the existence of a
sound deduction'' sufficient for a kind of knowledge, then $\circlearrowleft$
is suitable for investigating this notion of epistemology.  That is, if
$\mathbbm{M}, w \vDash \circlearrowleft \wedge \Box \phi$, then we may
naturally say that the agent (implicitly) {\tmem{knows}} $\phi$ on the basis
of $\mathcal{A}_w$, rather than merely {\tmem{believing}} $\phi$ on the basis
of $\mathcal{A}_w$.



The logic of models making true \tmtextbf{CSQ} and \tmtextbf{SND} is provided
in Table \ref{logic1}.

\begin{table}[h]
  \begin{tabular}{l}
    $\vdash \phi \rightarrow \psi \rightarrow \phi$\\
    $\vdash (\phi \rightarrow \psi \rightarrow \chi) \rightarrow (\phi
    \rightarrow \psi) \rightarrow \phi \rightarrow \chi$\\
    $\vdash ((\phi \rightarrow \bot) \rightarrow (\psi \rightarrow \bot))
    \rightarrow \psi \rightarrow \phi$\\
    $\vdash \Box(\phi \rightarrow \psi) \rightarrow \Box \phi \rightarrow \Box
    \psi$\\
    $\vdash A : \phi \rightarrow \Box \phi$\\
    $\vdash \circlearrowleft \rightarrow \Box \phi \rightarrow \phi$\\
    \\
    \begin{tabular}{lll}
      $\frac{\vdash \phi \rightarrow \psi \hspace{4em} \vdash \phi}{\vdash
      \psi}$ & {\hspace{6em}} & $\frac{\vdash \phi}{\vdash \Box \phi}$
    \end{tabular}
  \end{tabular}
  \caption{\label{logic1}Awareness Logic for \tmtextbf{CSQ} and
  \tmtextbf{SND}}
\end{table}

\begin{theorem}
  \label{completeness1}Assuming an infinite store of proposition letters
  $\Phi$, the awareness logic presented is sound and weakly complete for
  awareness models making true \tmtextbf{CSQ} and \tmtextbf{SND}.
\end{theorem}

\begin{proof}
  Soundness is straightforward, so we will only address completeness.  Assume
  $\nvdash \psi$.  Consider the finite canonical model $\mathbbm{M}= \langle
  W, V, R, \mathcal{A} \rangle$ formed of maximally consistent sets of
  subformulae of $\phi$ (closed under single negations{\footnote{A discussion
  of this property may be found in {\cite[pg.
  243]{blackburn_modal_2001}}.}}), as per the finitary modal completeness
  proofs found in {\cite[chapter 4.8]{blackburn_modal_2001}} and
  {\cite[chapter 5]{boolos_logic_1995}}.  Typical presentations do not
  include awareness, however it is defined intuitively in this case:
  \[ \phi \in \mathcal{A}_w \Longleftrightarrow A : \phi \in w \]
  Since $\psi$ is a subformula of itself, we have $\mathbbm{M}, w \nvDash
  \psi$ for some world $w$ by a finitary {\tmem{Lindenbaum Lemma}} and
  {\tmem{Truth Theorem}}.  Moreover, it is straightforward to verify that
  $\mathbbm{M}$ makes true the following properties:
  \begin{enumeratenumeric}
    \item $W$ and $\mathcal{A}_v$ are finite, and $\psi \in \mathcal{A}_v$
    only if $\psi$ is a (possibly negated) subformula of $\phi$
    
    \item if $\mathbbm{M}, v \vDash A : \psi$ then $\mathbbm{M}, v \vDash \Box
    \psi$
    
    \item if $\mathbbm{M}, v \vDash \circlearrowleft$ then $v R v$
  \end{enumeratenumeric}
  
  We next produce a bisimular{\footnote{The notion of bisimulation is
  discussed at length in {\cite[chapter 2.2]{blackburn_modal_2001}}.  We
  note that the basic definition and all of the usual theorems (modal
  equivalence, Henessy-Milner, etc.) may be generalized to awareness models in
  a straightforward fashion. }} model $\mathbbm{M}'$ which makes true (1), (2)
  and a stronger from of (3):
  
  {\hspace*{\fill}}\begin{tabular}{ll}
    \tmtextbf{3$'$.} & $\mathbbm{M}', v \vDash \circlearrowleft \iff v R v$
  \end{tabular}{\hspace*{\fill}}
  
  
  
  To this end define $\mathbbm{M}' \assign \langle W', V', R', \mathcal{A}'
  \rangle$ such that{\footnote{Throughout this article, we use $l$ and $r$ to
  denote the two canonical injections associated with the coproduct $W \uplus
  W$. We will use $v_l$ and $v_r$ as the shorthand for $l (v)$ and $r (v)$
  respectively.}}: {\hspace*{\fill}}
  \begin{itemizedot}
    \item $W' \assign W \uplus W$
    
    \item $V' (p) \assign \{v_l, v_r \hspace{1em} | \hspace{1em} v \in V
    (p)\}$
    
    \item $R' \assign \{(v_l, u_r), (v_r, u_l) \hspace{1em} | \hspace{1em} v R
    u\} \cup \{(v_l, v_l), (v_r, v_r) \hspace{1em} | \hspace{1em} \mathbbm{M},
    v \vDash \circlearrowleft\}$
    
    \item $\mathcal{A}' (v_l) \assign \mathcal{A}' (v_r) \assign \mathcal{A}
    (v)$
  \end{itemizedot}
  
  
  It is straightforward to verify that $\mathbbm{M}'$ makes true the desired
  properties.  Let $Z \assign \{(v, v_l), (v, v_r) \hspace{1em} |
  \hspace{1em} v \in W\}$.  then $Z$ is a bisimulation between $\mathbbm{M}$
  and $\mathbbm{M}'$.  Therefore we know there is some $w \in W$ such that
  $\mathbbm{M}', w_l \nvDash \psi$ and $\mathbbm{M}', w_r \nvDash \psi$.
  
  
  
  Finally we construct a model $\mathbbm{M}''$ which agrees with
  $\mathbbm{M}'$ for all subformulae of $\psi$, which makes true
  \tmtextbf{CSQ} and \tmtextbf{SND}.  Let $\Lambda$ be the set of proposition
  letters occurring in $\psi$.  We will make use members of $\Phi \backslash
  \Lambda$ as {\tmem{nominals}}, as per the tradition in hybrid logic.  This
  is possible since both $W'$ and $\Lambda$ are finite and $\Phi$ is infinite
  by hypothesis.  Let $\iota : W \uplus W \hookrightarrow \Phi \backslash
  \Lambda$ be an injection assigning a fresh nominal associated with each
  world in $\mathbbm{M}'$.  Now define $\mathbbm{M}'' \assign \langle W'',
  V'', R'', \mathcal{A}'' \rangle$, where:
  
  
  
  \begin{center}
    \begin{tabular}{lllll}
      $\bullet$ & $W'' \assign W'$ &  & $\bullet$ & $V'' (p) \assign \left\{
      \begin{array}{ll}
        V' (p) & p \in \Lambda\\
        \{v\} & p = \iota (v)\\
        \varnothing & o / w
      \end{array} \right.$\\
      $\bullet$ & $R'' \assign R'$ & {\hspace{3em}} & $\bullet$ &
      $\mathcal{A}''_v \assign \{\phi \hspace{1em} | \phi \in \mathcal{A}'_v
      \} \cup \{\neg \iota (u) \hspace{1em} | \hspace{1em} \neg v R' u\}$
    \end{tabular}
  \end{center}
  
  
  
  By induction we have that for every subformula $\phi$ of $\psi$ that
  $\mathbbm{M}', v \vDash \phi$ if and only if $\mathbbm{M}'', v \vDash \phi$,
  hence $\mathbbm{M}'', w_l \nvDash \psi$.
  
  
  
  All that is left to show is that $\mathbbm{M}''$ makes true \tmtextbf{CSQ}
  and \tmtextbf{SND}.  $\mathbbm{M}''$ has three further properties:
  \begin{enumerateroman}
    \item $\mathcal{A}''_v$ is finite for all worlds $v$
    
    \item  $v R' u$ if and only if $\mathbbm{M}'', u \vDash \bigwedge
    \mathcal{A}'' (v)$
    
    \item The logic presented in Table \ref{logic1} is sound for
    $\mathbbm{M}''$
  \end{enumerateroman}
  From (ii) and the fact that $\mathbbm{M}'$ makes true (3$'$), we have
  \tmtextbf{SND} for $\mathbbm{M}''$.
  
  
  
  All that is left is to demonstrate \tmtextbf{CSQ} for some sound logic.  We
  will use the logic in Table \ref{logic1} itself.  From (i), (ii), (iii),
  and the \tmtextit{deduction theorem} for modal logic, we have the following
  line of reasoning:
  
  \begin{align*}
    Th({\mathbbm{M}}''){\cup}\mathcal{A}''_v{\vdash}{\phi} &
    {\Longleftrightarrow}Th({\mathbbm{M}}''){\vdash}\bigwedge\mathcal{A}''_v{\rightarrow}{\phi}\\
    &
    {\Longleftrightarrow}\bigwedge\mathcal{A}''_v{\rightarrow}{\phi}{\in}Th({\mathbbm{M}}'')\\
    &
    {\Longleftrightarrow}{\mathbbm{M}}'',u{\vDash}\bigwedge\mathcal{A}''_v{\rightarrow}{\phi}\text{
    for all $u \in W''$}\\
    & {\Longleftrightarrow}\text{ for all $u \in W''$, if
    }{\mathbbm{M}}'',u{\vDash}\mathcal{A}''_v\text{ then
    }{\mathbbm{M}}'',u{\vDash}{\phi}\\
    & {\Longleftrightarrow}\text{ for all $u \in W''$, if }v R 'u \text{ then
    }{\mathbbm{M}}'',u{\vDash}{\phi}\\
    & {\Longleftrightarrow}{\mathbbm{M}}'',v{\models}\Box{\phi}
  \end{align*}
\end{proof}
%%% Local Variables: 
%%% mode: latex
%%% TeX-master: "paper"
%%% End: 


%%% Local Variables: 
%%% mode: latex
%%% TeX-master: "paper"
%%% End: 


\subsection{Hybrid Awareness Logic}
\label{hybrid_awareness}
The method of the previous completeness proofs make implicit use of concepts from
hybrid logic{\footnote{Hybrid logic was first presented in
{\cite{prior_revised_1969}} and later formally developed in
{\cite{bull_approach_1970}}.}}.  In this section we extend the logic
we have been developing to full hybrid logic.

No such concept like hybrid logic is present in Russell's writing,
however it is not hard to motivate an epistemic reading.
Colloquially, we might describe a particular fantasy fiction
as ``knowing a lot about Tolkien's \emph{middle earth}''.  One way of
making sense of how she ``knows'' is that her explicit and implicit reasoning about
Tolkien's fantasy world is  \emph{valid at middle
  earth}.  To model this in a hybrid framework, assume there is a
nominal entitled $\textup{\emph{middle earth}}$, and some world with
that label. We denote that the agent's reasoning is valid there as $\PP^{\textup{\emph{middle
      earth}}}$. 
We permit that worlds may be multiply labeled or not labeled at
all, and labels may fail to refer to anything.

We do not express $\PP^i$ directly; rather, we will define it in terms
of a universal modality.

\begin{definition}
  Let $\Phi$ be a set of letters and $\Psi$ a set of nominals, and define the
  language $\mathcal{L}_H (\Phi, \Psi)$ as:
  \[ \phi \  : : = \  p \in \Phi \  |
     \  i \in \Psi \  | \  \bot \  |
     \  \phi \rightarrow \psi \  | \  \Box \phi
     \  | \  A : \phi \  | \  \forall
     \phi \]
\end{definition}

Our approach in hybrid logic is to employ a universal modality along with
nominals.  This framework presents a logic where the agent may reason about
various labeled scenarios.    From these intuitions we have the following definition:

\begin{definition}
  \label{hybridsemantics}Let a \tmtextbf{hybrid model} $\mathbbm{M}= \langle
  W, V, R, \mathcal{A}, \ell \rangle$ be an awareness model as in Definition
  \ref{awarenessmodels}, along with a partial labeling function $\ell : \Psi
  \nrightarrow W$
  
  The semantics for $\vDash$ are the same as in Definition
  \ref{awarenessmodels0}, with the following extensions
  \begin{eqnarray*}
    \mathbbm{M}, w \vDash \forall \phi & \iff & \forall v \in
    W.\mathbbm{M}, v \vDash \phi\\
    \mathbbm{M}, w \vDash i & \iff & \ell (i) = w
  \end{eqnarray*}
\end{definition}

The other connectives and operators are defined as usual. We also employ a
the following shorthand:

\begin{notation}
  We employ the following special abbreviations:
  \begin{eqnarray}
    \text{\tmtextup{@}}^i \phi \assign \forall (i \rightarrow \phi)
    \hspace{2em} & \exists \phi \assign \neg \forall \neg \phi \hspace{3em} &
    \circlearrowleft^i \assign \diamondsuit i \nonumber
  \end{eqnarray}
\end{notation}

We have a validity reflecting one of the axioms we saw in
{\S}\ref{awarenesslogic}:
\[ \vDash \circlearrowleft^i \rightarrow \Box \phi \rightarrow
   \text{\tmtextup{@}}^i \phi \]
This can be read as ``If the agent's knowledge base is sound at world $i$,
then if they can deduce something from their knowledge base, what they deduce
must be true at world $i$.'.  In a way, this may be considered as
{\tmem{relativising}} knowledge particular named situations.



The semantics in Definition \ref{hybridsemantics} obviates the \tmtextbf{SND}
principle we previously presented, since there is no explicit
$\circlearrowleft$ symbol in this setting. \tmtextbf{CSQ} still makes sense
without modification, however.  The following gives a logic for hybrid models
making true \tmtextbf{CSQ}:
\begin{table}[h]
\centering
  \begin{tabular}{ll}
    $\vdash \phi \rightarrow \psi \rightarrow \phi$ & $\vdash \forall (\phi
    \rightarrow \psi) \rightarrow \forall \phi \rightarrow \forall \psi$\\
    $\vdash (\phi \rightarrow \psi \rightarrow \chi) \rightarrow (\phi
    \rightarrow \psi) \rightarrow \phi \rightarrow \chi$ & $\vdash \forall
    \phi \rightarrow \phi$\\
    $\vdash ((\phi \rightarrow \bot) \rightarrow (\psi \rightarrow \bot))
    \rightarrow \psi \rightarrow \phi$ & $\vdash \forall \phi \rightarrow
    \forall \forall \phi$\\
    $\vdash \Box(\phi \rightarrow \psi) \rightarrow \Box \phi \rightarrow \Box
    \psi$ & $\vdash \exists \phi \rightarrow \forall \exists \phi$\\
    $\vdash A : \phi \rightarrow \Box \phi$ & $\vdash \forall \phi \rightarrow
    \Box \phi$\\
    & $\vdash i \rightarrow \phi \rightarrow \text{\tmtextup{@}}^i \phi$\\
    &  
  \end{tabular}

\begin{tabular}{lllll}
      $\displaystyle\frac{\vdash \phi \rightarrow \psi \ \ \vdash \phi}{\vdash
      \psi}$ & \ \  & $\displaystyle \frac{\vdash \phi}{\vdash \Box \phi}$ &
      \ \  & $\displaystyle \frac{\vdash \phi}{\vdash \forall \phi}$
    \end{tabular}
  \caption{\label{logic2}Hybrid Awareness Logic for \tmtextbf{CSQ}}
\end{table}

\begin{theorem}
  \label{completeness2}Assuming an infinite store of nominals $\Psi$, the
  hybrid awareness logic presented is sound and weakly complete for all hybrid
  models making true $\textup{\tmtextbf{CSQ}}_{\vdash'}$ for some
  suitable logic $\vdash'$.
\end{theorem}
\begin{proof}
  As before, soundness is trivial.  Completeness is based on a modification
  of techniques found in {\cite[chapter 5]{boolos_logic_1995}}.  Assume
  that $\nvdash \psi$ and let $\Sigma_0$ be the set of subformulae of $\psi$,
  and let $\Upsilon$ be the set of nominals occurring in $\psi$.  Define:
  
  \begin{align*}
    {\Sigma}_1 &
    {\assign}{\Sigma}_0{\cup}\{{\forall}{\phi}{\hspace{1em}}|{\hspace{1em}}\Box{\phi}{\in}{\Sigma}_0\}\\
    {\Sigma}_2 &
    {\assign}{\Sigma}_1{\cup}\{\text{@}_i{\phi}{\hspace{1em}},i{\rightarrow}{\phi}|{\hspace{1em}}{\phi}{\in}{\Sigma}_1{\hspace{1em}}\&{\hspace{1em}}i{\in}{\Upsilon}\}\\
    {\Sigma}_3 &
    {\assign}{\Sigma}_2{\cup}\{{\neg}{\phi}{\hspace{1em}}|{\hspace{1em}}{\phi}{\in}{\Sigma}_2\}
  \end{align*}
  
  It is easy to see that $\Sigma_3$ is finite and closed under subformulae and
  single negations. Let $\mathbbm{M}= \langle W, V, R_{\Box}, \sim_{\forall},
  \mathcal{A} \rangle$ be the finite canonical model formed of maximally
  consistent subsets of $\Sigma_3$, where everything is defined as usual,
  except $\sim_{\forall}$, which obeys the following specification:
  
  \begin{align*}
    w {\sim}_{{\forall}}v & {\Longleftrightarrow}(({\forall}{\phi}){\in}w
    {\Longleftrightarrow}({\forall}{\phi}){\in}v)
  \end{align*}
  
  This model makes true the following properties:
  \begin{enumerateroman}
    \item $\sim_{\forall}$ is an equivalence relation
    
    \item $R_{\Box} \subseteq \sim_{\forall}$
    
    \item In each equivalence class specified by $\sim_{\forall}$, there is at
    most one world making true $i$ for all $i \in \Upsilon$
  \end{enumerateroman}
  As in our previous construction we have that there is some world $w$ where
  $\mathbbm{M}, w \nvDash \psi$.  Let $\mathbbm{M}^1 = \langle W', V',
  R_{\Box}', \sim_{\forall}', \mathcal{A}' \rangle$ be the submodel generated
  by $\{w\}$; we have that $\mathbbm{M}', w \nvDash \psi$ (see {\cite[chapter
  2.1]{blackburn_modal_2001}} for details on generated submodels).  In this
  model $\forall$ is a universal modality and either $\left\llbracket i
  \right\rrbracket^{\mathbbm{M}'} = \varnothing$ or $\left\llbracket i
  \right\rrbracket^{\mathbbm{M}'} =\{v\}${\footnote{As per the usual
  convention, here $\left\llbracket \phi \right\rrbracket^{\mathbbm{M}}$
  denotes $\{w \in W \hspace{1em} | \hspace{1em} \mathbbm{M}, w \vDash
  \phi\}$.  We will drop $\mathbbm{M}$ where it is unambiguous.}}. Since the
  store $\Psi$ of nominals is infinite and $\Upsilon$ is finite, we have that
  $\Psi \backslash \Upsilon$ is infinite, so there is some injection $\iota :
  W' \hookrightarrow \Psi \backslash \Upsilon$ which assigns a fresh nominal
  to every world.  Let $\mathbbm{M}'' = \langle W'', V'', R'', \mathcal{A}'',
  \ell \rangle$, where:
  
  \begin{center}
    \begin{tabular}{lllll}
      $\bullet$ & $W'' \assign W'$ &  & $\bullet$ & $V'' (p) \assign V' (p)$\\
      $\bullet$ & $R'' \assign R'_{\Box}$ & {\hspace{3em}} & $\bullet$ &
      $\mathcal{A}'' (v) \assign \{\phi \hspace{1em} | \hspace{1em} \phi \in
      \mathcal{A}' (v)\} \cup \{\neg \iota (u) \hspace{1em} | \hspace{1em}
      \neg v R' u\}$\\
      \begin{tabular}{ll}
        $\bullet$ & $\ell (i) \assign \left\{ \begin{array}{ll}
          w & i \in \Upsilon \hspace{1em} \& \hspace{1em} \mathbbm{M}, w
          \vDash i\\
          \iota^{- 1} (i) & i \in \iota [W]\\
          \text{undefined} & \text{o/w}
        \end{array} \right.$
      \end{tabular} & $R'' \assign R'_{\Box}$ &  &  & 
    \end{tabular}
  \end{center}
  
  The the worlds is $\mathbbm{M}'$ and $\mathbbm{M}''$ agree on all
  subformulae of $\psi$, hence $\mathbbm{M}'', w \nvDash \psi$. The same
  reasoning as in the proof of Theorem \ref{completeness1} establishes and
  that the logic in Table \ref{logic2} itself is sound for $\mathbbm{M}''$ and
  makes true \tmtextbf{CSQ}, establishing completeness.
\end{proof}
%%% Local Variables: 
%%% mode: latex
%%% TeX-master: "paper"
%%% End: 


%%% Local Variables: 
%%% mode: latex
%%% TeX-master: "paper"
%%% End: 


\section{Logics of Multiple Knowledge Bases}

So far, we have only dealt with systems that reason over a single knowledge
base.  In this section, we present various logics for reasoning over multiple
knowledge bases.

\subsection{Simple Justification Logic}
\label{simple_justification}
Justification Logic (JL) was originally developed by Artemov as the
\tmtextit{Logic of Proofs} (LP) {\cite{artemov_logic_1994}}.  Its original
purpose was to provide a framework for reasoning about explicit provability in
Peano Arithmetic.  The first introduction of Justification Logic can be found
in {\cite{artemov_introducing_2005}}, where Artemov and Nogina propose LP as a
logic for reasoning about evidence.  In {\cite{fitting_logic_2005}}, Melvin
Fitting provided JL with Kripke model based semantics.

The principle innovation of LP/JL is to extend awareness logic, such that
awareness operations are now \tmtextit{proof terms}.  Informally, we say that
``a proof term $X$ witnesses proposition $\phi$'', denoted ``$X : \phi$'',
whenever $X$ represents a proof of $\phi$.  Proofs may have
{\tmstrong{multiple-conclusions}} in this system, so it is possible that $X :
\phi$ and $X : \psi$ can be true where $\phi \neq \psi$.  In Fitting's Kripke
semantics, $X : \phi$ means that $\phi$ is in awareness set corresponding to
$X$.  Proof terms are thought be operating in a \tmtextit{multi-conclusion}
proof system, so the same proof term may witness many different propositions.



The logics LP/JL include operators over proofs terms, so that new proof terms
may be assembled.  One operation of particular interest to us is
{\tmem{choice}}, denoted $\oplus$.  The expression ``$X \oplus Y : \phi$''
denotes that either $X$ or $Y$ are proofs witnessing $\phi$.  There are other
operations which correspond to \tmtextit{modus ponens} and
\tmtextit{proof-theoretic reflection}, however we do not consider these
operations here.



In this section we consider a simplified form of JL suitable for reasoning
over multiple knowledge bases, which we call \tmtextit{simple justification
logic} (SJL). However, instead of thinking of terms as representing proofs, we
use them to represent knowledge bases. This is in the spirit of JL as a logic
of evidence: we consider each knowledge base to be a corpus of evidence. \
Each term ``names'' a different knowledge base at a particular world.  Terms
may not refer to the same knowledge base at different worlds, just as in how
in awareness logic the agent need not be aware of the same formulae at
different worlds.  Finally, we will make use of JL's {\tmem{choice}} operator
as a mechanism for forming the union of knowledge bases, creating new ones.

\begin{definition}
  Let $\Pi$ be a set of primitive terms.  Define $\tau (\Pi)$ with the
  following grammar:
  \[ X \hspace{1em} : : = \hspace{1em} t \in \Pi \hspace{1em} | \hspace{1em} X
     \oplus Y \]
  
  
  Let $\Phi$ be a set of letters and $\Pi$ a set of primitive terms, and
  define the language $\mathcal{L}_{\tmop{SJL}} (\Phi, \Pi)$ as:
  \[ \phi \hspace{1em} : : = \hspace{1em} p \in \Phi \hspace{1em} |
     \hspace{1em} \circlearrowleft_X \hspace{1em} | \hspace{1em} \bot
     \hspace{1em} | \hspace{1em} \phi \rightarrow \psi \hspace{1em} |
     \hspace{1em} \Box_X \phi \hspace{1em} | \hspace{1em} X : \phi \]
  where $X \in \tau (\Pi)$
\end{definition}

\begin{definition}
  \label{justmodels}A {\tmstrong{simple justification model}} $\mathbbm{M}=
  \langle W, V, R, \mathcal{A} \rangle$ is a Kripke model with a valuation $V
  : \Phi \cup \{\circlearrowleft_X \hspace{1em} | \hspace{1em} X \in \tau
  (\Pi)\} \rightarrow 2^W$, an indexed relation $R : \tau (\Pi) \rightarrow
  2^{W \times W}$, along with a modified awareness function $\mathcal{A} : W
  \times \tau (\Pi) \rightarrow 2^{\mathcal{L}_{\tmop{SJL}} (\Phi, \Pi)}$. \
  In practice we will denote $\mathcal{A} (w, X)$ by a curried shorthand,
  namely $\mathcal{A}_w (X)$.
  
  
  
  The semantics for $\vDash$ have the following modifications:
  \begin{eqnarray*}
    \mathbbm{M}, w \vDash \Box_X \phi & \text{iff} & \text{for all $v \in W$
    where $w R_X v$ we have $\mathbbm{M}, v \vDash v$}\\
    \mathbbm{M}, w \vDash X : \phi & \text{iff} & \phi \in \mathcal{A}_w (X)
  \end{eqnarray*}
\end{definition}

\begin{definition}
  The following defines properties a simple justification model may make true:
  
  \begin{descriptiondash}
    \item[JCSQ] $\mathbbm{M}, w \vDash \Box_X \phi \text{ iff } \tmop{Th}
    (\mathbbm{M}) \cup \mathcal{A}_w (X) \vdash \phi$
    
    \item[JSND] $\mathbbm{M}, w \vDash \circlearrowleft_X \text{ iff
    $\mathbbm{M}, w \vDash \phi$ for all $\phi \in \mathcal{A}_w (X)$}$
    
    \item[CHOICE] $\mathcal{A}_w (X \oplus Y) = \mathcal{A}_w (X) \cup
    \mathcal{A}_w (Y)$
  \end{descriptiondash}
  
  as before, $\vdash$ is any sound logical consequence relation for $\vDash$
  with \tmtextbf{modus ponens} and \tmtextbf{reflection}
\end{definition}

The above semantics are similar to the ones given in previous sections. \
\tmtextbf{JCSQ} is the same as \tmtextbf{CSQ} from {\S}\ref{awarenesslogic},
only it is relativised to a knowledge base denoted by $X$ at $w$.  The
awareness logic of knowledge bases is special case of simple justification
logic where there is only one term.



Simplified justification logic is given in Table \ref{logic5}.  We assert
without proof that this is a conservative extension of the awareness logic in
{\S}\ref{awarenesslogic}.



\begin{table}[h]
  \begin{tabular}{ll}
    $\vdash \phi \rightarrow \psi \rightarrow \phi$ & \\
    $\vdash (\phi \rightarrow \psi \rightarrow \chi) \rightarrow (\phi
    \rightarrow \psi) \rightarrow \phi \rightarrow \chi$ & \\
    $\vdash ((\phi \rightarrow \bot) \rightarrow (\psi \rightarrow \bot))
    \rightarrow \psi \rightarrow \phi$ & \\
    $\vdash \Box_X (\phi \rightarrow \psi) \rightarrow \Box_X \phi \rightarrow
    \Box_X \psi$ & \\
    $\vdash (X : \phi) \rightarrow \Box_X \phi$ & \\
    $\vdash \circlearrowleft_X \rightarrow \Box_X \phi \rightarrow \phi$ & \\
    $\vdash (X : \phi) \rightarrow (Y : \phi) \rightarrow X \oplus Y : \phi$ &
    \\
    $\vdash (X \oplus Y : \phi) \rightarrow X : \phi$ & $\vdash (X \oplus Y :
    \phi) \rightarrow Y : \phi$\\
    $\vdash \circlearrowleft_X \rightarrow \circlearrowleft_Y \rightarrow
    \circlearrowleft_{X \oplus Y}$ & \\
    $\vdash \circlearrowleft_{X \oplus Y} \rightarrow \circlearrowleft_X$ &
    $\vdash \circlearrowleft_{X \oplus Y} \rightarrow \circlearrowleft_Y$\\
    $\vdash \Box_X \phi \rightarrow \Box_{X \oplus Y} \phi$ & \\
    $\vdash \Box_{X \oplus X} \phi \rightarrow \Box_X \phi$ & $\vdash \Box_{(X
    \oplus X) \oplus Y} \phi \rightarrow \Box_{X \oplus Y} \phi$ \\
    $\vdash \Box_{X \oplus Y} \phi \rightarrow \Box_{Y \oplus X} \phi$ &
    $\vdash \Box_{(X \oplus Y) \oplus Z} \phi \rightarrow \Box_{(Y \oplus X)
    \oplus Z} \phi$\\
    $\vdash \Box_{(X \oplus Y) \oplus Z} \phi \rightarrow \Box_{X \oplus (Y
    \oplus Z)} \phi$ & \\
    & \\
    \begin{tabular}{lll}
      $\frac{\vdash \phi \rightarrow \psi \hspace{4em} \vdash \phi}{\vdash
      \psi}$ & {\hspace{6em}} & $\frac{\vdash \phi}{\vdash \Box_X \phi}$
    \end{tabular} & 
  \end{tabular}
  \caption{\label{logic5}Simple Justification Logic}
\end{table}

\begin{theorem}
  \label{completeness5}Assuming an infinite store of proposition letters
  $\Phi$, SJL is sound and weakly complete for simple justification models
  making true \tmtextbf{JCSQ}, \tmtextbf{JSND} and \tmtextbf{CHOICE}
\end{theorem}

\begin{proof}
  As in the previous proofs, we only show completeness.  Assume that $\nvdash
  \psi$.  Let $\Xi \subseteq \tau (\Pi)$ be all of the subterms occurring in
  $\psi$; we assume that $\Xi$ is non-empty, since otherwise we may obtain the
  theorem via the usual completeness theorem for classical propositional
  logic. Let $\Sigma_0$ be the subformulae of $\psi$.  Define:
  
  \begin{align*}
    {\Sigma}_1 &
    {\assign}{\Sigma}_0{\cup}\{\Box_Y{\phi}{\hspace{1em}}|{\hspace{1em}}\Box_X{\phi}{\in}{\Sigma}_0{\hspace{1em}}\&{\hspace{1em}}Y{\in}{\Xi}\}\\
    {\Sigma}_2 &
    {\assign}{\Sigma}_1{\cup}\{{\neg}{\phi}{\hspace{1em}}|{\hspace{1em}}{\phi}{\in}{\Sigma}_1\}
  \end{align*}
  
  As in the proof of Theorem \ref{completeness2}, it is simple to verify that
  $\Sigma_2$ is closed under subformulae and single negations.  Let
  $\mathbbm{M} \assign \langle W, V, R, \mathcal{A} \rangle$ be the finite
  canonical model formed of maximally consistent subsets of $\Sigma_2$, where
  $V$ and $\mathcal{A}$ are defined as usual, and $R : \Xi \rightarrow 2^{W
  \times W}$ is defined by:
  \[ w R_X v \Longleftrightarrow \{\phi \hspace{1em} | \hspace{1em} \Box_X
     \phi \in w\} \subseteq v \]
  As in previous constructions we may prove the usual Lindenbaum and Truth
  Lemmas an use them obtain a world $w \in W$ where $\mathbbm{M}, w \nvDash
  \psi$.
  
  
  
  Next, define an operator $\twonotes : \tau (\Pi) \rightarrow 2^{\Pi}$ where:
  \[ \twonotes (X) \assign \{t \in \Pi \hspace{1em} | \hspace{1em} t \text{
     occurs in } X\} \]
  In other words, $\twonotes (X)$ is the set of atomic terms in $X$.  As in
  our construction for Theorem \ref{completeness1}, $\mathbbm{M}$ makes true
  certain properties, along with some new properties:
  \begin{enumeratenumeric}
    \item \label{fin}$W$ and $\mathcal{A}_v (X)$ are finite, and $\phi \in
    \mathcal{A}_v (X)$ only if $\phi \in \Sigma_2$
    
    \item if $\mathbbm{M}, v \vDash X : \phi$ then $\mathbbm{M}, v \vDash
    \Box_X \phi$
    
    \item \label{refl}if $\mathbbm{M}, v \vDash \circlearrowleft_X$ then $v
    R_X v$
    
    \item \label{sndness}For all $X \oplus Y \in \Xi$, $\mathbbm{M}, v \vDash
    \circlearrowleft_{X \oplus Y}$ if and only if $\mathbbm{M}, v \vDash
    \circlearrowleft_X$ and $\mathbbm{M}, v \vDash \circlearrowleft_Y$
    
    \item \label{union}For all $X \oplus Y \in \Xi$, $\mathcal{A}_v (X \oplus
    Y) = \mathcal{A}_v (X) \cup \mathcal{A}_v (Y)$
    
    \item \label{sub}For all $X, Y \in \Xi$, if $\twonotes (X) \subseteq
    \twonotes (Y)$ then $R_Y \subseteq R_X$
  \end{enumeratenumeric}
  As in our previous constructions, it is necessary to refine this model using
  bisimulations to achieve properties which are not modally definable.  In
  particular, we shall strengthen (\ref{refl}) to a biconditional and
  (\ref{sub}) to:
  
  \begin{center}
    \begin{tabular}{ll}
      \tmtextbf{6$'$.} & For all $X, Y \in \Xi$, $R_{X \oplus Y} = R_X \cap
      R_Y$
    \end{tabular}
  \end{center}
  
  
  
  We first strengthen (\ref{refl}) by constructing $\mathbbm{M}' \assign
  \langle W', V', R', \mathcal{A}' \rangle$, just as in Theorem
  \ref{completeness1}: {\hspace*{\fill}}
  \begin{itemizedot}
    \item $W' \assign W \uplus W$
    
    \item $V' (p) \assign \{v_l, v_r \hspace{1em} | \hspace{1em} v \in V
    (p)\}$
    
    \item $R'_X \assign \{(v_l, u_r), (v_r, u_l) \hspace{1em} | \hspace{1em} v
    R_X u\} \cup \{(v_l, v_l), (v_r, v_r) \hspace{1em} | \hspace{1em}
    \mathbbm{M}, v \vDash \circlearrowleft_X \}$
    
    \item $\mathcal{A}' (v_l, X) \assign \mathcal{A}' (v_r, X) \assign
    \mathcal{A} (v, X)$
  \end{itemizedot}
  The same bisimulation relation $Z$ previously given suffices.
  
  
  
  We next make a strengthened the model where (\ref{sub}$'$) holds.  Define
  $\mathbbm{M}'' \assign \langle W'', V'', R'', \mathcal{A}'' \rangle$ such
  that{\footnote{Let $\{i_X \hspace{1em} | \hspace{1em} X \in \Xi\}$ be the
  family of canonical injections into the coproduct $\uplus_{\Xi} W$. As per
  our previous convention, we use $v_X$ as shorthand for $i_X (v)$.}}:
  \begin{itemizedot}
    \item $W'' \assign \uplus_{\Xi} W'$
    
    \item $V'' (p) \assign \{v_X \hspace{1em} | \hspace{1em} X \in \Xi
    \hspace{1em} \& \hspace{1em} v \in V' (p)\}$
    
    \item $R''_X \assign \{(v_Y, u_Z) \hspace{1em} | \hspace{1em} Y, Z \in \Xi
    \hspace{1em} \& \hspace{1em} v R'_Z u \hspace{1em} \& \hspace{1em}
    \twonotes (X) \subseteq \twonotes (Z)\}$
    
    \item $\mathcal{A}'' (v_Y, X) \assign \mathcal{A}' (v, X)$
  \end{itemizedot}
  Note that this construction makes use of the assumption that $\Xi$ is
  non-empty.  Let $Z \assign \{(v, v_X) \hspace{1em} | \hspace{1em} X \in \Xi
  \hspace{1em} \& \hspace{1em} v \in W' \}$; it is straightforward to prove
  that $Z$ is a bisimulation between $\mathbbm{M}'$ and $\mathbbm{M}''$, for
  all relations corresponding to terms $X \in \Xi$. Hence $\mathbbm{M}'', w_X
  \nvDash \psi$ for some $w_X \in W''$.  In addition, $\mathbbm{M}''$
  inherits (\ref{fin}) through (\ref{sub}) from $\mathbbm{M}'$, as well as the
  converse of (\ref{refl}).  Next observe, forall $X \oplus Y \in \Xi$:
  \begin{eqnarray*}
    \text{$(v_V, u_Z) \in R''_X \cap R''_Y$} & \Longleftrightarrow & \exists Z
    \in \Xi . \text{$v R'_Z u \hspace{1em} \& \hspace{1em} \twonotes (X)
    \subseteq \twonotes (Z) \hspace{1em} \& \hspace{1em} \twonotes (Y)
    \subseteq \twonotes (Z)$}\\
    & \Longleftrightarrow & \exists Z \in \Xi . \text{$v R'_Z u \hspace{1em}
    \& \hspace{1em} \twonotes (X) \cup \twonotes (Y) \subseteq \twonotes
    (Z)$}\\
    & \Longleftrightarrow & \exists Z \in \Xi . \text{$v R'_Z u \hspace{1em}
    \& \hspace{1em} \twonotes (X \oplus Y) \subseteq \twonotes (Z)$}\\
    & \Longleftrightarrow &  \text{$(v_V, u_Z) \in R''_{X \oplus Y}$}
  \end{eqnarray*}
  We now turn to our final construction. As in previous constructions, let
  $\iota : W'' \hookrightarrow \Phi \backslash \Psi$ be an injection assigning
  fresh nominals to worlds.  Define $\mathbbm{M}''' \assign \langle W''',
  V''', R''', \mathcal{A}''' \rangle$ just as in the final construction in the
  proof of Theorem \ref{completeness1}, except that $R'''_X$ and
  $\mathcal{A}_v''' (X)$ are defined inductively as follows:
  
  
  
  \begin{center}
    \begin{tabular}{lll}
      $\bullet$ & $R'''_t \assign \left\{ \begin{array}{ll}
        R''_t & t \in \Xi\\
        W''' \times W''' & o / w
      \end{array} \right.$ & \\
      $\bullet$ & $R'''_{X \oplus Y} \assign R'''_X \cap R'''_Y$ & \\
      &  & \\
      $\bullet$ & $\mathcal{A}'''_v (t) \assign \{\phi \hspace{1em} | \phi \in
      \mathcal{A}''_v \} \cup \{\neg \iota (u) \hspace{1em} | \hspace{1em}
      \neg v R'''_t u\}$ & where $t \in \Pi$\\
      $\bullet$ & $\mathcal{A}'''_v (X \oplus Y) \assign \mathcal{A}'''_v (X)
      \cup \mathcal{A}'''_v (Y)$ & 
    \end{tabular}
  \end{center}
  
  
  
  
  
  Induction over complexity of subformulae $\phi$ of $\psi$ yields
  $\mathbbm{M}'', w \vDash \phi \Longleftrightarrow \mathbbm{M}''', w \vDash
  \phi$, so we know that there is some world $w \in W'''$ such that
  $\mathbbm{M}''', w \nvDash \psi$.  All that is left is to illustrate that
  $\mathbbm{M}'''$ has the properties we desire.
  
  
  
  First note that by definition, $\mathbbm{M}'''$ makes true
  \tmtextbf{CHOICE}.  Next, an induction argument on the complexity of a
  terms $X$, making essential use of (\ref{sub}$'$), yields:
  \[ u R_X''' v \Longleftrightarrow \mathbbm{M}''', v \vDash \bigwedge
     \mathcal{A}'''_v (X) \]
  With this, $\mathbbm{M}'''$ makes true \tmtextbf{JSND} since it inherits
  (\ref{refl}).  Finally, we note that we may repeat our previous arguments
  from Theorem \ref{completeness1} to prove \tmtextbf{JCSQ}.
\end{proof}
%%% Local Variables: 
%%% mode: latex
%%% TeX-master: "paper"
%%% End: 


\subsection{Lattice Justification Logic}
\label{lattice_justification}
Implicit in our previous presentation of SJL is the following
principle: 
\todo[noline]{THIS SECTION IS  PROBABLY WRONG}
 $\oplus$ behaves like a union
operation for knowledge bases, and dually like an intersection operation on
accessibility relations.  It is natural to think of $\oplus$ as a join/meet
operator over a {\tmem{semi-lattice}}.  In this section we make this
explicit, by increasing the expressive power of simple JL to express the order
theory inherent in knowledge bases.  We call the novel logic presented in
this section \tmtextit{Lattice Justification Logic} (LJL){\footnote{We note
that ``Semi-Lattice Justification Logic'' may be a more appropriate name,
however we have decided to shorten semic-lattice to lattice in this article
for purposes of concision.}}.  Lattice JL may be considered a novel logic for
studying the dynamics of \tmtextit{theory change}, where the logic employed by
the changing theories could be LJL itself.



\begin{definition}
  $\mathcal{L}_{\tmop{LJL}} (\Phi, \Pi)$ extends $\mathcal{L}_{\tmop{SJL}}
  (\Phi, \Pi)$ to:
  \[ \phi \hspace{1em} : : = \hspace{1em} p \in \Phi \hspace{1em} |
     \hspace{1em} \circlearrowleft_X \hspace{1em} | \hspace{1em} \bot
     \hspace{1em} | \hspace{1em} \phi \rightarrow \psi \hspace{1em} |
     \hspace{1em} \Box_X \phi \hspace{1em} | \hspace{1em} X : \phi
     \hspace{1em} | \hspace{1em} X \sqsubseteq Y \]
  where $X, Y \in \tau (\Pi)$
\end{definition}

\begin{definition}
  \label{justmodels}A {\tmstrong{lattice justification model}} $\mathbbm{M}=
  \langle W, V, R, \mathcal{A}, \preccurlyeq \rangle$ is a simple
  justification model with a new relation $\preccurlyeq : W \rightarrow
  2^{\tau (\Pi) \times \tau (\Pi)}$ between terms, indexed by worlds.  The
  semantics of $\sqsubseteq$ correspond to $\preccurlyeq$ as follows:
  \[ \mathbbm{M}, w \vDash X \sqsubseteq Y \Longleftrightarrow X
     \preccurlyeq_w Y \]
\end{definition}

\begin{definition}
  A lattices justification model is said to make true \tmtextbf{LATTICE} if
  and only if:
  \[ \mathbbm{M}, w \vDash X \sqsubseteq Y \Longleftrightarrow \mathcal{A}_w
     (X) \subseteq \mathcal{A}_w (Y) \]
\end{definition}

Table \ref{logic6} lists the axioms for the logic of lattice justification
models with the properties we have been investigating:

\begin{table}[h]
  \begin{tabular}{ll}
    $\vdash \phi \rightarrow \psi \rightarrow \phi$ & \\
    $\vdash (\phi \rightarrow \psi \rightarrow \chi) \rightarrow (\phi
    \rightarrow \psi) \rightarrow \phi \rightarrow \chi$ & \\
    $\vdash ((\phi \rightarrow \bot) \rightarrow (\psi \rightarrow \bot))
    \rightarrow \psi \rightarrow \phi$ & \\
    $\vdash \Box_X (\phi \rightarrow \psi) \rightarrow \Box_X \phi \rightarrow
    \Box_X \psi$ & \\
    $\vdash (X : \phi) \rightarrow \Box_X \phi$ & \\
    $\vdash \circlearrowleft_X \rightarrow \Box_X \phi \rightarrow \phi$ & \\
    $\vdash (X : \phi) \rightarrow (Y : \phi) \rightarrow X \oplus Y : \phi$ &
    \\
    $\vdash X \sqsubseteq Y \rightarrow (X : \phi) \rightarrow Y : \phi$ & \\
    $\vdash \circlearrowleft_X \rightarrow \circlearrowleft_Y \rightarrow
    \circlearrowleft_{X \oplus Y}$ & \\
    $\vdash X \sqsubseteq Y \rightarrow \circlearrowleft_Y \rightarrow
    \circlearrowleft_X$ & \\
    $\vdash X \sqsubseteq Y \rightarrow \Box_X \phi \rightarrow \Box_Y \phi$ &
    \\
    $\vdash X \sqsubseteq X$ & \\
    $\vdash X \sqsubseteq Y \rightarrow Y \sqsubseteq Z \rightarrow X
    \sqsubseteq Z$ & \\
    $\vdash X \sqsubseteq Y \rightarrow X \sqsubseteq Z \oplus Y$ & $\vdash X
    \sqsubseteq Y \rightarrow X \sqsubseteq Y \oplus Z$\\
    $\vdash X \sqsubseteq Z \rightarrow Y \sqsubseteq Z \rightarrow X \oplus Y
    \sqsubseteq Z$ & \\
    & \\
    \begin{tabular}{lll}
      $\frac{\vdash \phi \rightarrow \psi \hspace{4em} \vdash \phi}{\vdash
      \psi}$ & {\hspace{6em}} & $\frac{\vdash \phi}{\vdash \Box_X \phi}$
    \end{tabular} & 
  \end{tabular}
  \caption{\label{logic6}Lattice Justification Logic}
\end{table}

\begin{theorem}
  \label{completeness6}Assuming an infinite store of proposition letters
  $\Phi$, LJL is sound and weakly complete for lattice justification models
  making true \tmtextbf{JCSQ}, \tmtextbf{JSND}, \tmtextbf{CHOICE} and
  \tmtextbf{LATTICE}
\end{theorem}

\begin{proof}
  Assume $\nvdash \psi$.  Let $\mathbbm{M} \assign \langle W, V, R,
  \mathcal{A}, \preccurlyeq \rangle$ be the same as the finitary canonical
  model we initially constructed in Theorem \ref{completeness5}, only
  $\preccurlyeq$ is defined as:
  \[ X \preccurlyeq_w Y \Longleftrightarrow w \vdash X \sqsubseteq Y \]
  We may deduce that there is a world $w \in W$ such that $\mathbbm{M}, w
  \nvDash \psi$.  This canonical model makes true properties (\ref{fin})
  through (\ref{union}) we listed in Theorem \ref{completeness5}, as well as a
  certain new properties:
  
  \begin{descriptioncompact}
    \item[6$\preccurlyeq$] For all $X, Y \in \Xi$, if $X \preccurlyeq_v Y$
    then $R_Y [v] \subseteq R_X [v]$
    
    \item[7] For all $X, Y \in \Xi$, if $X \preccurlyeq_v Y$ then
    $\mathcal{A}_v (X) \subseteq \mathcal{A}_v (Y)$
    
    \item[8] If $X \nin \Xi$, then $R_X = W \times W$ and $\mathcal{A}_v (X) =
    \varnothing$ for all $v \in W$
    
    \item[9] $\langle \preccurlyeq_v, \oplus \rangle$ is a join-semilattice
    over $\tau (\Pi)$ for all $v \in W$
  \end{descriptioncompact}
  
  As in previous constructions, completeness is achieved by through a series
  of model refinements.  Note that property 8 has been implicitly true in all
  of our previous constructions, although we have not been interested in
  enforcing it to be inherited by model refinements previously.
  
  
  
  We first construct a model $\mathbbm{M}'$ which is bisimular to
  $\mathbbm{M}$, where (\ref{refl}) is strengthened to a biconditional.  This
  is done exactly as we proceeded in previous constructions. We note that the
  notion of ``bisimulation'' here includes that two bisimular worlds $v$ and
  $u$ must have isomorphic semi-lattices $\preccurlyeq_v$ and
  $\preccurlyeq_w$.  This is achieved by enforcing that $\preccurlyeq_{v_l}
  \assign \preccurlyeq_{v_r} \assign \preccurlyeq_v$. As before, we have that
  $\mathbbm{M}', w \nvDash \psi$ for some world $w \in W'$.
  
  
  
  We next strengthen (6$\preccurlyeq$) to a biconditional.  Define
  $\mathbbm{M}'' \assign \langle W'', V'', R'', \mathcal{A}'' \rangle$ such
  that:
  \begin{itemizedot}
    \item $W'' \assign \uplus_{\Xi} W'$
    
    \item $V'' (p) \assign \{v_X \hspace{1em} | \hspace{1em} X \in \Xi
    \hspace{1em} \& \hspace{1em} v \in V' (p)\}$
    
    \item $R''_X \assign \{(v_Y, u_Z) \hspace{1em} | \hspace{1em} Y, Z \in \Xi
    \hspace{1em} \& \hspace{1em} v R'_Z u \hspace{1em} \& \hspace{1em} X
    \preccurlyeq_v Z\}$
    
    \item $\mathcal{A}'' (v_Y, X) \assign \mathcal{A}' (v, X)$
  \end{itemizedot}
  $\mathbbm{M}''$ is bisimular to $\mathbbm{M}'$ by the same mechanism as the
  previous construction.  The important feature of this structure is to
  observe that the converse of (6$\preccurlyeq$) holds:
  \begin{eqnarray*}
    R_Y [v] \subseteq R_X [v] & \Longleftrightarrow & \forall Z \in \Xi
    \forall u_Z \in W'' (v_Q R'_Z u_Z \Longrightarrow v_Q R_X' u_Z)\\
    & \Longleftrightarrow & \forall Z \in \Xi \forall u_Z \in W'' (v R'_Z
    \Longrightarrow v_Q R_X u_Z)
  \end{eqnarray*}
  
  
  Our first transformation strengthens (\ref{union}), (6$\preccurlyeq$a) and
  (6$\preccurlyeq$b) so that they hold for all terms in $\tau (\Pi)$, rather
  than being restricted to $\Xi$.  To this end define $\mathbbm{M}' \assign
  \langle W', V', R', \mathcal{A}', \preccurlyeq' \rangle$ where $W'$, $R'$
  and $\preccurlyeq'$ are the same as in $\mathbbm{M}$, but $R'$ and
  $\mathcal{A}'$ have the following modifications:
  \begin{eqnarray}
    R'_Y [v] & \assign & \bigcap_{X \preccurlyeq_v Y} R_X [v] \nonumber\\
    \mathcal{A}'_v (Y) & \assign & \bigcup_{X \preccurlyeq_v Y} \mathcal{A}'_v
    (X) \nonumber
  \end{eqnarray}
  
\end{proof}
%%% Local Variables: 
%%% mode: latex
%%% TeX-master: "paper"
%%% End: 


\subsection{Neighborhood Semantics}
\label{neighborhood_semantics}
Neighborhood semantics were originally developed by Dana Scott and Richard
Montague in the early 1970s as a generalization of Kripke semantics \
{\cite{montague_universal_2008,scott_advice_1970}}.  In
{\cite{fagin_belief_1988}}, Halpern and Fagin adapted neighborhood semantics
for reasoning about epistemic agents without logical omniscience.  In this
section will demonstrate how these semantics may be modified so that
neighborhoods corresponds to the logical consequences of a different knowledge
bases.  This allows for using logics with neighborhood semantics for
reasoning about multiple knowledge bases.  A point of distinction of
neighborhood semantics from JL is that it may be possible that two distinct
knowledge bases cannot be merged by the agent.

\begin{definition}
  Let $\Phi$ be a set of letters and define the language $\mathcal{L}_N (\Phi)$ as:
  \[ \phi\ : : =\ p \in \Phi\ |
    \ \bot\ |\ \phi \rightarrow \psi
    \ |\ \Box \phi\ |\ A :
     \phi\ |\ K \phi \]
\end{definition}

Just as previous semantics, ``$\Box \phi$'' is intended to mean that the agent
has an argument for $\phi$ from some knowledge base.  A novelty we present
here is that ``$K \phi$'' means that the agent has an argument for $\phi$ from
some \tmtextit{sound} knowledge base.

\begin{definition}
  \label{neighborhoodmodels}A {\tmstrong{neighborhood model}} $\mathbbm{M}=
  \langle W, V, \mathcal{N}, \mathcal{A} \rangle$ has a neighborhood function
  $\mathcal{N} : W \rightarrow 2^{2^W}$ and a multi-awareness function
  $\mathcal{A} : W \rightarrow 2^{2^{\mathcal{L}_N (\Phi)}}$
  
  
  
  The semantics for $\vDash$ have the following modifications:
  \begin{eqnarray*}
    \mathbbm{M}, w \vDash \Box \phi & \text{iff} & \text{there exists a $U \in
    \mathcal{N}_w$ where $\mathbbm{M}, v \vDash \phi$ for all $v \in U$}\\
    \mathbbm{M}, w \vDash K \phi & \text{iff} & \text{there exists a $U \in
    \mathcal{N}_w$ where $w \in U$ and $\mathbbm{M}, v \vDash \phi$ for all $v
    \in U$}\\
    \mathbbm{M}, w \vDash A : \phi & \text{iff} & \phi \in \bigcup
    \mathcal{A_w}
  \end{eqnarray*}
\end{definition}

Here we modify our previous notion of \tmtextbf{CSQ} and \tmtextbf{SND} to
match how our semantics are intended for reasoning over multiple bases:

\begin{definition}
  The following defines properties a neighborhood model may make true:
  
  \begin{descriptiondash}
    \item[NCSQ] $\mathbbm{M}, w \vDash \Box \phi \text{ iff there exists a set
    $X \in \mathcal{A}_w$ such that } \tmop{Th} (\mathbbm{M}) \cup X \vdash
    \phi$
    
    \item[NSND] $\mathbbm{M}, w \vDash K \phi \text{ iff there exists a set $X
    \in \mathcal{A}_w$ such that $\mathbbm{M}, w \vDash X$ and } \tmop{Th}
    (\mathbbm{M}) \cup X \vdash \phi$
  \end{descriptiondash}
  
  as before, $\vdash$ is any sound logical consequence relation for $\vDash$
  with \tmtextbf{modus ponens} and \tmtextbf{reflection}
\end{definition}

\begin{table}[h]
  \begin{tabular}{l}
    $\vdash \phi \rightarrow \psi \rightarrow \phi$\\
    $\vdash (\phi \rightarrow \psi \rightarrow \chi) \rightarrow (\phi
    \rightarrow \psi) \rightarrow \phi \rightarrow \chi$\\
    $\vdash ((\phi \rightarrow \bot) \rightarrow (\psi \rightarrow \bot))
    \rightarrow \psi \rightarrow \phi$\\
    $\vdash K \phi \rightarrow \Box \phi$\\
    $\vdash K \phi \rightarrow \phi$\\
    $\vdash A : \phi \rightarrow \Box \phi$\\
    \\
    \begin{tabular}{llll}
      $\frac{\vdash \phi \rightarrow \psi \hspace{4em} \vdash \phi}{\vdash
      \psi}$ & {\hspace{6em}}$\frac{\vdash \phi \rightarrow \psi}{\vdash \Box
      \phi \rightarrow \Box \psi}$ & {\hspace{6em}} & $\frac{\vdash \phi
      \rightarrow \psi}{\vdash K \phi \rightarrow K \psi}$
    \end{tabular}
  \end{tabular}
  \caption{\label{logic3}A Neighborhood Logic for \tmtextbf{NCSQ} and
  {\tmstrong{NSND}}}
\end{table}

\begin{theorem}
  Assuming an infinite store of letters $\Phi$, the logic in Table
  \ref{logic3} is sound and weakly complete for neighborhood semantics making
  true \tmtextbf{NCSQ} and \tmtextbf{NSND}.
\end{theorem}

\begin{proof}
  Assume that $\nvdash \psi$, and define the finitary canonical model
  $\mathbbm{M}= \langle W, V, \mathcal{N}, \mathcal{A} \rangle$
  where:
  \begin{itemizedot}
    \item $W \assign \text{the maximally consistent sets of subformulae of
    $\psi$, closed under single negation}$
    \item $V (p) \assign \{v \in W \  | \  p \in v\}$
    % \item $\mathcal{K}_w \assign \{S \in 2^W \  | \  \{v
    % \in W \  | \  \phi \in v\} \subseteq S \  \&
    % \  K \phi \in w\}$
    \item $\mathcal{N}_w \assign \{S \in 2^W \  | \  \llbracket \phi
  \rrbracket \subseteq S \  \&
    \  \Box \phi \in w\}$
    \item $\mathcal{A}(w) \assign \{\{\phi\} \  | \  A :
    \phi \in w\}$
  \end{itemizedot}
  % Note that in this model, there are neighborhoods for both knowledge and
  % belief; more work is necessary to find a model based on this one which
  % conforms to the semantics in Definition \ref{neighborhoodmodels}.  For now,
  % assume that $K$ is just another modality governed by the neighborhoods in
  % $\mathcal{K}_w$.  In both neighborhood functions, 
  Here $\llbracket \phi \rrbracket$ denotes $\{ v \in W\ |\ \phi
\in v\}$.
  
  The proof of the finitary Lindenbaum Lemma is straightforward,
  although the Truth Theorem is a little less routine. The proof
  proceeds by induction.  
  We only show one direction for the step for $\Box \chi$.
  Assume $\mathbbm{M}, w \vDash \Box \chi$, where the Truth Lemma has been proven to
  hold for $\chi$; we have to show $\Box \chi \in w$.  There must be some
  $S \in \mathcal{N}(w)$ such that $\mathbb{M},v \models \chi$ for all
  $v \in S$. By construction there is some $\phi$ where $\Box \phi \in
  w$ such that $\llbracket \phi \rrbracket \subseteq S$.  We argue that $\vdash \phi
  \rightarrow \chi$; for if not then by the finitary Lindenbaum lemma there is
  some $u \in W$ where $\phi \in u$ and $\chi \nin u$.  Hence $u \in S$ and
  $\mathbbm{M}, u \nvDash \chi$, a contradiction.  Since $\vdash \phi
  \rightarrow \chi$ then we know that $\vdash \Box \phi \rightarrow \Box \chi$
  by our rules, which means that $\Box \chi \in w$ by maximality.
  
  By the Truth Theorem and the Finitary Lindenbaum lemma, we have that
  there is some $w \in W$ where
  $\mathbbm{M}, w \nvDash \psi$.  Moreover, $\mathbbm{M}$ makes true a number of
  properties:
  \begin{enumeratenumeric}
    \item $W$ and $\mathcal{A}_v$ are finite for all $v \in W$, and $\psi \in
    \bigcup \mathcal{A}(w)$ only if $\psi$ is a subformula of $\phi$   
   % \item $\mathcal{K}_w \subseteq \mathcal{N}_w$
%    \item If $S \in \mathcal{K}_w$ then $w \in S$
    \item $\mathbbm{M}, v \vDash A : \phi \rightarrow \Box \phi$ for all $v
    \in W$
    \item $W \in \mathcal{N}(w)$ for all $w \in W$
 %   \item $\mathcal{N}_w$ and $\mathcal{K}_w$ are closed under supersets
  \end{enumeratenumeric}
  
  % As in the proof of Theorem \ref{completeness1}, we use a bisimulation to
  % move to a model where property (3) is strengthened to (3$'$): for all $S \in
  % \mathcal{N}_w$, $w \in S$ if and only if $S \in \mathcal{K}'_w$.  We defer
  % the reader to {\cite{hansen_bisimulation_2007,pauly_bisimulation_1999}} for
  % details regarding bisimulation in neighborhood semantics.  Define
  % $\mathbbm{M}' \assign \langle W', V', \mathcal{K}', \mathcal{N}',
  % \mathcal{A}' \rangle$ such that:
  % \begin{itemizedot}
  %   \item $W' \assign W \uplus W$ wher.  $l, r$ its associated canonical
  %   injections and $\theta (v_l) \assign \theta (v_r) \assign v$ is a
  %   left-inverse of $l, r$
    
  %   \item $V' (p) \assign \{v_l, v_r \  | \  v \in V
  %   (p)\}$
    
  %   \item $\mathcal{K}' (v_i) \assign \{S \  | \  v_i \in
  %   S \  \& \  \theta [S] \in \mathcal{K_v} \}$ where $i =
  %   l, r$
    
  %   \item $\mathcal{N}' (v_i) \assign \mathcal{K}' (v_i) \cup \{S \ 
  %   | \  v_i \nin S \  \& \  \theta [S] \in
  %   \mathcal{N_v} \}$ where $i = l, r$
    
  %   \item $\mathcal{A}' (v_l) \assign \mathcal{A}' (v_r) \assign \mathcal{A}
  %   (v)$
  % \end{itemizedot}
  
  
  % If we let $Z \assign \{(v, v_r), (v, v_l) \  | \  v \in
  % W\}$, it is straightforward to verify that $Z$ is a neighborhood
  % bisimulation.  Hence $\mathbbm{M}, w_l \nvDash \psi$.  Along with (3$'$),
  % this model makes true (1), (2), (4).  In fact, in light of ($3'$), we can
  % see that this model does not need $\mathcal{K}'$, and conforms to the
  % semantics in Definition \ref{neighborhoodmodels}.{\hspace*{\fill}}
  
  
  Let $\Lambda$ be the proposition letters occurring in
  $\psi$ and define an injection $\iota : W \hookrightarrow \Phi \backslash
  \Lambda$, assigning a nominal to every world in $\mathbbm{M}'$.  Define
  $\mathbbm{M}' \assign \langle W', V', \mathcal{N}', \mathcal{A}'
  \rangle$ such that:
  
  \begin{center}
    \begin{tabular}{lllll}
      $\bullet$ & $W' \assign W$ &  & $\bullet$ & $V' (p) \assign \left\{
      \begin{array}{ll}
        V (p) & p \in \Lambda\\
        \{v\} & p = \iota (v)\\
        \varnothing & o / w
      \end{array} \right.$\\
      $\bullet$ & $\mathcal{N}'(w) \assign \mathcal{N}(w)$ & {\hspace{3em}} &  $\bullet$ & $\mathcal{A}'(w) \assign \{\delta_w S \  |
      \  S \in \mathcal{N}'(w) \}$
    \end{tabular}
  \end{center}
  Where $\delta_w S \assign \{ \phi \in
  \bigcup \mathcal{A}(w) \  | \ S \subseteq \left\llbracket \phi
  \right\rrbracket \} \cup \{\neg \iota (u) \  |
  \  u \nin S\}$.
  
  By induction, this model agrees with $\mathbbm{M}$ on all subformulae of $\psi$, hence
  $\mathbbm{M}', w \nvDash \phi$ for some $w\in W'$.
  
  We first establish that \textbf{NON-EMPTY} is true for
  $\mathbb{M}$. We must show for every $v \in W'$ that $\mathcal{A}'(v)
  \neq \varnothing$.  We know by construction that $W' \in
  \mathcal{N}'(v)$, since $W \in \mathcal{N}(u)$ for all of the worlds
  $u$ in model $\mathbb{M}$.  Hence by construction $\delta_v W' \in
  \mathcal{A}'(v)$ by definition.

  All that is left is to show $\text{\tmtextbf{NCSQ}}_\vdash$ holds,
  and we will use the logic we are verifying completeness for. The way we have
  defined $\mathbb{M}'$ gives rise to certain properties:
  \begin{enumerateroman}
    \item For all words $v$ and all $X \in \mathcal{A}'(v)$, $X$ is finite and  $X = \delta_v S$ for some $S \in \mathcal{N}'(v)$
    \item For all $v$ and all $S \in \mathcal{N}'(v)$, $u \in S$ if and only if
    $\mathbbm{M}', u \vDash \bigwedge \delta_v S$
    \item The logic presented in Table \ref{logic3} is sound for
    $\mathbbm{M}'$
  \end{enumerateroman}

  Using the deduction theorem for $\vdash$ along with these
  properties, we may establish  $\text{\tmtextbf{NCSQ}}_\vdash$ via a familiar line of reasoning:
  
  \begin{align*}
    \text{$\exists X \in \mathcal{A}'(v) \text{ s.t. } \tmop{Th}
    (\mathbbm{M}') \cup X \vdash \phi$} & {\Longleftrightarrow}\text{$\exists
    S \in \mathcal{N'}(v) \text{ s.t. } \tmop{Th} (\mathbbm{M}') \cup
    \delta_v S \vdash \phi$}\\
    & {\Longleftrightarrow}\text{$\exists S \in \mathcal{N}'(v) \text{ s.t. }
    \tmop{Th} (\mathbbm{M}') \vdash \bigwedge \delta_v S \rightarrow
    \phi$}\\
    & {\Longleftrightarrow}\text{$\exists S \in \mathcal{N}'(v) \text{ s.t. }
    \bigwedge \delta_v S \rightarrow \phi \in \tmop{Th} (\mathbbm{M}')$}\\
    % & {\Longleftrightarrow}\text{$\exists S \in \mathcal{N}'(v) \text{ s.t. }
    % \left\llbracket \bigwedge \delta_v S \right\rrbracket \subseteq
    % \left\llbracket \phi \right\rrbracket$}\\
    % & {\Longleftrightarrow}\text{$\exists S \in \mathcal{N}'(v) \text{ s.t.
    % $\forall u \in W$ if } \mathbbm{M}, u \vDash \bigwedge \delta_v S  \phi$}\\
    & {\Longleftrightarrow}\text{$\exists S \in \mathcal{N}'(v) \text{ s.t.
    $\forall u \in W$ if } \mathbbm{M}, u \vDash \bigwedge \delta_v S \text{
    then } \mathbbm{M}, u \vDash \phi$}\\
    & {\Longleftrightarrow}\text{$\exists S \in \mathcal{N}'(v) \text{ s.t. }
    \mathbbm{M}, u \vDash \phi \text{ for all } u \in S$}\\
    & {\Longleftrightarrow}{\mathbbm{M}},v{\vDash}\Box{\phi}
  \end{align*}
  This completes the proof.
\end{proof}
%%% Local Variables: 
%%% mode: latex
%%% TeX-master: "paper"
%%% End: 


It is not too hard to identify the fragment of this logic that governs
knowledge alone.  The fragment given in Table \ref{logic4} has special
significance to us, since it governs the notion of knowledge presented in
{\S}\ref{quantifying}.

\begin{table}[h]
  \begin{tabular}{l}
    $\vdash \phi \rightarrow \psi \rightarrow \phi$\\
    $\vdash (\phi \rightarrow \psi \rightarrow \chi) \rightarrow (\phi
    \rightarrow \psi) \rightarrow \phi \rightarrow \chi$\\
    $\vdash ((\phi \rightarrow \bot) \rightarrow (\psi \rightarrow \bot))
    \rightarrow \psi \rightarrow \phi$\\
    $\vdash K \phi \rightarrow \phi$\\
    \\
    \begin{tabular}{llll}
      $\frac{\vdash \phi \rightarrow \psi \hspace{4em} \vdash \phi}{\vdash
      \psi}$ &  &  & $\frac{\vdash \phi \rightarrow \psi}{\vdash K \phi
      \rightarrow K \psi}$
    \end{tabular}
  \end{tabular}
  \caption{\label{logic4}A knowledge-only neighborhood logic for
  \tmtextbf{NCSQ} and {\tmstrong{NSND}}}
\end{table}

\begin{proposition}
  The knowledge-only logic presented governs the $\Box$-free fragment of the
  logic in Table \ref{logic3}
\end{proposition}

This concludes our analysis of knowledge bases using neighborhood
semantics.
%%% Local Variables: 
%%% mode: latex
%%% TeX-master: "paper"
%%% End: 


\subsection{\label{quantifying}Using Modalities to Quantify over Knowledge
Bases}
\label{modal-modal}
 

\section{Conclusions \& Further Research}
\label{conclusions}
(FIXME:  Needs to be extended a lot!)

The logic presented in Table \ref{logic1} is a conservative extension of the
basic modal logic $K$, which means that its decision problem is
\tmtextsf{PSPACE} hard (a lower bound).  Our finitary completeness proof
establishes that its complexity is in \tmtextsf{EXP2-TIME} (an upper bound).

Worlds in epistemic logic might correspond to fantastic, pretend scenarios;
however, it may be desirable for some philosophers to model knowledge about
imaginary lands.  After all, many children ``know'' a lot about Tolkien's
{\tmem{middle earth}} or Rowling's {\tmem{Hogwarts}} (the school Harry Potter
attends).

\bibliographystyle{plain}
\bibliography{zotero}

\end{document}
