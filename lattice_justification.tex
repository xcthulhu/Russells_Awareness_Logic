Implicit in our previous presentation of SJL is the following
principle: 
\todo[noline]{THIS SECTION IS  PROBABLY WRONG}
 $\oplus$ behaves like a union
operation for knowledge bases, and dually like an intersection operation on
accessibility relations.  It is natural to think of $\oplus$ as a join/meet
operator over a {\tmem{semilattice}}.  In this section we make this
explicit, by increasing the expressive power of simple JL to express the order
theory inherent in knowledge bases.  We call the novel logic presented in
this section \tmtextit{Lattice Justification Logic} (LJL){\footnote{We note
that ``Semilattice Justification Logic'' may be a more appropriate name,
however we have decided to shorten semilattice to lattice in this article
for purposes of concision.}}.  Lattice JL may be considered a novel logic for
studying the dynamics of \tmtextit{theory change}, where the logic employed by
the changing theories could be LJL itself.



\begin{definition}
  $\mathcal{L}_{\tmop{LJL}} (\Phi, \Pi)$ extends $\mathcal{L}_{\tmop{SJL}}
  (\Phi, \Pi)$ to:
  \[ \phi \  : : = \  p \in \Phi \  |
     \  \circlearrowleft_X \  | \  \bot
     \  | \  \phi \rightarrow \psi \  |
     \  \Box_X \phi \  | \  X : \phi
     \  | \  X \sqsubseteq Y \]
  where $X, Y \in \tau (\Pi)$
\end{definition}

\begin{definition}
  \label{justmodels}A {\tmstrong{lattice justification model}} $\mathbbm{M}=
  \langle W, V, R, \mathcal{A}, \preccurlyeq \rangle$ is a simple
  justification model with a new relation $\preccurlyeq : W \rightarrow
  2^{\tau (\Pi) \times \tau (\Pi)}$ between terms, indexed by worlds.  The
  semantics of $\sqsubseteq$ correspond to $\preccurlyeq$ as follows:
  \[ \mathbbm{M}, w \vDash X \sqsubseteq Y \Longleftrightarrow X
     \preccurlyeq_w Y \]
\end{definition}

\begin{definition}
  A lattices justification model is said to make true \tmtextbf{LATTICE} if
  and only if:
  \[ \mathbbm{M}, w \vDash X \sqsubseteq Y \Longleftrightarrow \mathcal{A}_w
     (X) \subseteq \mathcal{A}_w (Y) \]
\end{definition}

Table \ref{logic6} lists the axioms for the logic of lattice justification
models with the properties we have been investigating:

\begin{table}[h]
  \begin{tabular}{ll}
    $\vdash \phi \rightarrow \psi \rightarrow \phi$ & \\
    $\vdash (\phi \rightarrow \psi \rightarrow \chi) \rightarrow (\phi
    \rightarrow \psi) \rightarrow \phi \rightarrow \chi$ & \\
    $\vdash ((\phi \rightarrow \bot) \rightarrow (\psi \rightarrow \bot))
    \rightarrow \psi \rightarrow \phi$ & \\
    $\vdash \Box_X (\phi \rightarrow \psi) \rightarrow \Box_X \phi \rightarrow
    \Box_X \psi$ & \\
    $\vdash (X : \phi) \rightarrow \Box_X \phi$ & \\
    $\vdash \circlearrowleft_X \rightarrow \Box_X \phi \rightarrow \phi$ & \\
    $\vdash (X : \phi) \rightarrow (Y : \phi) \rightarrow X \oplus Y : \phi$ &
    \\
    $\vdash X \sqsubseteq Y \rightarrow (X : \phi) \rightarrow Y : \phi$ & \\
    $\vdash \circlearrowleft_X \rightarrow \circlearrowleft_Y \rightarrow
    \circlearrowleft_{X \oplus Y}$ & \\
    $\vdash X \sqsubseteq Y \rightarrow \circlearrowleft_Y \rightarrow
    \circlearrowleft_X$ & \\
    $\vdash X \sqsubseteq Y \rightarrow \Box_X \phi \rightarrow \Box_Y \phi$ &
    \\
    $\vdash X \sqsubseteq X$ & \\
    $\vdash X \sqsubseteq Y \rightarrow Y \sqsubseteq Z \rightarrow X
    \sqsubseteq Z$ & \\
    $\vdash X \sqsubseteq Y \rightarrow X \sqsubseteq Z \oplus Y$ & $\vdash X
    \sqsubseteq Y \rightarrow X \sqsubseteq Y \oplus Z$\\
    $\vdash X \sqsubseteq Z \rightarrow Y \sqsubseteq Z \rightarrow X \oplus Y
    \sqsubseteq Z$ & \\
    & \\
    \begin{tabular}{lll}
      $\frac{\vdash \phi \rightarrow \psi \hspace{4em} \vdash \phi}{\vdash
      \psi}$ & {\hspace{6em}} & $\frac{\vdash \phi}{\vdash \Box_X \phi}$
    \end{tabular} & 
  \end{tabular}
  \caption{\label{logic6}Lattice Justification Logic}
\end{table}

\begin{theorem}
  \label{completeness6}Assuming an infinite store of proposition letters
  $\Phi$, LJL is sound and weakly complete for lattice justification models
  making true \tmtextbf{JCSQ}, \tmtextbf{JSND}, \tmtextbf{CHOICE} and
  \tmtextbf{LATTICE}
\end{theorem}

\begin{proof}
  Assume $\nvdash \psi$.  Let $\mathbbm{M} \assign \langle W, V, R,
  \mathcal{A}, \preccurlyeq \rangle$ be the same as the finitary canonical
  model we initially constructed in Theorem \ref{completeness5}, only
  $\preccurlyeq$ is defined as:
  \[ X \preccurlyeq_w Y \Longleftrightarrow w \vdash X \sqsubseteq Y \]
  We may deduce that there is a world $w \in W$ such that $\mathbbm{M}, w
  \nvDash \psi$.  This canonical model makes true properties (\ref{fin})
  through (\ref{union}) we listed in Theorem \ref{completeness5}, as well as a
  certain new properties:
  
  \begin{descriptioncompact}
    \item[6$\preccurlyeq$] For all $X, Y \in \Xi$, if $X \preccurlyeq_v Y$
    then $R_Y [v] \subseteq R_X [v]$
    
    \item[7] For all $X, Y \in \Xi$, if $X \preccurlyeq_v Y$ then
    $\mathcal{A}_v (X) \subseteq \mathcal{A}_v (Y)$
    
    \item[8] If $X \nin \Xi$, then $R_X = W \times W$ and $\mathcal{A}_v (X) =
    \varnothing$ for all $v \in W$
    
    \item[9] $\langle \preccurlyeq_v, \oplus \rangle$ is a join-semilattice
    over $\tau (\Pi)$ for all $v \in W$
  \end{descriptioncompact}
  
  As in previous constructions, completeness is achieved by through a series
  of model refinements.  Note that property 8 has been implicitly true in all
  of our previous constructions, although we have not been interested in
  enforcing it to be inherited by model refinements previously.
  
  
  
  We first construct a model $\mathbbm{M}'$ which is bisimular to
  $\mathbbm{M}$, where (\ref{refl}) is strengthened to a biconditional.  This
  is done exactly as we proceeded in previous constructions. We note that the
  notion of ``bisimulation'' here includes that two bisimular worlds $v$ and
  $u$ must have isomorphic semilattices $\preccurlyeq_v$ and
  $\preccurlyeq_w$.  This is achieved by enforcing that $\preccurlyeq_{v_l}
  \assign \preccurlyeq_{v_r} \assign \preccurlyeq_v$. As before, we have that
  $\mathbbm{M}', w \nvDash \psi$ for some world $w \in W'$.
  
  
  
  We next strengthen (6$\preccurlyeq$) to a biconditional.  Define
  $\mathbbm{M}'' \assign \langle W'', V'', R'', \mathcal{A}'' \rangle$ such
  that:
  \begin{itemizedot}
    \item $W'' \assign \uplus_{\Xi} W'$
    
    \item $V'' (p) \assign \{v_X \  | \  X \in \Xi
    \  \& \  v \in V' (p)\}$
    
    \item $R''_X \assign \{(v_Y, u_Z) \  | \  Y, Z \in \Xi
    \  \& \  v R'_Z u \  \& \  X
    \preccurlyeq_v Z\}$
    
    \item $\mathcal{A}'' (v_Y, X) \assign \mathcal{A}' (v, X)$
  \end{itemizedot}
  $\mathbbm{M}''$ is bisimular to $\mathbbm{M}'$ by the same mechanism as the
  previous construction.  The important feature of this structure is to
  observe that the converse of (6$\preccurlyeq$) holds:
  \begin{eqnarray*}
    R_Y [v] \subseteq R_X [v] & \Longleftrightarrow & \forall Z \in \Xi
    \forall u_Z \in W'' (v_Q R'_Z u_Z \Longrightarrow v_Q R_X' u_Z)\\
    & \Longleftrightarrow & \forall Z \in \Xi \forall u_Z \in W'' (v R'_Z
    \Longrightarrow v_Q R_X u_Z)
  \end{eqnarray*}
  
  
  Our first transformation strengthens (\ref{union}), (6$\preccurlyeq$a) and
  (6$\preccurlyeq$b) so that they hold for all terms in $\tau (\Pi)$, rather
  than being restricted to $\Xi$.  To this end define $\mathbbm{M}' \assign
  \langle W', V', R', \mathcal{A}', \preccurlyeq' \rangle$ where $W'$, $R'$
  and $\preccurlyeq'$ are the same as in $\mathbbm{M}$, but $R'$ and
  $\mathcal{A}'$ have the following modifications:
  \begin{eqnarray}
    R'_Y [v] & \assign & \bigcap_{X \preccurlyeq_v Y} R_X [v] \nonumber\\
    \mathcal{A}'_v (Y) & \assign & \bigcup_{X \preccurlyeq_v Y} \mathcal{A}'_v
    (X) \nonumber
  \end{eqnarray}
  
\end{proof}
%%% Local Variables: 
%%% mode: latex
%%% TeX-master: "paper"
%%% End: 
