Implicit in our previous presentation of SJL is the following
principle: 
\todo[noline]{THIS SECTION IS  PROBABLY WRONG}
 $\oplus$ behaves like a union
operation for knowledge bases, and dually like an intersection operation on
accessibility relations.  It is natural to think of $\oplus$ as a join/meet
operator over a {\tmem{semilattice}}.  In this section we make this
explicit, by increasing the expressive power of simple JL to express the order
theory inherent in knowledge bases.  We call the novel logic presented in
this section \tmtextit{Lattice Justification Logic} (LJL){\footnote{We note
that ``Semilattice Justification Logic'' may be a more appropriate name,
however we have decided to shorten semilattice to lattice in this article
for purposes of concision.}}.  Lattice JL may be considered a novel logic for
studying the dynamics of \tmtextit{theory change}, where the logic employed by
the changing theories could be LJL itself.



\begin{definition}
  $\mathcal{L}_{\tmop{LJL}} (\Phi, \Pi)$ extends $\mathcal{L}_{\tmop{SJL}}
  (\Phi, \Pi)$ to:
  \[ \phi \  : : = \  p \in \Phi \  |
     \  \circlearrowleft_X \  | \  \bot
     \  | \  \phi \rightarrow \psi \  |
     \  \Box_X \phi \  | \  X : \phi
     \  | \  X \sqsubseteq Y \]
  where $X, Y \in \tau (\Pi)$
\end{definition}

\begin{definition}
  \label{latmodels}A {\tmstrong{lattice justification model}} $\mathbbm{M}=
  \langle W, V, R, \mathcal{A}, \preccurlyeq \rangle$ is a simple
  justification model with a new relation $\preccurlyeq : W \rightarrow
  2^{\tau (\Pi) \times \tau (\Pi)}$ between terms, indexed by worlds.  The
  semantics of $\sqsubseteq$ correspond to $\preccurlyeq$ as follows:
  \[ \mathbbm{M}, w \vDash X \sqsubseteq Y \Longleftrightarrow X
     \preccurlyeq_w Y \]
\end{definition}

\begin{definition}
  A lattices justification model is said to make true \tmtextbf{LATTICE} if
  and only if:
  \[ \mathbbm{M}, w \vDash X \sqsubseteq Y \Longleftrightarrow \mathcal{A}_w
     (X) \subseteq \mathcal{A}_w (Y) \]
\end{definition}

Table \ref{logic6} lists the axioms for the logic of lattice justification
models with the properties we have been investigating:

\begin{table}[h]
  \begin{tabular}{ll}
    $\vdash \phi \rightarrow \psi \rightarrow \phi$ & \\
    $\vdash (\phi \rightarrow \psi \rightarrow \chi) \rightarrow (\phi
    \rightarrow \psi) \rightarrow \phi \rightarrow \chi$ & \\
    $\vdash ((\phi \rightarrow \bot) \rightarrow (\psi \rightarrow \bot))
    \rightarrow \psi \rightarrow \phi$ & \\
    $\vdash \Box_X (\phi \rightarrow \psi) \rightarrow \Box_X \phi \rightarrow
    \Box_X \psi$ & \\
    $\vdash (X : \phi) \rightarrow \Box_X \phi$ & \\
    $\vdash \circlearrowleft_X \rightarrow \Box_X \phi \rightarrow \phi$ & \\
    $\vdash (X : \phi) \rightarrow (Y : \phi) \rightarrow X \oplus Y : \phi$ &
    \\
    $\vdash X \sqsubseteq Y \rightarrow (X : \phi) \rightarrow Y : \phi$ & \\
    $\vdash \circlearrowleft_X \rightarrow \circlearrowleft_Y \rightarrow
    \circlearrowleft_{X \oplus Y}$ & \\
    $\vdash X \sqsubseteq Y \rightarrow \circlearrowleft_Y \rightarrow
    \circlearrowleft_X$ & \\
    $\vdash X \sqsubseteq Y \rightarrow \Box_X \phi \rightarrow \Box_Y \phi$ &
    \\
    $\vdash X \sqsubseteq X$ & \\
    $\vdash X \sqsubseteq Y \rightarrow Y \sqsubseteq Z \rightarrow X
    \sqsubseteq Z$ & \\
    $\vdash X \sqsubseteq Y \rightarrow X \sqsubseteq Z \oplus Y$ & $\vdash X
    \sqsubseteq Y \rightarrow X \sqsubseteq Y \oplus Z$\\
    $\vdash X \sqsubseteq Z \rightarrow Y \sqsubseteq Z \rightarrow X \oplus Y
    \sqsubseteq Z$ & \\
    & \\
    \begin{tabular}{lll}
      $\frac{\vdash \phi \rightarrow \psi \hspace{4em} \vdash \phi}{\vdash
      \psi}$ & {\hspace{6em}} & $\frac{\vdash \phi}{\vdash \Box_X \phi}$
    \end{tabular} & 
  \end{tabular}
  \caption{\label{logic6}Lattice Justification Logic}
\end{table}

\begin{theorem}
  \label{completeness6}Assuming an infinite store of proposition letters
  $\Phi$, LJL is sound and weakly complete for lattice justification models
  making true \tmtextbf{JCSQ}, \tmtextbf{JSND}, \tmtextbf{CHOICE} and
  \tmtextbf{LATTICE}
\end{theorem}

\input{completeness4}
%%% Local Variables: 
%%% mode: latex
%%% TeX-master: "paper"
%%% End: 
