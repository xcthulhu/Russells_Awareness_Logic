\subsection{Russell's Theory of Knowledge}\label{Russell}

In this section we review Russell's epistemology, propose a reading of
Russell in
terms of modern epistemic logic, and review Hintikka's dismissal of a similar reading.

Russell's theory of knowledge is a form of \emph{strong
  foundationalism}.  We specifically focus on the view presented in
Chapter XIII of \emph{The Problems of Philosophy}, entitled
``Knowledge, Error, and Probable Opinion'' \cite[pg. 52]{russell_problems_1936}:
\begin{quote}
We may now take a survey of the sources of our knowledge, as they have
appeared in the course of our analysis. We have first to distinguish knowledge
of things and knowledge of truths. In each there are two kinds, one immediate
and one derivative. Our immediate knowledge of things, which we called
\emph{acquaintance}, consists of two sorts, according as the things known are
particulars or universals. Among particulars, we have acquaintance with
sense-data and (probably) with ourselves.%  Among universals, there seems to
% be no principle by which we can decide which can be known by acquaintance,
% but it is clear that among those that can be so known are sensible qualities,
% relations of space and time, similarity, and certain abstract
% logical universals.
\ldots Our derivative knowledge of things, which we call knowledge by \emph{description},
always involves both acquaintance with something and knowledge of truths.
Our immediate knowledge of truths may be called \emph{intuitive} knowledge, and the
truths so known may be called \emph{self-evident} truths. Among such truths are
included those which merely state what is given in sense, and also certain
abstract logical and arithmetical principles, and (though with less certainty)
some ethical propositions. Our derivative knowledge of truths consists of
everything that we can deduce from self-evident truths by the use of self-
evident principles of deduction.
\end{quote}

  In this essay, we will focus on modeling Russell's ``knowledge of things'',
  although implicit knowledge in particular will play an important
  role.

  Russell epistemology equates naturally to the modern concepts in
  epistemic logic. 
  \emph{Acquaintance} can be seen as explicit knowledge, while
  \emph{knowledge by description} can be seen as implicit knowledge.
  In particular, knowledge by description is the logical closure of
  acquaintance and \emph{intuitive} knowledge.

\subsection{Jaako Hintika}\label{Russell}

Jaako Hintikka briefly entertains a reading of
epistemic logic similar to Russell's epistemology 
in \emph{Knowledge and Belief}, where he provides the
first formulation of epistemic logic
\cite{hintikka_knowledge_1969}[pgs. 38--39].  
There he proposes that ``$K_a
p$" might be read as ``It follows from what $a$ knows that $p$''.

Hintikka rejects the Russellian reading of epistemic logic after
proposing it. His stated grounds are that
such a reading appeals to some given standard of logical consequence.  
He asserts that this is ``diametrically opposed'' to his research program in epistemic logic,
which is intended to develop a system of logical consequence using his
invented semantics as a foundation.

Hintikka's objection is reasonable.  It is counter intuitive to assume
a system of logical consequence is present at the semantic level in
epistemic logic.
However, it is not \emph{impossible}.  In the next section we sketch
an approach to epistemic logic that incorporates logical consequence
into awareness logic.  The intuition behind this move follows
Russell's formulation of \emph{acquaintance} and \emph{knowledge by
  description}.  After providing formal systems that express our
proposed reading of epistemic modality, we return Hintikka's objection
in \S\ref{hintikka-obj} in our concluding remarks.

% % The modern tradition in epistemic logic is to assume knowledge modalities
% % conform to the $S 5$ axiom schema.  As noted in
% % {\cite{halpern_set-theoretic_1999,rubinstein_modeling_1998}}, the semantics of
% % $S 5$ knowledge correspond exactly to partitioning a collection of situations
% % into \tmtextit{information sets}, which is the traditional approach in game
% % theory and decision theory.  While it is not commonly acknowledged in
% % epistemic logic, economists and philosophers accept that traditional decision
% % theory is externalist and behaviorist in nature
% % \footnote{An early essay by
% % Amartya Sen on the philosophical foundations of traditional decision theory
% % makes the behaviorist reading of decision theory clear
% % \cite{sen_behaviour_1973}.  Kaushik Basu also discusses the behaviorist
% % nature of decision theory {\cite[pgs. 53--54]{basu_revealed_1980}}. 
% % Finally, Donald Davidson appeals to decision theory to motivate an externalist
% % epistemology in \cite{davidson_could_1995}.}.

% \subsection{Parallels in Epistemic Logic}

% Russell's theory of \emph{deductive} and \emph{intuitive}
% knowledge is paralleled by modern research into implicit and explicit
% belief. Levesque introduced these terms in his 1984 paper \emph{A Logic of Implicit and Explicit Belief}
% \cite{levesque_logic_1984}.

% \todo{Find quote by Levesque which exposes his reinvention of Russell}

% In 1991 Johan van Benthem proposed a research program to find logics
% for explicit knowledge {\cite{benthem_reflections_1991}}.  Subsequently,
% Sergei Artemov and Elena Nogina have proposed a logic of explicit justification,
% which has come to be known as \tmtextit{Justification Logic} (JL)
% {\cite{artemov_introducing_2005}}, based on Artemov's \tmtextit{Logic of
% Proofs} {\cite{artemov_logic_1994}}.  While the original semantics of
% JL were based on interpretability into Peano Arithmetic, Melvin Fitting proposed Kripke
% semantics for JL in {\cite{fitting_logic_2005}}.  One may understand
% JL as extend awareness logic with operations for combining and
% composing awareness sets, although we note that this was \emph{not}
% the intended interpretation of the original logic of proofs.  In
% \cite{velazquez-quesada_inference_2009}, Fernando Vel\'azquez
% develops a logic similar to JL, using update operations make changes
% to models corresponding to changes in knowledge, based on ordinary
% awareness logic.  Unfortunately,
% Vel\'azquez' logic has a severe grammar restriction on the formula
% permissible for awareness, making it unattractive. 
% In {\cite{benthem_inference_2009}},
% Vel\'azquez-Quesada and van Benthem simplified Vel\'azquez-Quesada's
% framework so that the grammar restriction is relaxed.  Unlike the other efforts, this
% logic is more abstract and does not attempt to semantically model
% deduction; there is in fact no relationship between explicit and
% implicit in this logic at all.

% The purpose of this essay is to propose novel awareness logics which
% connect the research program surrounding explicit knowledge to
% traditional foundationalism.

\subsection{Modeling Russell}\label{mod-russell}

In this section we introduce models of Russell's epistemology.  We
will make use of \emph{awareness models} $\mathbb{M}$, using semantics originally developed in \cite{fagin_belief_1988}.

As we discussed in \S\ref{Russell}, \emph{Acquaintance with $\phi$} is
naturally interpreted as denoted as \emph{explicit awareness of
  $\phi$}.  We will denote this as $A:\phi$, with the usual
semantics of $\mathbb{M},w \models A:\phi \iff \phi \in \mathcal{A}(w)$
where $\mathcal{A}(w)$ is an awareness set.

Modeling descriptive knowledge is not as straightforward.  We will model
\emph{knowledge by description that $\phi$} as \emph{implicit knowledge
  that $\phi$}, denoted $\Box\phi$.  However, it is necessary to introduce
further constraints.
Minimally, we want to enforce that $\mathbb{M} \models A : \phi \to \Box \phi$ and
$\mathbb{M} \models \Box \phi \to \phi$, reflecting that knowledge by
description extends acquaintance and that knowledge is necessarily
\emph{true}.

All that is left is to establish a connection between knowledge by
description and logical
consequence.  Let $\vdash$ be any Hilbert-style calculus,
that is a sequent that takes a set of premises $\Gamma$ on the left and yields a
single conclusion $\phi$ on the right. Define the following terminology:
\begin{definition}\ 
\begin{itemizedot}
\item $\vdash$ is \emph{relatively sound} for $\mathbb{M}$ if and only if
  when $\Gamma \vdash \phi$, then if $\mathbb{M},w\models \gamma$, for
  all $\gamma \in \Gamma$, then
  $\mathbb{M},w \models \phi$
\item $\vdash$ is \emph{reflexive} if and only if whenever $\phi \in
  \Gamma$ then $\Gamma \vdash \phi$
\item $\vdash$ makes true \emph{modus ponens} if and only if whenever
  $\Gamma \vdash \phi \to \psi$ and $\Gamma \vdash \phi$, then $\Gamma
  \vdash \psi$
\item $\vdash$ is said to be \emph{suitable} for $\mathbb{M}$ just in
  case that it is reflexive, makes true modus ponens, and is
  relatively sound for $\mathbb{M}$
\end{itemizedot}
\end{definition}

A model $\mathbb{M}$ is thought to model Russell's epistemology in the
case that there is some suitable logic $\vdash$ such that for all
worlds $w$:

\[ \mathbb{M}, w \models \Box \phi \textup{ if and only if } Th(\mathbb{M}) \cup
\mathcal{A}(w) \vdash \phi \]

Here $Th(\mathbb{M})$ is the \emph{theory of $\mathbb{M}$}, and
denotes the set $\{ \phi\ |\ \mathbb{M},w \models \phi$ for all worlds
$w \}$.  We take it to represent Russell's \emph{intuitive
  knowledge} \footnote{A caveat: we are being very liberal with our interpretation
  of Russell's
  epistemology here.  The set $Th(\mathbb{M})$ contains, among other
  things, all tautologies for any logic $\mathbb{M}$ obeys.  Russell
  intended for intuitive knowledge to correspond to empirical
  knowledge and a small number of a priori truths, which serve as
  an axiom system for derivations.  We defend choice to call $Th(\mathbb{M})$ intuitive
  knowledge, since (1) it may be considered as a sort of a priori knowledge, (2) it fits the definition of knowledge by description well,
  and (3) it provides a framework for our subsequent formal
  results.}. 
  
The property we suggest is \emph{not} definitional of modality.  Rather, it is a
feature that an awareness model might or might not make.  Restricting
oneself to models with the above property is
similar to restricting models to the class which has reflexive
relations.

Models where implicit knowledge is
equated to the presence
of a derivation are faithful to Russell's epistemology.  As we stated
before, Russell's knowledge by description is the logical closure of
acquaintance and intuitive knowledge, which is exactly our proposal.  Our response
to Hintikka's criticism of this approach is that while epistemic logic may be
based on models, it could also be based on particular models
which have logic associated with them in a special way.  Assuming that
knowledge in epistemic logic corresponds to \emph{some} notion of logical consequence
does not require us to have already formulated the notion of logical
consequence we will ultimately
develop.  While logic about logic is circular, it is not
\emph{viciously} so, and moreover poses no inescapable paradoxes.

The rest of this essay is devoted to developing awareness logics which have
the mechanics we have sketched.

% In this essay we repurpose various externalist logics to take on an
% internalist reading.  The concept of a \tmtextit{knowledge base}, from which
% beliefs may be implicitly deduced, will play a crucial role in our discussion.
%  We propose this as an avenue for representing foundationalist perspectives
% on epistemology in epistemic logic.  Our philosophical motivation is taken
% from two sources.  The first is Vincent Hendricks in
% {\cite{hendricks_mainstream_2006}}, where he characterizes the principal of
% {\tmem{logical omniscience}} for implicit knowledge in epistemic
% logic{\footnote{We have modified Hendricks' notation here slightly to match
% our own.}}:

% \begin{quote}
%   {\tmem{Whenever an agent $X$ knows all of the formulae in $\mathcal{A}$,
%   and $\phi$ follows logically from $\mathcal{A}$, then $\Xi$ [implicitly]
%   knows $\phi$.}}
% \end{quote}

% We will design our semantics such that ``$\Box \phi$'' may be equated with
% ``$\phi$ follows logically from a knowledge base $\mathcal{A}$,'' which is
% sometimes written as $\phi \in \tmop{Cn} ( \mathcal{A})$ in the artificial
% intelligence literature.  Our second inspiration comes from what   The
% rest of Kornblith's paper is devoted to attacking this view and proposing a
% form of naturalized epistemology; we will not address this debate here,
% however.

% We adopt a arguments-on-paper-thesis perspective on epistemic logic in this
% paper.  We consider a \tmtextit{good argument} to be a logical derivation
% from propositions present in a knowledge base in this setting.  
% Special attention will be given to \tmtextit{sound derivations}, which will be
% thought of as a form of knowledge.  In \emph{Awareness Logic}, we will interpret
% awareness of a formula as membership in a knowledge base.  We prove
% completeness for basic awareness logic and a hybrid logic extension.  Logics
% of multiple knowledge bases are also presented: a simplified form of
% Justification Logic, a logic with neighborhood semantics, an a logic with
% modalities for quantifying over knowledge bases.  We conclude with an
% application to naturalized epistemology found in the psychological
% literature.
%%% Local Variables: 
%%% mode: latex
%%% TeX-master: "paper"
%%% End: 
