In this section we provide our final analysis of Hintikka's objection
to our proposed reading of epistemic modality, connections of our
logic to other approaches, and we propose future avenues of research.

\subsection{Hintikka's Objection}\label{hintikka-conc}

Contrary to Hintikka's opposition (as we presented in \S\ref{Hintikka}), we have demonstrated various logics
where $\Box \phi$ has a Russellian interpretation: 
``$\phi$ may be derived logically from basic propositions that the
agent is aware of that $\phi$''.  This is accomplished
by first providing basic semantics, and then restricting ourselves to
classes of models which express the property we have labeled as 
$\textup{\textbf{CSQ}}_\vdash$, where $\vdash$ is some suitable
logic.  Our specification of what suitable means is minimal, although it
is enough to determine a complete notion of logical consequence in
every case.

However, we propose that a more radical reading of $\Box \phi$ is adequate: ``$\phi$ may be derived logically from basic propositions that the
agent is aware of that $\phi$, \emph{using epistemic logic itself}''.
In other words, one may interpret epistemic agents as using the
same logic we are ultimately developing.
We take the relevant notion of epistemic logic to be application dependent;
it may be awareness logic, hybrid logic, etc.
  This is because of the
following observation, which we assert without proof: \emph{for every
  logic $\vdash_L$ we propose, $\vdash_L$ is sound and weakly complete
  for an appropriate class of models $\mathbb{M}$ making true
  $\textup{\textbf{CSQ}}_{\vdash_L}$}.
To see why this is true, it is routine to verify soundness, and then
check that each of our completeness proofs deliver suitable counter-models.

Our more radical reading directly contradicts Hintikka: we are proposing that semantics
may appeal to their own notion of logical consequence.  The method of
developing systems that accomplish this is analogous to other
techniques in mathematics.  In
order to derive the static solution of a diffusion equation, one
 minimally assumes the solution to be \emph{separable}.  
From this assumption, the form of the solution may be derived, and one may
check that the derived solution fulfills all of the initial assumptions.  Formulating logics that have
semantics which incorporate themselves may be regarded as a similar
exercise to solving differential equations, with our assumption of
\emph{suitability} playing the similar r\^{o}le as \emph{separability}.

\subsection{Related Work}

All of the logics in this paper are variations of
systems that have been developed elsewhere for different purposes.
We defer to the cited literature for readers interested in the
comparing our approach to others.

There is a connection we did not anticipate between the neighborhood 
logic in \S\ref{neighborhood_semantics} and a logic for statistical inference
presented by Kyburg et al. \cite{kyburg_logic_2002}.  We assert
without proof that the awareness free fragment of our logic is exactly
the same as the logic developed in Kyburg et al., which is precisely Chellas'
neighborhood logic $EMN$ \cite[chapter 9]{chellas_modal_1980}.

An important similarity between Kyburg et al.'s approach and our own is
our respective philosophical motivations.  In both systems, the guiding intuition lies in pragmatic
considerations in scientific epistemology; specifically it is meant
for reasoning about the ``Lottery Paradox''
\cite[pg. 197]{kyburg_probability_1961}.  To sketch this puzzle, let
$\phi_X$ denote ``$X$ will win the lottery'', for some lottery with
1000 tickets.  For any given person $X$, call them ``$1$'', we have
that $Pr(\neg \phi_1)
\approx 1$, so one may reasonably conclude that $\neg\phi_1$.  One
could similarly conclude $\neg \phi_2$, and it seems $Pr(\phi_1 \vee
\phi_2) \approx 0$, so one could conclude $\neg\phi_1 \wedge \neg \phi_2$.
However, can cannot rationally conclude that $\bigvee_{1 \leq X \leq 1000}
\neg \phi_X$, since \emph{somebody} must win the lottery.  

One method of resolving this paradox is to employ a system of statistical
inference known as the \emph{Evidential Probability
  Calculus} \cite{kyburg_uncertain_2001}.  
This may be seen as a variation on Maynard Keynes' interval probability 
calculus \cite{keynes_treatise_1921}.  Using this system, Kyburg et
al. write $\Gamma
\vdash_\epsilon \phi$ to mean that a body of statistical evidence
$\Gamma$ determines that $Pr(\phi) \in [p,q]$ where $p \geq
1 - \epsilon$ (given certain other constraints).  They then propose semantics which equate doxastic
modality $\Box_\epsilon \phi$ with an inference $\Gamma \vdash_\epsilon\phi$ for some
body of statistical evidence $\Gamma$.  The similarity between this
approach and the Russellian epistemic logics we propose is clear.

Furthermore, we regard it as significant that two disparate efforts,
both motivated by scientific epistemology,
have converged Chellas' neighborhood logic $EMN$.

\subsection{Conclusion}

Philosophers and logicians have been informally reading $\Box \phi$ as
``it follows from what the agent knows that $\phi$'' for years,
contrary to Hintikka's rejection of this interpretation.  It is natural that formal systems
may be developed with correspond to this intuition.  What is somewhat
unexpected is that Russell's epistemology effectively fits this
perspective, and anticipates awareness logic.  We have demonstrated that a Russellian
reading is robustly applicable to modal, submodal, and hybrid awareness
logics.  In future work, we want to investigate the relationship
between our approach and Kyburg et al.'s, given that the two
employ similar semantics and share a common logic. 

%%% Local Variables: 
%%% mode: latex
%%% TeX-master: "paper"
%%% End: 
