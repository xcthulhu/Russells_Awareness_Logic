Justification Logic (JL) was originally developed by Artemov as the
\tmtextit{Logic of Proofs} (LP) {\cite{artemov_logic_1994}}.  Its original
purpose was to provide a framework for reasoning about explicit provability in
Peano Arithmetic.  The first introduction of Justification Logic can be found
in {\cite{artemov_introducing_2005}}, where Artemov and Nogina propose LP as a
logic for reasoning about evidence.  In {\cite{fitting_logic_2005}}, Melvin
Fitting provided JL with Kripke model based semantics.

The principle innovation of LP/JL is to extend awareness logic, such that
awareness operations are now \tmtextit{proof terms}.  Informally, we say that
``a proof term $X$ witnesses proposition $\phi$'', denoted ``$X : \phi$'',
whenever $X$ represents a proof of $\phi$.  Proofs may have
{\tmstrong{multiple-conclusions}} in this system, so it is possible that $X :
\phi$ and $X : \psi$ can be true where $\phi \neq \psi$.  In Fitting's Kripke
semantics, $X : \phi$ means that $\phi$ is in awareness set corresponding to
$X$.  Proof terms are thought be operating in a \tmtextit{multi-conclusion}
proof system, so the same proof term may witness many different propositions.



The logics LP/JL include operators over proofs terms, so that new proof terms
may be assembled.  One operation of particular interest to us is
{\tmem{choice}}, denoted $\oplus$.  The expression ``$X \oplus Y : \phi$''
denotes that either $X$ or $Y$ are proofs witnessing $\phi$.  There are other
operations which correspond to \tmtextit{modus ponens} and
\tmtextit{proof-theoretic reflection}, however we do not consider these
operations here.



In this section we consider a simplified form of JL suitable for reasoning
over multiple knowledge bases, which we call \tmtextit{simple justification
logic} (SJL). However, instead of thinking of terms as representing proofs, we
use them to represent knowledge bases. This is in the spirit of JL as a logic
of evidence: we consider each knowledge base to be a corpus of evidence. \
Each term ``names'' a different knowledge base at a particular world.  Terms
may not refer to the same knowledge base at different worlds, just as in how
in awareness logic the agent need not be aware of the same formulae at
different worlds.  Finally, we will make use of JL's {\tmem{choice}} operator
as a mechanism for forming the union of knowledge bases, creating new ones.

\begin{definition}
  Let $\Pi$ be a set of primitive terms.  Define $\tau (\Pi)$ with the
  following grammar:
  \[ X \hspace{1em} : : = \hspace{1em} t \in \Pi \hspace{1em} | \hspace{1em} X
     \oplus Y \]
  
  
  Let $\Phi$ be a set of letters and $\Pi$ a set of primitive terms, and
  define the language $\mathcal{L}_{\tmop{SJL}} (\Phi, \Pi)$ as:
  \[ \phi \hspace{1em} : : = \hspace{1em} p \in \Phi \hspace{1em} |
     \hspace{1em} \circlearrowleft_X \hspace{1em} | \hspace{1em} \bot
     \hspace{1em} | \hspace{1em} \phi \rightarrow \psi \hspace{1em} |
     \hspace{1em} \Box_X \phi \hspace{1em} | \hspace{1em} X : \phi \]
  where $X \in \tau (\Pi)$
\end{definition}

\begin{definition}
  \label{justmodels}A {\tmstrong{simple justification model}} $\mathbbm{M}=
  \langle W, V, R, \mathcal{A} \rangle$ is a Kripke model with a valuation $V
  : \Phi \cup \{\circlearrowleft_X \hspace{1em} | \hspace{1em} X \in \tau
  (\Pi)\} \rightarrow 2^W$, an indexed relation $R : \tau (\Pi) \rightarrow
  2^{W \times W}$, along with a modified awareness function $\mathcal{A} : W
  \times \tau (\Pi) \rightarrow 2^{\mathcal{L}_{\tmop{SJL}} (\Phi, \Pi)}$. \
  In practice we will denote $\mathcal{A} (w, X)$ by a curried shorthand,
  namely $\mathcal{A}_w (X)$.
  
  
  
  The semantics for $\vDash$ have the following modifications:
  \begin{eqnarray*}
    \mathbbm{M}, w \vDash \Box_X \phi & \text{iff} & \text{for all $v \in W$
    where $w R_X v$ we have $\mathbbm{M}, v \vDash v$}\\
    \mathbbm{M}, w \vDash X : \phi & \text{iff} & \phi \in \mathcal{A}_w (X)
  \end{eqnarray*}
\end{definition}

\begin{definition}
  The following defines properties a simple justification model may make true:
  
  \begin{descriptiondash}
    \item[JCSQ] $\mathbbm{M}, w \vDash \Box_X \phi \text{ iff } \tmop{Th}
    (\mathbbm{M}) \cup \mathcal{A}_w (X) \vdash \phi$
    
    \item[JSND] $\mathbbm{M}, w \vDash \circlearrowleft_X \text{ iff
    $\mathbbm{M}, w \vDash \phi$ for all $\phi \in \mathcal{A}_w (X)$}$
    
    \item[CHOICE] $\mathcal{A}_w (X \oplus Y) = \mathcal{A}_w (X) \cup
    \mathcal{A}_w (Y)$
  \end{descriptiondash}
  
  as before, $\vdash$ is any sound logical consequence relation for $\vDash$
  with \tmtextbf{modus ponens} and \tmtextbf{reflection}
\end{definition}

The above semantics are similar to the ones given in previous sections. \
\tmtextbf{JCSQ} is the same as \tmtextbf{CSQ} from {\S}\ref{awarenesslogic},
only it is relativised to a knowledge base denoted by $X$ at $w$.  The
awareness logic of knowledge bases is special case of simple justification
logic where there is only one term.



Simplified justification logic is given in Table \ref{logic5}.  We assert
without proof that this is a conservative extension of the awareness logic in
{\S}\ref{awarenesslogic}.



\begin{table}[h]
  \begin{tabular}{ll}
    $\vdash \phi \rightarrow \psi \rightarrow \phi$ & \\
    $\vdash (\phi \rightarrow \psi \rightarrow \chi) \rightarrow (\phi
    \rightarrow \psi) \rightarrow \phi \rightarrow \chi$ & \\
    $\vdash ((\phi \rightarrow \bot) \rightarrow (\psi \rightarrow \bot))
    \rightarrow \psi \rightarrow \phi$ & \\
    $\vdash \Box_X (\phi \rightarrow \psi) \rightarrow \Box_X \phi \rightarrow
    \Box_X \psi$ & \\
    $\vdash (X : \phi) \rightarrow \Box_X \phi$ & \\
    $\vdash \circlearrowleft_X \rightarrow \Box_X \phi \rightarrow \phi$ & \\
    $\vdash (X : \phi) \rightarrow (Y : \phi) \rightarrow X \oplus Y : \phi$ &
    \\
    $\vdash (X \oplus Y : \phi) \rightarrow X : \phi$ & $\vdash (X \oplus Y :
    \phi) \rightarrow Y : \phi$\\
    $\vdash \circlearrowleft_X \rightarrow \circlearrowleft_Y \rightarrow
    \circlearrowleft_{X \oplus Y}$ & \\
    $\vdash \circlearrowleft_{X \oplus Y} \rightarrow \circlearrowleft_X$ &
    $\vdash \circlearrowleft_{X \oplus Y} \rightarrow \circlearrowleft_Y$\\
    $\vdash \Box_X \phi \rightarrow \Box_{X \oplus Y} \phi$ & \\
    $\vdash \Box_{X \oplus X} \phi \rightarrow \Box_X \phi$ & $\vdash \Box_{(X
    \oplus X) \oplus Y} \phi \rightarrow \Box_{X \oplus Y} \phi$ \\
    $\vdash \Box_{X \oplus Y} \phi \rightarrow \Box_{Y \oplus X} \phi$ &
    $\vdash \Box_{(X \oplus Y) \oplus Z} \phi \rightarrow \Box_{(Y \oplus X)
    \oplus Z} \phi$\\
    $\vdash \Box_{(X \oplus Y) \oplus Z} \phi \rightarrow \Box_{X \oplus (Y
    \oplus Z)} \phi$ & \\
    & \\
    \begin{tabular}{lll}
      $\frac{\vdash \phi \rightarrow \psi \hspace{4em} \vdash \phi}{\vdash
      \psi}$ & {\hspace{6em}} & $\frac{\vdash \phi}{\vdash \Box_X \phi}$
    \end{tabular} & 
  \end{tabular}
  \caption{\label{logic5}Simple Justification Logic}
\end{table}

\begin{theorem}
  \label{completeness5}Assuming an infinite store of proposition letters
  $\Phi$, SJL is sound and weakly complete for simple justification models
  making true \tmtextbf{JCSQ}, \tmtextbf{JSND} and \tmtextbf{CHOICE}
\end{theorem}

\begin{proof}
  As in the previous proofs, we only show completeness.  Assume that $\nvdash
  \psi$.  Let $\Xi \subseteq \tau (\Pi)$ be all of the subterms occurring in
  $\psi$; we assume that $\Xi$ is non-empty, since otherwise we may obtain the
  theorem via the usual completeness theorem for classical propositional
  logic. Let $\Sigma_0$ be the subformulae of $\psi$.  Define:
  
  \begin{align*}
    {\Sigma}_1 &
    {\assign}{\Sigma}_0{\cup}\{\Box_Y{\phi}{\hspace{1em}}|{\hspace{1em}}\Box_X{\phi}{\in}{\Sigma}_0{\hspace{1em}}\&{\hspace{1em}}Y{\in}{\Xi}\}\\
    {\Sigma}_2 &
    {\assign}{\Sigma}_1{\cup}\{{\neg}{\phi}{\hspace{1em}}|{\hspace{1em}}{\phi}{\in}{\Sigma}_1\}
  \end{align*}
  
  As in the proof of Theorem \ref{completeness2}, it is simple to verify that
  $\Sigma_2$ is closed under subformulae and single negations.  Let
  $\mathbbm{M} \assign \langle W, V, R, \mathcal{A} \rangle$ be the finite
  canonical model formed of maximally consistent subsets of $\Sigma_2$, where
  $V$ and $\mathcal{A}$ are defined as usual, and $R : \Xi \rightarrow 2^{W
  \times W}$ is defined by:
  \[ w R_X v \Longleftrightarrow \{\phi \hspace{1em} | \hspace{1em} \Box_X
     \phi \in w\} \subseteq v \]
  As in previous constructions we may prove the usual Lindenbaum and Truth
  Lemmas an use them obtain a world $w \in W$ where $\mathbbm{M}, w \nvDash
  \psi$.
  
  
  
  Next, define an operator $\twonotes : \tau (\Pi) \rightarrow 2^{\Pi}$ where:
  \[ \twonotes (X) \assign \{t \in \Pi \hspace{1em} | \hspace{1em} t \text{
     occurs in } X\} \]
  In other words, $\twonotes (X)$ is the set of atomic terms in $X$.  As in
  our construction for Theorem \ref{completeness1}, $\mathbbm{M}$ makes true
  certain properties, along with some new properties:
  \begin{enumeratenumeric}
    \item \label{fin}$W$ and $\mathcal{A}_v (X)$ are finite, and $\phi \in
    \mathcal{A}_v (X)$ only if $\phi \in \Sigma_2$
    
    \item if $\mathbbm{M}, v \vDash X : \phi$ then $\mathbbm{M}, v \vDash
    \Box_X \phi$
    
    \item \label{refl}if $\mathbbm{M}, v \vDash \circlearrowleft_X$ then $v
    R_X v$
    
    \item \label{sndness}For all $X \oplus Y \in \Xi$, $\mathbbm{M}, v \vDash
    \circlearrowleft_{X \oplus Y}$ if and only if $\mathbbm{M}, v \vDash
    \circlearrowleft_X$ and $\mathbbm{M}, v \vDash \circlearrowleft_Y$
    
    \item \label{union}For all $X \oplus Y \in \Xi$, $\mathcal{A}_v (X \oplus
    Y) = \mathcal{A}_v (X) \cup \mathcal{A}_v (Y)$
    
    \item \label{sub}For all $X, Y \in \Xi$, if $\twonotes (X) \subseteq
    \twonotes (Y)$ then $R_Y \subseteq R_X$
  \end{enumeratenumeric}
  As in our previous constructions, it is necessary to refine this model using
  bisimulations to achieve properties which are not modally definable.  In
  particular, we shall strengthen (\ref{refl}) to a biconditional and
  (\ref{sub}) to:
  
  \begin{center}
    \begin{tabular}{ll}
      \tmtextbf{6$'$.} & For all $X, Y \in \Xi$, $R_{X \oplus Y} = R_X \cap
      R_Y$
    \end{tabular}
  \end{center}
  
  
  
  We first strengthen (\ref{refl}) by constructing $\mathbbm{M}' \assign
  \langle W', V', R', \mathcal{A}' \rangle$, just as in Theorem
  \ref{completeness1}: {\hspace*{\fill}}
  \begin{itemizedot}
    \item $W' \assign W \uplus W$
    
    \item $V' (p) \assign \{v_l, v_r \hspace{1em} | \hspace{1em} v \in V
    (p)\}$
    
    \item $R'_X \assign \{(v_l, u_r), (v_r, u_l) \hspace{1em} | \hspace{1em} v
    R_X u\} \cup \{(v_l, v_l), (v_r, v_r) \hspace{1em} | \hspace{1em}
    \mathbbm{M}, v \vDash \circlearrowleft_X \}$
    
    \item $\mathcal{A}' (v_l, X) \assign \mathcal{A}' (v_r, X) \assign
    \mathcal{A} (v, X)$
  \end{itemizedot}
  The same bisimulation relation $Z$ previously given suffices.
  
  
  
  We next make a strengthened the model where (\ref{sub}$'$) holds.  Define
  $\mathbbm{M}'' \assign \langle W'', V'', R'', \mathcal{A}'' \rangle$ such
  that{\footnote{Let $\{i_X \hspace{1em} | \hspace{1em} X \in \Xi\}$ be the
  family of canonical injections into the coproduct $\uplus_{\Xi} W$. As per
  our previous convention, we use $v_X$ as shorthand for $i_X (v)$.}}:
  \begin{itemizedot}
    \item $W'' \assign \uplus_{\Xi} W'$
    
    \item $V'' (p) \assign \{v_X \hspace{1em} | \hspace{1em} X \in \Xi
    \hspace{1em} \& \hspace{1em} v \in V' (p)\}$
    
    \item $R''_X \assign \{(v_Y, u_Z) \hspace{1em} | \hspace{1em} Y, Z \in \Xi
    \hspace{1em} \& \hspace{1em} v R'_Z u \hspace{1em} \& \hspace{1em}
    \twonotes (X) \subseteq \twonotes (Z)\}$
    
    \item $\mathcal{A}'' (v_Y, X) \assign \mathcal{A}' (v, X)$
  \end{itemizedot}
  Note that this construction makes use of the assumption that $\Xi$ is
  non-empty.  Let $Z \assign \{(v, v_X) \hspace{1em} | \hspace{1em} X \in \Xi
  \hspace{1em} \& \hspace{1em} v \in W' \}$; it is straightforward to prove
  that $Z$ is a bisimulation between $\mathbbm{M}'$ and $\mathbbm{M}''$, for
  all relations corresponding to terms $X \in \Xi$. Hence $\mathbbm{M}'', w_X
  \nvDash \psi$ for some $w_X \in W''$.  In addition, $\mathbbm{M}''$
  inherits (\ref{fin}) through (\ref{sub}) from $\mathbbm{M}'$, as well as the
  converse of (\ref{refl}).  Next observe, forall $X \oplus Y \in \Xi$:
  \begin{eqnarray*}
    \text{$(v_V, u_Z) \in R''_X \cap R''_Y$} & \Longleftrightarrow & \exists Z
    \in \Xi . \text{$v R'_Z u \hspace{1em} \& \hspace{1em} \twonotes (X)
    \subseteq \twonotes (Z) \hspace{1em} \& \hspace{1em} \twonotes (Y)
    \subseteq \twonotes (Z)$}\\
    & \Longleftrightarrow & \exists Z \in \Xi . \text{$v R'_Z u \hspace{1em}
    \& \hspace{1em} \twonotes (X) \cup \twonotes (Y) \subseteq \twonotes
    (Z)$}\\
    & \Longleftrightarrow & \exists Z \in \Xi . \text{$v R'_Z u \hspace{1em}
    \& \hspace{1em} \twonotes (X \oplus Y) \subseteq \twonotes (Z)$}\\
    & \Longleftrightarrow &  \text{$(v_V, u_Z) \in R''_{X \oplus Y}$}
  \end{eqnarray*}
  We now turn to our final construction. As in previous constructions, let
  $\iota : W'' \hookrightarrow \Phi \backslash \Psi$ be an injection assigning
  fresh nominals to worlds.  Define $\mathbbm{M}''' \assign \langle W''',
  V''', R''', \mathcal{A}''' \rangle$ just as in the final construction in the
  proof of Theorem \ref{completeness1}, except that $R'''_X$ and
  $\mathcal{A}_v''' (X)$ are defined inductively as follows:
  
  
  
  \begin{center}
    \begin{tabular}{lll}
      $\bullet$ & $R'''_t \assign \left\{ \begin{array}{ll}
        R''_t & t \in \Xi\\
        W''' \times W''' & o / w
      \end{array} \right.$ & \\
      $\bullet$ & $R'''_{X \oplus Y} \assign R'''_X \cap R'''_Y$ & \\
      &  & \\
      $\bullet$ & $\mathcal{A}'''_v (t) \assign \{\phi \hspace{1em} | \phi \in
      \mathcal{A}''_v \} \cup \{\neg \iota (u) \hspace{1em} | \hspace{1em}
      \neg v R'''_t u\}$ & where $t \in \Pi$\\
      $\bullet$ & $\mathcal{A}'''_v (X \oplus Y) \assign \mathcal{A}'''_v (X)
      \cup \mathcal{A}'''_v (Y)$ & 
    \end{tabular}
  \end{center}
  
  
  
  
  
  Induction over complexity of subformulae $\phi$ of $\psi$ yields
  $\mathbbm{M}'', w \vDash \phi \Longleftrightarrow \mathbbm{M}''', w \vDash
  \phi$, so we know that there is some world $w \in W'''$ such that
  $\mathbbm{M}''', w \nvDash \psi$.  All that is left is to illustrate that
  $\mathbbm{M}'''$ has the properties we desire.
  
  
  
  First note that by definition, $\mathbbm{M}'''$ makes true
  \tmtextbf{CHOICE}.  Next, an induction argument on the complexity of a
  terms $X$, making essential use of (\ref{sub}$'$), yields:
  \[ u R_X''' v \Longleftrightarrow \mathbbm{M}''', v \vDash \bigwedge
     \mathcal{A}'''_v (X) \]
  With this, $\mathbbm{M}'''$ makes true \tmtextbf{JSND} since it inherits
  (\ref{refl}).  Finally, we note that we may repeat our previous arguments
  from Theorem \ref{completeness1} to prove \tmtextbf{JCSQ}.
\end{proof}
%%% Local Variables: 
%%% mode: latex
%%% TeX-master: "paper"
%%% End: 
