\begin{proof}
  Soundness is straightforward, so we will only address completeness.  Assume
  $\nvdash \psi$.  Consider the finite canonical model $\mathbbm{M}= \langle
  W, V, R, \mathcal{A} \rangle$ formed of maximally consistent sets of
  subformulae of $\psi$ (closed under single negations{\footnote{A discussion
  of this property may be found in {\cite[pg.
  243]{blackburn_modal_2001}}.}}), as per the finitary modal completeness
  proofs found in {\cite[chapter 4.8]{blackburn_modal_2001}} and
  {\cite[chapter 5]{boolos_logic_1995}}.  Typical presentations do not
  include awareness, however it is defined intuitively in this case:
  \[ \phi \in \mathcal{A}_w \Longleftrightarrow A : \phi \in w \]
  Since $\psi$ is a subformula of itself, we have $\mathbbm{M}, w \nvDash
  \psi$ for some world $w$ by finitary forms of the {\tmem{Lindenbaum Lemma}} and
  {\tmem{Truth Theorem}}.  Moreover, it is straightforward to verify that
  $\mathbbm{M}$ makes true the following properties:
  \begin{enumeratenumeric}
    \item $W$ and $\mathcal{A}_v$ are finite, and $\psi \in \mathcal{A}_v$
    only if $\psi$ is a (possibly negated) subformula of $\phi$
    \item \label{awareness0-prop2} if $\mathbbm{M}, v \vDash A : \psi$ then $\mathbbm{M}, v \vDash \Box
    \psi$ 
    \item $v R v$ for all worlds $v$
  \end{enumeratenumeric}
  
  % We next produce a bisimular{\footnote{The notion of bisimulation is
  % discussed at length in {\cite[chapter 2.2]{blackburn_modal_2001}}.  We
  % note that the basic definition and all of the usual theorems (modal
  % equivalence, Henessy-Milner, etc.) may be generalized to awareness models in
  % a straightforward fashion. }} model $\mathbbm{M}'$ which makes true (1), (2)
  % and a stronger from of (3):
  
  % {\hspace*{\fill}}\begin{tabular}{ll}
  %   \tmtextbf{3$'$.} & $\mathbbm{M}', v \vDash \circlearrowleft \iff v R v$
  % \end{tabular}{\hspace*{\fill}} 
  
  % To this end define $\mathbbm{M}' \assign \langle W', V', R', \mathcal{A}'
  % \rangle$ such that{\footnote{Throughout this article, we use $l$ and $r$ to
  % denote the two canonical injections associated with the coproduct $W \uplus
  % W$. We will use $v_l$ and $v_r$ as the shorthand for $l (v)$ and $r (v)$
  % respectively.}}: {\hspace*{\fill}}
  % \begin{itemizedot}
  %   \item $W' \assign W \uplus W$
    
  %   \item $V' (p) \assign \{v_l, v_r \  | \  v \in V
  %   (p)\}$
    
  %   \item $R' \assign \{(v_l, u_r), (v_r, u_l) \  | \  v R
  %   u\} \cup \{(v_l, v_l), (v_r, v_r) \  | \  \mathbbm{M},
  %   v \vDash \circlearrowleft\}$
    
  %   \item $\mathcal{A}' (v_l) \assign \mathcal{A}' (v_r) \assign \mathcal{A}
  %   (v)$
  % \end{itemizedot}
  
  
  % It is straightforward to verify that $\mathbbm{M}'$ makes true the desired
  % properties.  Let $Z \assign \{(v, v_l), (v, v_r) \  |
  % \  v \in W\}$.  then $Z$ is a bisimulation between $\mathbbm{M}$
  % and $\mathbbm{M}'$.  Therefore we know there is some $w \in W$ such that
  % $\mathbbm{M}', w_l \nvDash \psi$ and $\mathbbm{M}', w_r \nvDash \psi$.
  
  We next construct a model $\mathbbm{M}'$ which agrees with
  $\mathbbm{M}$ for all subformulae of $\psi$, but which also makes true
  \tmtextbf{CSQ}.  Let $\Lambda$ be the set of proposition
  letters occurring in $\psi$.  We will make use members of $\Phi \backslash
  \Lambda$ as {\tmem{nominals}}, as per the tradition in hybrid logic.  This
  is possible since both $W$ and $\Lambda$ are finite and $\Phi$ is infinite
  by hypothesis.  Let $\iota : W \uplus W \hookrightarrow \Phi \backslash
  \Lambda$ be an injection assigning a fresh nominal associated with each
  world in $W$.  Now define $\mathbbm{M}' \assign \langle W',
  V', R', \mathcal{A}' \rangle$, where:
  
  \begin{center}
    \begin{tabular}{lllll}
      $\bullet$ & $W' \assign W$ &  & $\bullet$ & $V' (p) \assign \left\{
      \begin{array}{ll}
        V (p) & p \in \Lambda\\
        \{v\} & p = \iota (v)\\
        \varnothing & o / w
      \end{array} \right.$\\
      $\bullet$ & $R' \assign R$ & {\hspace{3em}} & $\bullet$ &
      $\mathcal{A}'_v \assign \{\phi \  | \phi \in \mathcal{A}_v
      \} \cup \{\neg \iota (u) \  | \  \neg v R' u\}$
    \end{tabular}
  \end{center}
  
  By induction we have that for every subformula $\phi$ of $\psi$ that
  $\mathbbm{M}, v \vDash \phi$ if and only if $\mathbbm{M}', v \vDash \phi$,
  hence $\mathbbm{M}', w \nvDash \psi$.
    
  All that is left to show is that $\mathbbm{M}'$ is reflexive and
  makes true $\text{\tmtextbf{CSQ}}_{\vdash'}$ for some suitable logic $\vdash'$.  Since $\mathbb{M}$ was reflexive, $W' =
  W$ and $R' = R$, we know that $\mathbb{M}'$ is reflexive too.  We
  also have the following, additional properties:
  \begin{enumerateroman}
    \item \label{awareness0-one} $\mathcal{A}'_v$ is finite for all worlds $v$
    \item \label{awareness0-two} $\vdash$, the logic presented in Table \ref{logic0}, is
      relatively sound for $\mathbbm{M}'$
    \item\label{awareness0-three}  $v R' u$ if and only if $\mathbbm{M}', u \vDash \bigwedge
    \mathcal{A}'_v$
  \end{enumerateroman}

   We only prove \eqref{awareness0-three}.

  \begin{descriptiondash}
   \item[Left to Right]  Assume $v R' u$. We must show that if
   $\phi \in \mathcal{A}'_v$ then $\mathbb{M}', u \models \phi$.  If
   $\phi \in \mathcal{A}'_v$, either (a) $\phi \in \mathcal{A}_v$ or
   (b) $\phi = \neg \iota(x)$ for some world $x$ where $\neg v R x$. Note
   that by definition, $v R' u$ is equivalent to $v R u$.  So in the
   case of (a) we know have by property \eqref{awareness0-prop2} of
   our canonical model construction that $\mathbb{M}, u \models \phi$.
   Moreover, since $\phi$ is a subformula of $\psi$, we have that
   $\mathbb{M},u \models \phi \iff \mathbb{M}',u \models \phi$, which
   gives the desired result.  In the case of (b),
   suppose that $\mathbb{M},u \nmodels \phi$, then $\mathbb{M}',u
   \models \iota(x)$ for some $x$ such that $\neg v R' x$.  However,
   since $\iota$ is an injection, then by construction $\mathbb{M}',u
   \models \iota(x)$ if and only if $u = x$.  Hence $\neg v R' u$ after
   all, a contradiction.
  
   \item[Right to Left] It suffices to establish the contrapositive.  Assume $\neg v R' u$,
   we must find some formula $\phi \in \mathcal{A}'_v$ such that
   $\mathbb{M}, u \nmodels \phi$.  Evidently $\phi = \iota(u)$ suffices.
 \end{descriptiondash}
  
  We may now demonstrate $\textup{\tmtextbf{CSQ}}_{\vdash'}$ for some
  sound logic $\vdash'$.  Let $\vdash'$ be the logic presented in
  Table \ref{logic0}.  From \eqref{awareness0-one}, \eqref{awareness0-two}, \eqref{awareness0-three},
  and the \tmtextit{deduction theorem} for modal logic, we have the following
  line of reasoning:
  
  \begin{align*}
    Th({\mathbbm{M}}'){\cup}\mathcal{A}'_v{\vdash}{\phi}\ &
    {\Longleftrightarrow}\ Th({\mathbbm{M}}'){\vdash}\bigwedge\mathcal{A}'_v{\rightarrow}{\phi}\\
    &
    {\Longleftrightarrow}\ \bigwedge\mathcal{A}'_v{\rightarrow}{\phi}{\in}Th({\mathbbm{M}}')\\
    &
    {\Longleftrightarrow}\ {\mathbbm{M}}',u{\vDash}\bigwedge\mathcal{A}'_v{\rightarrow}{\phi}\text{
    for all $u \in W'$}\\
    & {\Longleftrightarrow}\ \text{for all $u \in W'$, if
    }{\mathbbm{M}}',u{\vDash}\bigwedge\mathcal{A}'_v\text{ then
    }{\mathbbm{M}}',u{\vDash}{\phi}\\
    & {\Longleftrightarrow}\ \text{for all $u \in W'$, if }v R 'u \text{ then
    }{\mathbbm{M}}',u{\vDash}{\phi}\\
    & {\Longleftrightarrow}\ {\mathbbm{M}}',v{\models}\Box{\phi}
  \end{align*}
\end{proof}
%%% Local Variables: 
%%% mode: latex
%%% TeX-master: "paper"
%%% End: 
