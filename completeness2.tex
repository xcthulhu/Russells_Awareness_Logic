\begin{proof}
  As before, soundness is trivial.  Completeness is based on a modification
  of techniques found in {\cite[chapter 5]{boolos_logic_1995}}.  Assume
  that $\nvdash \psi$ and let $\Sigma_0$ be the set of subformulae of $\psi$,
  and let $\Upsilon$ be the set of nominals occurring in $\psi$.  Define:
  
  \begin{align*}
    {\Sigma}_1 &
    {\assign}{\Sigma}_0{\cup}\{{\forall}{\phi}{\ }|{\ }\Box{\phi}{\in}{\Sigma}_0\}\\
    {\Sigma}_2 &
    {\assign}{\Sigma}_1{\cup}\{\text{@}_i{\phi}{\ },i{\rightarrow}{\phi}|{\ }{\phi}{\in}{\Sigma}_1{\ }\&{\ }i{\in}{\Upsilon}\}\\
    {\Sigma}_3 &
    {\assign}{\Sigma}_2{\cup}\{{\neg}{\phi}{\ }|{\ }{\phi}{\in}{\Sigma}_2\}
  \end{align*}
  
  It is easy to see that $\Sigma_3$ is finite and closed under subformulae and
  single negations. Let $\mathbbm{M}= \langle W, V, R_{\Box}, \sim_{\forall},
  \mathcal{A} \rangle$ be the finite canonical model formed of maximally
  consistent subsets of $\Sigma_3$, where everything is defined as usual,
  except $\sim_{\forall}$, which obeys the following specification:
  
  \begin{align*}
    w {\sim}_{{\forall}}v & {\Longleftrightarrow}(({\forall}{\phi}){\in}w
    {\Longleftrightarrow}({\forall}{\phi}){\in}v)
  \end{align*}
  
  This model makes true the following properties:
  \begin{enumerateroman}
    \item $\sim_{\forall}$ is an equivalence relation    
    \item $R_{\Box} \subseteq \sim_{\forall}$
    \item In each equivalence class specified by $\sim_{\forall}$, there is at
    most one world making true $i$ for all $i \in \Upsilon$
  \end{enumerateroman}
  As in our previous construction we have that there is some world $w$ where
  $\mathbbm{M}, w \nvDash \psi$.  Let $\mathbbm{M}' = \langle W', V',
  R_{\Box}', \sim_{\forall}', \mathcal{A}' \rangle$ be the submodel generated
  by $\{w\}$; we have that $\mathbbm{M}', w \nvDash \psi$ (see {\cite[chapter
  2.1]{blackburn_modal_2001}} for details on generated submodels).  In this
  model $\forall$ is a universal modality and either $\left\llbracket i
  \right\rrbracket^{\mathbbm{M}'} = \varnothing$ or $\left\llbracket i
  \right\rrbracket^{\mathbbm{M}'} =\{v\}${\footnote{As per the usual
  convention, here $\left\llbracket \phi \right\rrbracket^{\mathbbm{M}}$
  denotes $\{w \in W \  | \  \mathbbm{M}, w \vDash
  \phi\}$.  We will drop $\mathbbm{M}$ where it is unambiguous.}}. Since the
  store $\Psi$ of nominals is infinite and $\Upsilon$ is finite, we have that
  $\Psi \backslash \Upsilon$ is infinite, so there is some injection $\iota :
  W' \hookrightarrow \Psi \backslash \Upsilon$ which assigns a fresh nominal
  to every world.  Let $\mathbbm{M}'' = \langle W'', V'', R'', \mathcal{A}'',
  \ell \rangle$, where:
  \begin{center}
    \begin{tabular}{lllll}
      $\bullet$ & $W'' \assign W'$ &  & $\bullet$ & $V'' (p) \assign V' (p)$\\
      $\bullet$ & $R'' \assign R'_{\Box}$ & \ \  & $\bullet$ &
      $\mathcal{A}'' (v) \assign \{\phi \  | \  \phi \in
      \mathcal{A}' (v)\} \cup \{\neg \iota (u) \  | \ 
      \neg v R' u\}$\\
       $\bullet$ & \begin{tabular}{ll}
       $\ell (i) \assign \left\{ \begin{array}{ll}
          w & i \in \Upsilon \  \& \  \mathbbm{M}, w
          \vDash i\\
          \iota^{- 1} (i) & i \in \iota [W]\\
          \text{undefined} & \text{o/w}
        \end{array} \right.$
      \end{tabular} &  & $\bullet$ &  $R'' \assign R'_{\Box}$ 
    \end{tabular}
  \end{center}
  
  The the worlds is $\mathbbm{M}'$ and $\mathbbm{M}''$ agree on all
  subformulae of $\psi$, hence $\mathbbm{M}'', w \nvDash
  \psi$. Moreover, the logic in Table \ref{logic2} itself 
  is suitable for $\mathbbm{M}''$ and
  $\mathbbm{M}''$ makes true $\textup{\tmtextbf{CSQ}}_{\vdash}$.  From
  this it follows that we have completeness.
\end{proof}
%%% Local Variables: 
%%% mode: latex
%%% TeX-master: "paper"
%%% End: 
