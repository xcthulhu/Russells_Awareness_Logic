\begin{proof}
  Assume that $\nvdash \psi$, and define the finitary canonical model
  $\mathbbm{M}= \langle W, V, \mathcal{N}, \mathcal{A} \rangle$
  where:
  \begin{itemizedot}
    \item $W \assign \text{the maximally consistent sets of subformulae of
    $\psi$, closed under single negation}$
    \item $V (p) \assign \{v \in W \  | \  p \in v\}$
    % \item $\mathcal{K}_w \assign \{S \in 2^W \  | \  \{v
    % \in W \  | \  \phi \in v\} \subseteq S \  \&
    % \  K \phi \in w\}$
    \item $\mathcal{N}_w \assign \{S \in 2^W \  | \  \llbracket \phi
  \rrbracket \subseteq S \  \&
    \  \Box \phi \in w\}$
    \item $\mathcal{A}(w) \assign \{\{\phi\} \  | \  A :
    \phi \in w\}$
  \end{itemizedot}
  % Note that in this model, there are neighborhoods for both knowledge and
  % belief; more work is necessary to find a model based on this one which
  % conforms to the semantics in Definition \ref{neighborhoodmodels}.  For now,
  % assume that $K$ is just another modality governed by the neighborhoods in
  % $\mathcal{K}_w$.  In both neighborhood functions, 
  Here $\llbracket \phi \rrbracket$ denotes $\{ v \in W\ |\ \phi
\in v\}$.
  
  The proof of the finitary Lindenbaum Lemma is straightforward,
  although the Truth Theorem is a little less routine. The proof
  proceeds by induction.  
  We only show one direction for the step for $\Box \chi$.
  Assume $\mathbbm{M}, w \vDash \Box \chi$, where the Truth Lemma has been proven to
  hold for $\chi$; we have to show $\Box \chi \in w$.  There must be some
  $S \in \mathcal{N}(w)$ such that $\mathbb{M},v \models \chi$ for all
  $v \in S$. By construction there is some $\phi$ where $\Box \phi \in
  w$ such that $\llbracket \phi \rrbracket \subseteq S$.  We argue that $\vdash \phi
  \rightarrow \chi$; for if not then by the finitary Lindenbaum lemma there is
  some $u \in W$ where $\phi \in u$ and $\chi \nin u$.  Hence $u \in S$ and
  $\mathbbm{M}, u \nvDash \chi$, a contradiction.  Since $\vdash \phi
  \rightarrow \chi$ then we know that $\vdash \Box \phi \rightarrow \Box \chi$
  by our rules, which means that $\Box \chi \in w$ by maximality.
  
  By the Truth Theorem and the Finitary Lindenbaum lemma, we have that
  there is some $w \in W$ where
  $\mathbbm{M}, w \nvDash \psi$.  Moreover, $\mathbbm{M}$ makes true a number of
  properties:
  \begin{enumeratenumeric}
    \item $W$ and $\mathcal{A}_v$ are finite for all $v \in W$, and $\psi \in
    \bigcup \mathcal{A}(w)$ only if $\psi$ is a subformula of $\phi$   
   % \item $\mathcal{K}_w \subseteq \mathcal{N}_w$
%    \item If $S \in \mathcal{K}_w$ then $w \in S$
    \item $\mathbbm{M}, v \vDash A : \phi \rightarrow \Box \phi$ for all $v
    \in W$
    \item $W \in \mathcal{N}(w)$ for all $w \in W$
 %   \item $\mathcal{N}_w$ and $\mathcal{K}_w$ are closed under supersets
  \end{enumeratenumeric}
  
  % As in the proof of Theorem \ref{completeness1}, we use a bisimulation to
  % move to a model where property (3) is strengthened to (3$'$): for all $S \in
  % \mathcal{N}_w$, $w \in S$ if and only if $S \in \mathcal{K}'_w$.  We defer
  % the reader to {\cite{hansen_bisimulation_2007,pauly_bisimulation_1999}} for
  % details regarding bisimulation in neighborhood semantics.  Define
  % $\mathbbm{M}' \assign \langle W', V', \mathcal{K}', \mathcal{N}',
  % \mathcal{A}' \rangle$ such that:
  % \begin{itemizedot}
  %   \item $W' \assign W \uplus W$ wher.  $l, r$ its associated canonical
  %   injections and $\theta (v_l) \assign \theta (v_r) \assign v$ is a
  %   left-inverse of $l, r$
    
  %   \item $V' (p) \assign \{v_l, v_r \  | \  v \in V
  %   (p)\}$
    
  %   \item $\mathcal{K}' (v_i) \assign \{S \  | \  v_i \in
  %   S \  \& \  \theta [S] \in \mathcal{K_v} \}$ where $i =
  %   l, r$
    
  %   \item $\mathcal{N}' (v_i) \assign \mathcal{K}' (v_i) \cup \{S \ 
  %   | \  v_i \nin S \  \& \  \theta [S] \in
  %   \mathcal{N_v} \}$ where $i = l, r$
    
  %   \item $\mathcal{A}' (v_l) \assign \mathcal{A}' (v_r) \assign \mathcal{A}
  %   (v)$
  % \end{itemizedot}
  
  
  % If we let $Z \assign \{(v, v_r), (v, v_l) \  | \  v \in
  % W\}$, it is straightforward to verify that $Z$ is a neighborhood
  % bisimulation.  Hence $\mathbbm{M}, w_l \nvDash \psi$.  Along with (3$'$),
  % this model makes true (1), (2), (4).  In fact, in light of ($3'$), we can
  % see that this model does not need $\mathcal{K}'$, and conforms to the
  % semantics in Definition \ref{neighborhoodmodels}.{\hspace*{\fill}}
  
  
  Let $\Lambda$ be the proposition letters occurring in
  $\psi$ and define an injection $\iota : W \hookrightarrow \Phi \backslash
  \Lambda$, assigning a nominal to every world in $\mathbbm{M}'$.  Define
  $\mathbbm{M}' \assign \langle W', V', \mathcal{N}', \mathcal{A}'
  \rangle$ such that:
  
  \begin{center}
    \begin{tabular}{lllll}
      $\bullet$ & $W' \assign W$ &  & $\bullet$ & $V' (p) \assign \left\{
      \begin{array}{ll}
        V (p) & p \in \Lambda\\
        \{v\} & p = \iota (v)\\
        \varnothing & o / w
      \end{array} \right.$\\
      $\bullet$ & $\mathcal{N}'(w) \assign \mathcal{N}(w)$ & {\hspace{3em}} &  $\bullet$ & $\mathcal{A}'(w) \assign \{\delta_w S \  |
      \  S \in \mathcal{N}'(w) \}$
    \end{tabular}
  \end{center}
  Where $\delta_w S \assign \{ \phi \in
  \bigcup \mathcal{A}(w) \  | \ S \subseteq \left\llbracket \phi
  \right\rrbracket \} \cup \{\neg \iota (u) \  |
  \  u \nin S\}$.
  
  By induction, this model agrees with $\mathbbm{M}$ on all subformulae of $\psi$, hence
  $\mathbbm{M}', w \nvDash \phi$ for some $w\in W'$.
  
  We first establish that \textbf{NON-EMPTY} is true for
  $\mathbb{M}$. We must show for every $v \in W'$ that $\mathcal{A}'(v)
  \neq \varnothing$.  We know by construction that $W' \in
  \mathcal{N}'(v)$, since $W \in \mathcal{N}(u)$ for all of the worlds
  $u$ in model $\mathbb{M}$.  Hence by construction $\delta_v W' \in
  \mathcal{A}'(v)$ by definition.

  All that is left is to show $\text{\tmtextbf{NCSQ}}_\vdash$ holds,
  and we will use the logic we are verifying completeness for. The way we have
  defined $\mathbb{M}'$ gives rise to certain properties:
  \begin{enumerateroman}
    \item For all words $v$ and all $X \in \mathcal{A}'(v)$, $X$ is finite and  $X = \delta_v S$ for some $S \in \mathcal{N}'(v)$
    \item For all $v$ and all $S \in \mathcal{N}'(v)$, $u \in S$ if and only if
    $\mathbbm{M}', u \vDash \bigwedge \delta_v S$
    \item The logic presented in Table \ref{logic3} is sound for
    $\mathbbm{M}'$
  \end{enumerateroman}

  Using the deduction theorem for $\vdash$ along with these
  properties, we may establish  $\text{\tmtextbf{NCSQ}}_\vdash$ via a familiar line of reasoning:
  
  \begin{align*}
    \text{$\exists X \in \mathcal{A}'(v) \text{ s.t. } \tmop{Th}
    (\mathbbm{M}') \cup X \vdash \phi$} & {\Longleftrightarrow}\text{$\exists
    S \in \mathcal{N'}(v) \text{ s.t. } \tmop{Th} (\mathbbm{M}') \cup
    \delta_v S \vdash \phi$}\\
    & {\Longleftrightarrow}\text{$\exists S \in \mathcal{N}'(v) \text{ s.t. }
    \tmop{Th} (\mathbbm{M}') \vdash \bigwedge \delta_v S \rightarrow
    \phi$}\\
    & {\Longleftrightarrow}\text{$\exists S \in \mathcal{N}'(v) \text{ s.t. }
    \bigwedge \delta_v S \rightarrow \phi \in \tmop{Th} (\mathbbm{M}')$}\\
    % & {\Longleftrightarrow}\text{$\exists S \in \mathcal{N}'(v) \text{ s.t. }
    % \left\llbracket \bigwedge \delta_v S \right\rrbracket \subseteq
    % \left\llbracket \phi \right\rrbracket$}\\
    % & {\Longleftrightarrow}\text{$\exists S \in \mathcal{N}'(v) \text{ s.t.
    % $\forall u \in W$ if } \mathbbm{M}, u \vDash \bigwedge \delta_v S  \phi$}\\
    & {\Longleftrightarrow}\text{$\exists S \in \mathcal{N}'(v) \text{ s.t.
    $\forall u \in W$ if } \mathbbm{M}, u \vDash \bigwedge \delta_v S \text{
    then } \mathbbm{M}, u \vDash \phi$}\\
    & {\Longleftrightarrow}\text{$\exists S \in \mathcal{N}'(v) \text{ s.t. }
    \mathbbm{M}, u \vDash \phi \text{ for all } u \in S$}\\
    & {\Longleftrightarrow}{\mathbbm{M}},v{\vDash}\Box{\phi}
  \end{align*}
  This completes the proof.
\end{proof}
%%% Local Variables: 
%%% mode: latex
%%% TeX-master: "paper"
%%% End: 
